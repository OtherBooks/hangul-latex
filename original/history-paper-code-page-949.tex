%%%%%%%%%%%%%%%%%%%%%%%%%%%%%%%%%%%%%%%%%%%%%%%%%%%%%%%%%%%%%%%%%%%%%%%%%%%%%%%%%%%%%%%%%%%%%%
%
%
%              Çмú¿ø ÃѼ­ Æò¼³ ³í¹®
%
%                Åä¿¡Çø®Ã÷ ÀÛ¿ë¼Ò¿¡ ´ëÇÑ ºê¸®Áö ÀÌ·Ð
%                        ÀÌ ¿ì ¿µ
%                  ¼­¿ï´ëÇб³ ¼ö¸®°úÇкÎ
%
%                     2013³â 9¿ù 21ÀÏ
%
%%%%%%%%%%%%%%%%%%%%%%%%%%%%%%%%%%%%%%%%%%%%%%%%%%%%%%%%%%%%%%%%%%%%%%%%%%%%%%%%%%%%%%%%%%%%%%
\documentclass[12pt,a4paper,2sided]{article}
\usepackage{hangul}
\usepackage{amsmath}
\usepackage{amssymb}
\usepackage{graphicx}
\usepackage{amscd}
\usepackage{amsfonts}
%\usepackage{fancyhdr}

\topmargin 2mm \textwidth 140mm \textheight 230mm

\pagestyle{empty}

\newcommand{\mm}{\multimap}
\newcommand{\sub}{\subset}
\newcommand{\g}{\gamma}
\newcommand{\G}{\Gamma}
\newcommand{\ol}{\overline}
\newcommand{\ul}{\underline}
\newcommand{\ti}{\times}
\newcommand{\la}{\lambda}
\newcommand{\La}{\Lambda}
\newcommand{\bcup}{\bigcup}
\newcommand{\bcap}{\bigcap}
\newcommand{\De}{\Delta}
\newcommand{\de}{\delta}
\newcommand{\back}{\backslash}
\newcommand{\f}{\frak}
\newcommand{\ve}{\varepsilon}
\newcommand{\lan}{\langle}
\newcommand{\ran}{\rangle}
\newcommand{\wh}{\widehat}
\newcommand{\emp}{\emptyset}
\newcommand{\vs}{\vspace}
\newcommand{\hs}{\hspace}


\newtheorem{thm}{Theorem}[subsection]
\newtheorem{df}[thm]{Definition}
\newtheorem{pro}[thm]{Proposition}
\newtheorem{cor}[thm]{Corollary}
\newtheorem{ex}[thm] {Example}
\newtheorem{rema}[thm] {Remark}
\newtheorem{lem}[thm] {Lemma}
\newtheorem{prob}[thm]{Problem}


%%%%%%%%%%%%%%%%%%%%%%%%%%%%%%%%%%%%%%%%%%%%%%%%%%%%%%%%%%%%%%%%%%%%%%%%%%%%%%%%%%%%%%%%%%%%%%%


\begin{document}

\rightline{\large\it 173} \vspace{1.5cm}

\centerline{\Large{\bf Åä¿¡Çø®Ã÷ ÀÛ¿ë¼Ò¿¡ ´ëÇÑ ºê¸®Áö ÀÌ·Ð}}
\vspace{1.5 cm} \rightline{\large{\bf ÀÌ  ¿ì  ¿µ}\
{\small(¼­¿ï´ëÇб³ ±³¼ö)}}

\vspace{.5 cm}
\begin{abstract}
Toeplitz ÀÛ¿ë¼Ò´Â ¼öÇаú ¹°¸®ÇÐÀÇ ´Ù¾çÇÑ ¹®Á¦¿¡¼­ ÀÚ¿¬½º·´°Ô
¹ß»ýÇÑ´Ù. ¶ÇÇÑ ¾ÆÁ¤±Ô¿Í ºÎºÐÁ¤±Ô ÀÛ¿ë¼Ò ÀÌ·ÐÀº ¿À´Ã³¯ ±¤¹üÀ§ÇÏ°Ô Àß
¹ß´ÞµÈ ºÐ¾ß·Î ¼ºÀåÇÏ¿´À¸¸ç ÇÔ¼öÇؼ®ÇÐ, ÀÛ¿ë¼Ò ÀÌ·Ð, ¼ö¸®¹°¸®ÇÐ µîÀÇ
¸¹Àº ¹®Á¦¸¦ Ǫ´Â µ¥ Áö´ëÇÑ ±â¿©¸¦ ÇØ¿Ô´Ù. µû¶ó¼­, Toeplitz ÀÛ¿ë¼Ò¿¡
´ëÇÑ ¾ÆÁ¤±Ô¼º°ú ºÎºÐÁ¤±Ô¼ºÀ» »ó¼¼ÇÏ°Ô ¹¦»çÇÏ´Â °ÍÀº ¸Å¿ì Áß¿äÇÑ
¹®Á¦·Î ¶°¿Ã¶ú´Ù. ±×·± Àǹ̿¡¼­ ´ÙÀ½ Áú¹®Àº Èï¹Ì·Ó°í ¸Å·ÂÀûÀÌ´Ù: {\it
¾î¶² Toeplitz ÀÛ¿ë¼Ò°¡ ¾ÆÁ¤±ÔÀ̰ųª ºÎºÐÁ¤±ÔÀΰ¡?} Á¤±Ô¼º°ú
ºÎºÐÁ¤±Ô¼º¿¡ ´ëÇÑ Á¤È®ÇÑ °ü°è°¡ ±¤¹üÀ§ÇÏ°Ô ¿¬±¸µÇ¾î ¿Â ¹Ý¸é¿¡,
¾ÆÁ¤±Ô¼º ¾È¿¡¼­ÀÇ ºÎºÐÁ¤±Ô¼ºÀÇ À§Ä¡¿¡ °üÇÑ ¿¬±¸´Â ¿©ÀüÈ÷ ÀÌÇØ°¡
ºÎÁ·ÇÑ ÆíÀÌ´Ù. ¹«ÇÑÂ÷¿ø º¹¼Ò Hilbert (ȤÀº Banach) °ø°£ »óÀÇ
À¯°èÀÛ¿ë¼Ò¿¡ °üÇÑ ¾ÆÁ¤±Ô¼º°ú ºÎºÐÁ¤±Ô¼º »çÀÌÀÇ Æ´À» ¿¬±¸ÇÏ´Â ºÐ¾ß¸¦
`ºê¸®Áö ÀÌ·Ð'À̶ó ºÎ¸£´Âµ¥, ÀÌ ±Û¿¡¼­´Â Toeplitz ÀÛ¿ë¼Ò¿¡ ´ëÇÑ
ºê¸®Áö ÀÌ·ÐÀÇ ¹ßÀüÀ» ¼³¸íÇÑ´Ù. ÀÌ´Â 1970³â¿¡ P.\,R.\ Halmos°¡ Á¦½ÃÇÑ
¿­ °³ÀÇ ¹®Á¦ Áß ´ÙÀ½°ú °°Àº ´Ù¼¸¹ø° ¹®Á¦·ÎºÎÅÍ ¾ß±âµÈ´Ù: {\it ¸ðµç
ºÎºÐÁ¤±Ô Toeplitz ÀÛ¿ë¼Ò´Â Á¤±ÔÀ̰ųª Çؼ®ÀûÀΰ¡?}
\end{abstract}

\vspace{.5 cm} {\large\bf 1. ¼­·Ð}

\vspace{.5 cm}

ÀÛ¿ë¼Ò ÀÌ·Ð(Operator Theory)Àº ÇÔ¼ö °ø°£°ú ±× °ø°£ »ó¿¡¼­ ÀÛ¿ëÇÏ´Â
ÇÔ¼ö¸¦ ´Ù·ç´Â Çй®ÀÌ´Ù. ÀÛ¿ë¼Ò ÀÌ·ÐÀÇ ¿¬±¸´Â 1713³â¿¡ B.\ Taylor°¡
Áøµ¿ÇÏ´Â\linebreak

\vspace{.5 cm} \footnoterule

\vspace{.2 cm} {\footnotesize 2010 Mathematics Subject
Classification: 47B20, 47B35, 46J15, 15A83, 30H10, 47A20.

Key words and phrases: Hardy °ø°£, Toeplitz ÀÛ¿ë¼Ò, Hankel ÀÛ¿ë¼Ò,
Á¤±Ô¼º, ºÎºÐÁ¤±Ô¼º, ¾ÆÁ¤±Ô¼º, $k$-¾ÆÁ¤±Ô¼º.}

\newpage

\leftline{\large{\it 174}\ \ \ \small Çй® ¿¬±¸ÀÇ µ¿Çâ°ú ÀïÁ¡ --
¼öÇÐ}  \vspace{0.8cm}

\noindent ²öÀ» ³íÀÇÇϸ鼭 ±× ¾¾¾ÑÀÌ »Ñ·ÁÁ³´Âµ¥ ÀÌ ¶§°¡ NewtonÀÇ
`ÇÁ¸°Å°ÇǾÆ'°¡ Ãâ°£µÈÁö 25³âÂë Áö³­ ÈÄÀÌ´Ù. Çö´ë¼öÇÐÀÇ ¿ª»ç Ä¡°í ²Ï
À¯¼­°¡ ±íÀº Çй®À̶ó ÇÏ°Ú´Ù.

  ±× ÈÄ ¼öÇÐÀÌ ±Þ°ÝÈ÷ ¹ßÀüÇÏ°Ô µÇ°í 19¼¼±â ÃÊÀÇ
Sturm°ú Liouville¿¡ ÀÇÇؼ­ ¸Å¿ì °­·ÂÇÑ ÀÌ·ÐÀÌ ¸¸µé¾îÁ³À¸¸ç 1910³â¿¡
H.\ WeylÀÇ ÀÚ±â¼ö¹Ý ÀÛ¿ë¼Ò¿¡ °üÇÑ Áß¿äÇÑ ¹ß°ßµé°ú 20¼¼±â ÃÊÀÇ F.\
Volterra, M.\ Riesz, D.\ Hilbert, ±«ÆðÕÇÐÆÄ, Banach, Mazur,
Schauder, ±×¸®°í Æú¶õµåÀÇ ¸¹Àº ¼öÇÐÀڵ鿡 ÀÇÇÏ¿© dz¼ºÇÑ ¿­¸Å°¡
¿­·È´Ù. ¶Ç 1930³â°ú 1970³â´ë »çÀÌ¿¡ Gelfand, Krein, Naimark,
Kakutani, Kato, ÇÁ¶û½ºÀÇ Bourbaki ±×·ì, ·ç¸¶´Ï¾ÆÀÇ Colojoara, Foias,
¹Ì±¹ÀÇ Hille, Phillips, Friedrichs, von Neumann, Paley, Wiener µîÀÌ
ÀÌ ºÐ¾ßÀÇ ¿¬±¸¸¦ ´õ¿í dz¼ºÇÏ°Ô ÇÏ¿´´Ù. ¿À´Ã³¯, ÀÛ¿ë¼Ò ÀÌ·ÐÀÌ
Çö´ë¼öÇп¡¼­ Â÷ÁöÇÏ´Â ºñÁßÀº ½Ç·Î ´ë´ÜÇÏ´Ù. ¿ì¼± ¼öÇÐ ºÐ¾ß¿¡¼­¸¸
º¸´õ¶óµµ Æí¹ÌºÐ¹æÁ¤½Ä·Ð, ±Ù»çÀÌ·Ð, ÀûºÐ¹æÁ¤½Ä·Ð, È®·ü·Ð µî¿¡ ±íÀº
¿µÇâÀ» ÁÖ°í ÀÖÀ¸¸ç, ÀÀ¿ëºÐ¾ß¿¡¼­´Â ¾çÀÚ¿ªÇÐ, À¯Ã¼¿ªÇÐ, ÀüÀÚ±âÇÐ,
»ý¸í°øÇÐ µî ´Ù¾çÇÑ ºÐ¾ß·Î ÀÀ¿ëµÇ°í ÀÖ´Ù.

Toeplitz ÀÛ¿ë¼Ò¿¡ ´ëÇÏ¿© ²÷ÀÓ¾øÀÌ °ü½ÉÀÌ Áõ´ëµÇ´Â ÀÌÀ¯´Â Àû¾îµµ µÎ
°¡Áö°¡ ÀÖ´Ù. ù°´Â Toeplitz ÀÛ¿ë¼Ò°¡ ¹°¸®ÇÐ, È®·ü·Ð, Á¤º¸ ¹×
Á¦¾îÀÌ·Ð, ±× ¹ÛÀÇ ¿©·¯ ºÐ¾ßÀÇ ´Ù¾çÇÑ ¹®Á¦µé°ú ¹ÐÁ¢ÇÑ °ü·ÃÀÌ ÀÖ±â
¶§¹®À̸ç, µÑ°´Â Toeplitz ÀÛ¿ë¼Ò°¡ À¯°è ÀÛ¿ë¼ÒÀÇ ÁýÇÕ Áß °¡Àå Å«
ÁýÇÕÀ¸·Î¼­ ÀÛ¿ë¼Ò ÀÌ·Ð, ÇÔ¼ö·Ð, Banach ´ë¼ö ÀÌ·Ð µîÀÇ ÁÖÁ¦µé »çÀÌÀÇ
¸Å·ÂÀûÀÎ »óÈ£ ¿¬°ü¼ºÀ» º¸¿©ÁÖ°í dzºÎÇÑ ¿¹µéÀ» Á¦°øÇϱ⠶§¹®ÀÌ´Ù.

ÇÑÆí, 1950³â¿¡´Â Paul Halmos°¡ ¾ÆÁ¤±Ô¼º(hyponormality)°ú
ºÎºÐÁ¤±Ô¼º\newline\noindent(subnormality)À» µµÀÔÇÏ¿´´Âµ¥, ±× ÀÌÈÄ·Î
¾ÆÁ¤±Ô¼º°ú ºÎºÐÁ¤±Ô¼º ÀÌ·ÐÀº ÀÛ¿ë¼Ò·Ð¿¡¼­ ±¤¹üÀ§ÇÏ°Ô ¿¬±¸µÇ¾î ¸Å¿ì
ÈǸ¢ÇÑ ÀÌ·ÐÀ¸·Î ÀÚ¸®¸¦ Àâ¾Ò´Ù. ±×·± Àǹ̿¡¼­, ´ÙÀ½ ¹®Á¦´Â Èï¹Ì·Ó°í
¸Å·ÂÀûÀÌ´Ù:

\vspace{.2 cm}\centerline {\it ¾î¶² Toeplitz ÀÛ¿ë¼Ò°¡ ¾ÆÁ¤±ÔÀΰ¡? ¶Ç
ºÎºÐÁ¤±ÔÀΰ¡?}\vspace{.2 cm}

ÀÌ ±Û¿¡¼­´Â ÀÌ ¹®Á¦¿¡ ´ëÇÑ ¿ª»ç¿Í ¹ßÀü, °ü·ÃµÈ ¹ÌÇØ°á ¹®Á¦µéÀ»
Á¦½ÃÇÏ°íÀÚ ÇÑ´Ù.


%%%%%%%%%%%%%%%%%%%%%%%%%%%%%%%%%%%%%%%%%%%%%%%%%%%%%%%%%%%%%%%%%%%%%%%%%%%%%%%%%%%%%%%%%%%%%%%%%%%%%%

\vspace{.8 cm} {\large\bf 2. ¿¹ºñ ÀÌ·Ð}

\vspace{.5 cm}

ÀÌ  ±Û¿¡¼­´Â ¸ðµç Hilbert °ø°£Àº º¹¼Ò Hilbert °ø°£À¸·Î ÀÌÇØÇϸç
$\mathcal{H}$¸¦\linebreak

\rightline{\small À̿쿵: Åä¿¡Çø®Ã÷ ÀÛ¿ë¼Ò¿¡ ´ëÇÑ ºê¸®Áö ÀÌ·Ð\ \ \
\large\it 175}

\vspace{.8 cm}

\noindent  °¡»ê Hilbert °ø°£À¸·Î ¾à¼ÓÇÑ´Ù. ¶ÇÇÑ, $\mathcal{B(H)}$´Â
$\mathcal{H}$ »óÀÇ ¸ðµç À¯°è ¼±ÇüÀÛ¿ë¼ÒÀÇ ÁýÇÕÀ̶ó ÇÏ°í
$\mathcal{K(H)}$´Â ÄÄÆÑÆ® ÀÛ¿ë¼ÒµéÀÇ ÁýÇÕÀ̶ó ÇÏÀÚ. ¸¸ÀÏ
$T\in\mathcal{B(H)}$À̸é $T$ÀÇ ½ºÆåÆ®·³(spectrum) $\sigma(T)$°ú Á¡
½ºÆåÆ®·³(point spectrum) $\sigma_p(T)$Àº ´ÙÀ½°ú °°ÀÌ Á¤ÀǵȴÙ.
\begin{align*}
\sigma(T)&:=\{\lambda\in\mathbb{C}: T-\lambda\ \text{°¡ °¡¿ªÀÌ ¾Æ´Ï´Ù}\};\\
\sigma_p(T)&:=\{\lambda\in\mathbb{C}: T-\lambda\ \text{°¡ ÀÏ´ëÀÏÀÌ ¾Æ´Ï´Ù}\}.
\end{align*}



¸¸ÀÏ $U$°¡ $\ell^2$ »óÀÇ À̵¿ ÀÛ¿ë¼Ò(the unilateral shift)
$$
U:=
\begin{pmatrix}
0\\
1&0\\
&1&0\\&&1&0\\&&&\ddots&\ddots
\end{pmatrix}
$$
À̸é $\sigma_p(T)=\emptyset$ÀÌ´Ù. \ ¾çÀÇ À¯°è¼ö¿­
$\alpha:\alpha_0,\alpha_1,\cdots$ÀÌ ÁÖ¾îÁú ¶§,
$\ell^2(\mathbb{Z}_+)$ »óÀÇ $\alpha$¿¡ ´ëÇÑ °¡ÁßÀ̵¿ ÀÛ¿ë¼Ò (the
unilateral weighted shift) $W_\alpha$´Â ´ÙÀ½°ú °°ÀÌ Á¤ÀǵȴÙ.
$$
W_\alpha e_n:=\alpha_n e_{n+1}\quad \hbox{(¸ðµç $n\ge 0$)},
$$
¿©±â¿¡¼­ $\{e_n\}_{n=0}^\infty$Àº $\ell^2$¿¡ ´ëÇÑ
´ÜÀ§Á÷±³±âÀúÀÌ´Ù. ÀÛ¿ë¼Ò $T\in\mathcal{B(H)}$ÀÇ
¼ö¹ÝÀÛ¿ë¼Ò(adjoint)¸¦ $T^*$·Î ³ªÅ¸³½´Ù. $T$°¡ Á¤±Ô(normal)¶ó ÇÔÀº
$T^*T=TT^*$ÀÌ°í, ¾ÆÁ¤±Ô(hyponormal)¶ó ÇÔÀº
ÀڱⰡȯÀÚ(self-commutator) $[T^*,T]\equiv T^*T-TT^*\ge 0$ÀÌ°í,
ºÎºÐÁ¤±Ô(subnormal)¶ó ÇÔÀº $T=N\vert_{\mathcal{H}}$ (¿©±â¿¡¼­,
$N$Àº ¾î¶² Hilbert °ø°£ $\mathcal{K}\supseteq \mathcal{H}$ »ó¿¡¼­
Á¤±ÔÀÌ´Ù). ¸í¹éÈ÷, $T$°¡ ºÎºÐÁ¤±ÔÀ̸é $T$´Â ¾ÆÁ¤±ÔÀÌ´Ù.

´ÙÀ½Àº ¹®Çå¿¡¼­ Àß ¾Ë·ÁÁø ¾ÆÁ¤±Ô ÀÛ¿ë¼ÒÀÇ ±âº» ¼ºÁúÀÌ´Ù.

\vs{0.3cm}{\bf Proposition 2.0.1.} {\rm (Basic Properties of
Hyponormal Operators)\cite{Con2}}\label{pro3.4} {\sl Let
$T\in\mathcal{L(H)}$ be a hyponormal operator. Then we have:
\begin{itemize}
\item[\rm(a)] If $T\cong S$ then $S$ is also hyponormal;

\item[\rm(b)] $T-\lambda$ is hyponormal for every
$\lambda\in\mathbb{C}$;

\item[\rm(c)] If $T\mathcal{M}\subset\mathcal{M}$ then
$T\vert_\mathcal{M}$ is hyponormal; \end{itemize}}

\newpage
\leftline{\large{\it 176}\ \ \ \small Çй® ¿¬±¸ÀÇ µ¿Çâ°ú ÀïÁ¡ --
¼öÇÐ}

 \vspace{0.4cm}

{\sl \begin{itemize} \item[\rm(d)] $||T^*h||\le ||Th||$ for all $h$,
so that $\text{\rm ker}(T-\lambda)\subset \text{\rm
ker}(T-\lambda)^*$;
\item[\rm(e)] If $f$ and $g$ are eigenvectors corresponding to
distinct eigenvalues of $T$ then $f\perp g$; \item[\rm(f)] If
$\lambda\in\sigma_p(T)$ then $\text{\rm ker}\,(T-\lambda)$ reduces
$T$; \item[\rm(g)] If $T$ is invertible then $T^{-1}$ is hyponormal;
\item[\rm(h)] \text{\rm(Stampfli, 1962)} $||T^n||=||T||^n$, so that
$||T||=r(T)$ {\rm (}$r(\cdot)$ denotes spectral radius{\rm )};
\item[\rm(i)] $T$ is isoloid, i.e., $\text{\rm
iso}\,\sigma(T)\subset \sigma_p(T)$; \item[\rm(j)] If $\lambda\notin
\sigma(T)$ then $\text{\rm dist}(\lambda,\,
\sigma(T))=||(T-\lambda)^{-1}||^{-1}$. \item[\rm(k)] \text{\rm
(Berger-Shaw theorem)} If $T$ is cyclic then $\text{\rm
tr}\,[T^*,T]\le\frac{1}{\pi}\mu(\sigma(T))$; \item[\rm(l)] \text{\rm
(Putnam's Inequality)} $||\,[T^*,T]\,||\le\frac{1}{\pi}
\mu(\sigma(T))$.
\end{itemize} }
%\end{pro}



´ÙÀ½µµ ¹®Çå¿¡¼­ Àß ¾Ë·ÁÁø ºÎºÐÁ¤±ÔÀÇ Æ¯¼ºÈ­ÀÌ´Ù.

\vs{0.3cm}{\bf Proposition 2.0.2.} {\rm (A Characterization of
Subnormality) \cite{Con2}}\label{thm3.5} {\sl If
$T\in\mathcal{L(H)}$ then the following are equivalent:
\begin{itemize}
\item[\rm(a)] $T$ is subnormal;
\item[\rm(b)] {\rm (Bram-Halmos, 1955)}
$$
\begin{pmatrix}
I&T^*&\hdots& T^{*k}\\
T& T^*T& \hdots& T^{*k}T\\
\vdots&\vdots& \ddots& \vdots\\
T^k& T^*T^k& \hdots& T ^{*k}T^k
\end{pmatrix}
\ge 0\qquad\text{(all $k\ge 1$)}.
$$
\item[\rm(c)]
$$
\begin{pmatrix}
[T^*,T]&[T^{*2},T]&\hdots&[T^{*k},T]\\
[T^*,T^2]&[T^{*2},T^2]&\hdots&[T^{*k},T^2]\\
\vdots&\vdots&\hdots&\vdots\\
[T^*,T^k]&[T^{*2},T^k]&\hdots&[T^{*k},T^k]
\end{pmatrix}
\ge 0\qquad\text{(all $k\ge 1$)}.
$$  \end{itemize}}

\newpage

\rightline{\small À̿쿵: Åä¿¡Çø®Ã÷ ÀÛ¿ë¼Ò¿¡ ´ëÇÑ ºê¸®Áö ÀÌ·Ð\ \ \
\large\it 177} \vspace{.4 cm}

{\sl \begin{itemize}
\item[\rm(d)] {\rm (Embry, 1973)}  There is a positive
operator-valued measure $Q$ on some interval $[0,a]\subset
\mathbb{R}$ such that
$$
T^{*n}T^n=\int t^{2n} dQ(t)\quad\text{for all}\ n\ge 0.
$$
\end{itemize}}
%\end{pro}

À§ÀÇ Á¶°Ç (b) (ȤÀº µ¿Ä¡·Î¼­ Á¶°Ç (c))´Â ¾ÆÁ¤±Ô¿Í ºÎºÐÁ¤±Ô »çÀÌÀÇ
Æ´¿¡ ´ëÇÑ Ãøµµ¸¦ Á¦°øÇÑ´Ù. »ç½Ç, $k=1$¿¡ ´ëÇÑ ¾çÀÇ Á¶°Ç (b)°¡ $T$ÀÇ
¾ÆÁ¤±Ô¼ºÀÌ°í, ºÎºÐÁ¤±Ô´Â ¸ðµç $k$¿¡ ´ëÇÑ Á¶°Ç (b)ÀÌ´Ù. µû¶ó¼­, (b)¿¡
ÀÖ´Â $(k+1)\times (k+1)$ ÀÛ¿ë¼Ò Çà·ÄÀÌ ¾çÀÇ Çà·ÄÀÏ ¶§, $T$¸¦
$k$-¾ÆÁ¤±Ô¶ó°í Á¤ÀÇÇÒ ¼ö ÀÖ´Ù. ±×·¯¸é, Bram-Halmos Ư¼ºÈ­´Â $T$°¡
ºÎºÐÁ¤±ÔÀ̱â À§ÇÑ ÇÊ¿äÃæºÐÁ¶°ÇÀº ¸ðµç $k$¿¡ ´ëÇÏ¿© $T$°¡
$k$-¾ÆÁ¤±ÔÀÓÀ» ¸»ÇØÁØ´Ù.

°¡ÁßÀ̵¿ ÀÛ¿ë¼ÒÀÇ ºÎºÐÁ¤±Ô¼ºÀº Ãøµµ¿Í °ü·ÃµÈ ¸ð¸àÆ® ¹®Á¦·Î ÆÇÁ¤µÉ ¼ö ÀÖ´Ù.

\vs{0.2cm}{\bf Proposition 2.0.2.} {\rm (Berger's
Theorem)}\label{thm3.6} {\sl Let $T\equiv W_\alpha$ be a weighted
shift with weight sequence $\alpha\equiv\{\alpha_n\}$ and define the
moment of $T$ by
$$
\gamma_0:=1\quad\text{and}\quad \gamma_n:=\alpha_0^2\alpha_1^2\cdots \alpha_{n-1}^2\ (n\ge 1).
$$
Then $T$ is subnormal if and only if there exists a probability
measure $\nu$ on $[0,||T||^2]$ such that}
\begin{equation}\label{2.0.1}
\gamma_n=\int_{[0,||T||^2]} t^{n} d\nu(t)\quad (n\ge 1).
\end{equation}
%\end{pro}

\vs{0.2cm} ÇÑÆí,  $T\in\mathcal{B(H)}$°¡ {¾à $k$-¾ÆÁ¤±Ô}(weakly
$k$-hyponormal)¶ó ÇÔÀº ´ÙÀ½ÀÇ ÁýÇÕ
$$
LS((T,T^2,\cdots,T^k)):=\left\{\sum_{j=1}^k \alpha_jT^j: \alpha=
(\alpha_1,\cdots,\alpha_k)\in {\mathbb C}^k\right\}
$$
°¡ ¾ÆÁ¤±Ô ÀÛ¿ë¼Ò·Î ÀÌ·ç¾îÁö´Â °ÍÀÌ´Ù. ¸¸ÀÏ $k=2$À̸é, $T$¸¦
2Â÷-¾ÆÁ¤±Ô (quadratically hyponormal)¶ó ÇÏ°í, $p(T)$°¡ ¸ðµç ´ÙÇ×½Ä
$p\in{\mathbb C}[z]$¿¡ ´ëÇÏ¿© ¾ÆÁ¤±ÔÀ̸é $T$¸¦ ´ÙÇ×Á¤±Ô(polynomially
hyponormal)¶ó ÇÑ´Ù. ÀϹÝÀûÀ¸·Î, $k$-¾ÆÁ¤±Ô $\Rightarrow$ ¾à
$k$-¾ÆÁ¤±ÔÀÌ°í, ±× ¿ªÀº ¼º¸³ÇÏÁö ¾Ê´Â´Ù. ¾à-¾ÆÁ¤±Ô ÀÛ¿ë¼Ò´Â ¾ÆÁ¤±Ô¿Í
ºÎºÐÁ¤±Ô »çÀÌÀÇ Æ´À» ¿¬°áÇÏ´Â ½Ãµµ¿¡¼­ ¸¹Àº ¿¬±¸°¡ ÀÌ·ç¾îÁ³´Ù.

\newpage\leftline{\large{\it 178}\ \ \ \small Çй® ¿¬±¸ÀÇ
µ¿Çâ°ú ÀïÁ¡ -- ¼öÇÐ}

 \vspace{0.8cm}

ÀÌÁ¦, $P$¸¦ ${\bold L}^2(\mathbb T)\equiv{\bold L}^2$¿¡¼­ ${\bold
H}^2(\mathbb T)\equiv {\bold H}^2$ À§·ÎÀÇ Á÷±³»ç¿µÀ̶ó°í ÇÏÀÚ.
±×·¯¸é ÇÔ¼ö $\varphi\in{\bold L}^{\infty}(\mathbb T)\equiv {\bold
L}^{\infty}$¿¡ ´ëÇÏ¿©, ½É¹ú $\varphi$¸¦ °¡Áö´Â Toeplitz ÀÛ¿ë¼Ò
$T_\varphi$´Â ´ÙÀ½°ú °°ÀÌ Á¤ÀǵȴÙ:
$$
T_\varphi f=P(\varphi f)\quad\text{($f\in {\bold H}^2$)}.
$$
ÇÑÆí, $\{z^n : n=0,1,2,\cdots \}$ÀÌ ${\bold H}^{2}$ÀÇ Á¤±ÔÁ÷±³±âÀú¶ó
ÇÏ°í $\varphi\in{\bold L}^{\infty}$°¡ ´ÙÀ½°ú °°Àº Fourier °è¼ö¸¦
°¡Áø´Ù°í ÇÏÀÚ:
$$
\widehat\varphi(n)=\frac{1}{2\pi}\int_{0}^{2\pi} \varphi{\overline z}^n dt
$$
±×·¯¸é ±âÀú $\{z^n : n=0,1,2,\cdots\}$¿¡ °üÇÑ $T_\varphi$ÀÇ Çà·Ä $(a_{ij})$´Â ´ÙÀ½°ú °°´Ù:
$$
a_{ij}=(T_{\varphi}z^j,z^i)=\frac{1}{2\pi}\int_{0}^{2\pi} \varphi \overline{z}^{i-j} dt=\widehat\varphi(i-j).
$$
Áï, $T_\varphi$¿¡ ´ëÇÑ Çà·ÄÀº ´ë°¢¼±ÀÌ ¸ðµÎ ÀÏÁ¤ÇÑ Çà·ÄÀÌ´Ù:
$$
(a_{ij})=\begin{pmatrix}
c_0 & c_{-1}& c_{-2}& c_{-3}& \cdots \\
c_1 & c_{0}& c_{-1}& c_{-2}& \cdots \\
c_2 & c_{1}& c_{0}& c_{-1}& \cdots \\
c_3 & c_{2}& c_{1}& c_{0}& \cdots \\
\vdots& \ddots& \ddots& \ddots& \ddots
\end{pmatrix}, \quad \hbox{(¿©±â¼­, $c_{j}=\widehat\varphi(j)$)}:
$$
ÀÌ·¯ÇÑ Çà·ÄÀ» Toeplitz Çà·ÄÀ̶ó°í ÇÑ´Ù.



%%%%%%%%%%%%%%%%%%%%%%%%%%%%%%%%%%%%%%%%%%%%%%%%%%%%%%%%%%%%%%%%%%%%%%%%%%%%%%%%%%%%%%%%%%%%%%%%


\vspace{.8 cm} {\large\bf 3. Toeplitz ÀÛ¿ë¼ÒÀÇ ¾ÆÁ¤±Ô¼º}

\vspace{.5 cm}


1988³â¿¡, C. Cowen \cite{Cow3}ÀÇ ¿ì¾ÆÇÏ°í À¯¿ëÇÑ Á¤¸®°¡ ´ÜÀ§¿ø
${\mathbb T}\subset{\mathbb C}$ÀÇ Hardy °ø°£ $\bold H^2$ »óÀÇ
Toeplitz ÀÛ¿ë¼ÒÀÇ ¾ÆÁ¤±Ô¼ºÀ» ±×ÀÇ ½É¹ú $\varphi\in \bold
L^{\infty}$À» ½á¼­ Ư¼ºÈ­ÇÏ¿´´Ù. ÀÌ °á°ú´Â ÀÛ¿ë¼Ò ÀÌ·ÐÀ¸·ÎºÎÅÍ
¶°¿À¸¥ ´ë¼öÀû ¹®Á¦ -- Áï, $T_{\varphi}$°¡ ¾ÆÁ¤±ÔÀΰ¡? -- ÀÇ ÇØ´äÀÌ
ÇÔ¼ö $\varphi$¸¦ ¿¬±¸ÇÔÀ¸·Î½á °¡´ÉÇÏ°Ô µÇ¾ú´Ù. Á¤±Ô Toeplitz
ÀÛ¿ë¼Ò´Â 1960³â´ë¿¡ A.\ Brown°ú P.\,R.\ Halmos \cite{BH}°¡ ±× ½É¹úÀ»
½á¼­ Ư¼ºÈ­ÇÏ¿´´Ù. ±×·±µ¥ ´Ù¼Ò ³î¶ø°Ôµµ ¾ÆÁ¤±Ô Toeplitz ÀÛ¿ë¼ÒÀÇ
½É¹ú¿¡ ÀÇÇÑ Æ¯¼ºÈ­(CowenÀÇ Á¤¸®)°¡ ¹ß°ßµÇ±â±îÁö ¹«·Á 25³âÀÌ
Èê·¯°¬´Ù. ÀÌ´Â CowenÀÌ ±×ÀÇ Çؼ³³í¹® \cite{Cow2}¿¡¼­ ÁöÀûÇÏ¿´µíÀÌ
¾Æ¸¶µµ 1970³â´ë¿Í 1980³â´ë¿¡ ºÎºÐÁ¤±Ô\linebreak

\newpage
\rightline{\small À̿쿵: Åä¿¡Çø®Ã÷ ÀÛ¿ë¼Ò¿¡ ´ëÇÑ ºê¸®Áö ÀÌ·Ð\ \ \
\large\it 179}
\vspace{.8 cm}


\noindent  Toeplitz ÀÛ¿ë¼Ò¿¡ ´ëÇÑ ¶ß°Å¿î ¿¬±¸ ºÐÀ§±â°¡ ¾ÆÁ¤±ÔÀÇ
¿¬±¸¸¦ µÚ´Ê°Ô ÇßÀ» °ÍÀ̶ó´Â ÃßÃøÀÌ °¡´ÉÇÏ´Ù. CowenÀÇ Á¤¸®¿¡ ÀÇÇÑ
¾ÆÁ¤±Ô¼ºÀÇ Æ¯¼ºÈ­´Â $\bold H^{\infty}$ÀÇ ´ÜÀ§±¸¿¡¼­ÀÇ ¾î¶²
ÇÔ¼ö¹æÁ¤½ÄÀ» Ǫ´Â °ÍÀÌ´Ù. ±×·¯³ª ÀÌ´Â ÀÌ·ÐÀûÀ¸·Î °¡´ÉÇÏ´Ù´Â °ÍÀÌÁö
½ÇÁ¦·Î ÀÌ ÇÔ¼ö¹æÁ¤½ÄÀ» Ǫ´Â ¹®Á¦´Â ¹«Ã´ ¾î·Æ´Ù. ÂüÀ¸·Î, ½É¹ú
$\varphi$¿¡ °üÇÑ ¾î¶² Á¦ÇÑÀÌ ¾ø´Ù¸é $T_{\varphi}$ÀÇ ¾ÆÁ¤±Ô¼ºÀ»
$\varphi$ÀÇ Fourier °è¼ö·Î ³ªÅ¸³½´Ù´Â °ÍÀº ¾î¼¸é ºÒ°¡´ÉÇÒ ¼öµµ
ÀÖ´Ù. ÀÌ Àý¿¡¼­´Â ÀÌ ¿¬±¸ÀÇ ÃÖ±ÙÀÇ ¹ßÀü »óȲÀ» »ìÆ캻´Ù.

\vspace{.3 cm} {\bf 3.1. CowenÀÇ Á¤¸®} \vspace{.3 cm}

ÀÌ Àý¿¡¼­´Â Cowen Á¤¸®¸¦ Á¦½ÃÇÑ´Ù. CowenÀÇ Á¤¸®´Â Toeplitz ÀÛ¿ë¼ÒÀÇ
¾ÆÁ¤±Ô¼º¿¡ ´ëÇÑ ÀÛ¿ë¼Ò·ÐÀÇ ¹®Á¦¸¦ ±× ½É¹ú°ú °ü·ÃµÈ ÇÔ¼ö¹æÁ¤½ÄÀÇ Çظ¦
¹ß°ßÇÏ´Â ¹®Á¦·Î ¹Ù²Ù¾î ÁØ´Ù. ÀÌ¿Í °°Àº Á¢±Ù¹ýÀº Toeplitz ÀÛ¿ë¼Ò
¿¬±¸ÀÇ ¸¹Àº ³í¹®¿¡ µîÀåÇÏ¿´´Ù.

\vs{0.3cm} ´ÙÀ½ÀÇ µÎ °á°ú´Â \cite{BH}¿¡¼­ ¾Ë·ÁÁø »ç½ÇÀÌ´Ù.

\vs{0.3cm}{\bf Lemma 3.1.1.} {\sl A necessary and sufficient
condition that two Toeplitz operators commute is that either both be
analytic or both be co-analytic or one be a linear function of the
other.}

\vs{0.3cm}{\bf Theorem 3.1.2.} {\rm(Brown-Halmos) \cite{BH}} {\sl
Normal Toeplitz operators are translations and rotations of
hermitian Toeplitz operators, i.e.,
$$
T_\varphi\ \mbox{normal}\ \Longleftrightarrow\ \exists\ \alpha,\beta\in\mathbb{C},
\ \mbox{a real valued}\ \psi\in {\bold L}^{\infty}\ \mbox{s.t.}\
T_{\varphi}=\alpha T_{\psi} +\beta 1.
$$}

ÇÔ¼ö $\psi\in{\bold L}^{\infty}$¿¡ ´ëÇÏ¿©, ${\bold H}^2$ »óÀÇ Hankel
ÀÛ¿ë¼Ò $H_\psi$´Â ´ÙÀ½°ú °°ÀÌ Á¤ÀǵȴÙ:
$$
H_{\psi}f=J(I-P)(\psi f)\quad (f\in {\bold H}^{2}),
$$
¿©±â¿¡¼­ $J$\,´Â ¾Æ·¡¿Í °°ÀÌ Á¤ÀÇµÈ $({{\bold H}^2})^{\perp}$\,¿¡¼­
${\bold H}^2$ À§·ÎÀÇ À¯´ÏÅ͸® ÀÛ¿ë¼ÒÀÌ´Ù:
$$
J(z^{-n})=z^{n-1}\ (n\ge 1).
$$
\medskip
¸¸ÀÏ $\psi$°¡ Fourier ±Þ¼ö $\psi:=\sum_{n=-\infty}^{\infty}a_n
z^n$·Î Ç¥ÇöµÇ¸é, $H_\psi$ÀÇ Çà·ÄÀº ´ÙÀ½°ú °°ÀÌ ÁÖ¾îÁø´Ù:

\bigskip\bigskip

\newpage\leftline{\large{\it 180}\ \ \ \small Çй® ¿¬±¸ÀÇ
µ¿Çâ°ú ÀïÁ¡ -- ¼öÇÐ}  \vspace{0.3cm}

$$
H_\psi \equiv
\begin{pmatrix}
a_{-1}&a_{-2}&a_{-3}&\cdots\\
a_{-2}&a_{-3}& & \\
a_{-3}& &\ddots&  \\
\vdots& & &\ddots \\
\end{pmatrix}.
$$

´ÙÀ½Àº Hankel ÀÛ¿ë¼ÒÀÇ ±âº» ¼ºÁúµéÀÌ´Ù.
\begin{itemize}
\item[1.]  $H_{\psi}^*=H_{{\psi}^*}$;
\item[2.]  $H_{\psi}U=U^*H_{\psi}$ ($U$´Â À̵¿ÀÛ¿ë¼Ò);
\item[3.]  ${\rm Ker}H_{\psi}=\{0\}$ ¶Ç´Â $\theta{\bold H}^2$ (¾î¶² ³»Àû(inner)ÇÔ¼ö $\theta$) (Beurling's theorem¿¡ ÀÇÇØ);
\item[4.]  $T_{\varphi\psi}-T_{\varphi}T_{\psi}=H_{\overline{\varphi}}^*H_{\psi}$;
\item[5.]  $H_{\varphi}T_{h}=H_{\varphi h}=T_{h^*}^*H_{\varphi}\ (h\in{\bold H}^{\infty})$.
\end{itemize}

1988³â¿¡ Carl CowenÀÌ Toeplitz ÀÛ¿ë¼Ò $T_\varphi$ÀÇ ¾ÆÁ¤±Ô¼ºÀ» ½É¹ú
$\varphi$¿¡ ÀÇÇØ Æ¯¼ºÈ­ÇÏ´Â µ¥ ¼º°øÇÏ¿´´Ù.

\vs{0.2cm}{\bf Theorem 3.1.3.} (Cowen's Theorem)\ [7]\ {\sl If
$\varphi\in{\bold L}^{\infty}$ is such that \linebreak
$\varphi=\overline{g} +f$ {\rm (}$f,g\in {\bold H}^2${\rm )}, then
$$
T_\varphi\ \mbox{is hyponormal}\ \Longleftrightarrow\ g=c+T_{\overline{h}}f
$$
for some constant $c$ and some $h\in{\bold
H}^{\infty}(\mathbb{D})$ with $||h||_{\infty}\le 1$.}
%\end{thm}


\vs{0.2cm}{\bf Theorem 3.1.4.} (Nakazi-Takahashi Variation) [23]
{\sl For $\varphi\in{\bold L}^{\infty}$, put
$$
\mathcal{E} (\varphi):=\{k\in{\bold H}^{\infty}:||k||_{\infty}\le 1\
\mbox{and}\ \varphi-k{\overline{\varphi}}\in{\bold H}^{\infty}\}.
$$
Then $T_\varphi$ is hyponormal if and only if} $\mathcal{E}
(\varphi)\neq\varnothing$.
%\end{thm}


\vspace{.3 cm}

{\bf 3.2. »ï°¢´ÙÇ× ½É¹úÀÇ °æ¿ì} \vspace{.3 cm}

ÀÌ ºÐÀý¿¡¼­´Â »ï°¢´ÙÇ× ½É¹úÀ» °¡Áø Toeplitz ÀÛ¿ë¼ÒÀÇ ¾ÆÁ¤±Ô¼ºÀ»
»ìÆ캻´Ù. À̸¦ À§ÇÏ¿©  ÆØâÀÌ·Ð(dilation theory)À» ¸ÕÀú »ìÆ캻´Ù.
¸¸ÀÏ $B=\begin{pmatrix} A&*\\ *&* \end{pmatrix}$À̸é, $B$¸¦ $A$ÀÇ
ÆØâÀ̶ó ÇÏ°í, $A$¸¦ $B$ÀÇ Ãà¼Ò¶ó°í ÇÑ´Ù. ¸ðµç ¼öÃà \linebreak

\rightline{\small À̿쿵: Åä¿¡Çø®Ã÷ ÀÛ¿ë¼Ò¿¡ ´ëÇÑ ºê¸®Áö ÀÌ·Ð\ \ \
\large\it 181}

\vspace{.8 cm} \noindent ÀÛ¿ë¼Ò(contraction)´Â À¯´ÏÅ͸® ÆØâÀ»
°¡Áø´Ù´Â °ÍÀÌ Àß ¾Ë·ÁÁ® ÀÖ´Ù. ½ÇÁ¦·Î, $||A||\le 1$À̸é,
$$
B\equiv
\begin{pmatrix} A&(I-AA^*)^{\frac{1}{2}}\\(I-A^*A)^{\frac{1}{2}}&-A^*\end{pmatrix}
$$
´Â À¯´ÏÅ͸®ÀÌ´Ù. ÇÑÆí, ¸ðµç ÀÚ¿¬¼ö $n$¿¡ ´ëÇÏ¿© $B^n$ÀÌ $A^n$ÀÇ
ÆØâÀ̸é, ÀÛ¿ë¼Ò $B$¸¦ $A$ÀÇ °­ÆØâ(strong dilation)À̶ó°í ÇÑ´Ù.
µû¶ó¼­ ¸¸ÀÏ $B$°¡ $A$ÀÇ °­ÆØâÀ̸é, $B$´Â $B=\begin{pmatrix} A &0\\
*&*\end{pmatrix}$ÀÇ ²Ã·Î ³ªÅ¸³­´Ù. Á¾Á¾ $B$¸¦ $A$ÀÇ
¸®ÇÁÆÃ(lifting)À̶ó ÇÏ°í, $A$°¡ $B$·Î ½Â°­µÈ´Ù°í ¸»ÇÑ´Ù. ¸ðµç ¼öÃà
ÀÛ¿ë¼Ò°¡ µîÀå(isometric) °­ÆØâÀ» °¡Áø´Ù´Â °ÍÀÌ ¾Ë·ÁÁ® ÀÖ´Ù. »ç½Ç
¼öÃà ÀÛ¿ë¼Ò $A$ÀÇ ÃÖ¼Ò µîÀå °­ÆØâÀº ´ÙÀ½°ú °°ÀÌ ÁÖ¾îÁø´Ù:
$$
B\equiv
\begin{pmatrix}A&0&0&0&\cdots\\
(I-A^*A)^{\frac{1}{2}}&0&0&0&\cdots\\
0&I&0&0&\cdots\\
0&0&I&0&\cdots\\
\vdots&\vdots&\vdots&\ddots&
\end{pmatrix}.
$$

±×·¯¸é  ´ÙÀ½ÀÇ À¯¸íÇÑ Á¤¸®¸¦ ¾òÀ» ¼ö ÀÖ´Ù.


\vs{0.2cm}{\bf Theorem 3.2.1.} (Commutant Lifting Theorem) {\rm
\cite[p.658]{GGK}} {\sl  Let $A$ be a contraction and $T$ be a
minimal isometric dilation of $A$. If $BA=AB$ then there exists a
dilation $S$ of $B$ such that}
$$
S=\begin{pmatrix} B &0\\ *&*\end{pmatrix},\quad ST=TS,\quad \mbox{and}\quad ||S||=||B||.
$$
%\end{thm}


ÀÌÁ¦ ´ÙÀ½ÀÇ º¸°£¹®Á¦(interpolation problem) -- ¼ÒÀ§ Carath\'
eodory-Schur º¸°£¹®Á¦·Î ºÒ¸² -- (CSIP)¸¦ »ìÆ캸ÀÚ. º¹¼Ò¼ö $c_0,
\cdots, c_{N-1}$ÀÌ ÁÖ¾îÁú ¶§, ´ÙÀ½À» ¸¸Á·ÇÏ´Â $\mathbb{D}$ »óÀÇ
Çؼ®ÇÔ¼ö $\varphi$¸¦ ã¾Æ¶ó:
\begin{itemize}
\item[(i)] $\widehat{\varphi}(j)=c_j\ (j=0,\cdots, N-1)$;
\item[(ii)] $||\varphi||_{\infty}\le 1$.
\end{itemize}

±×·¯¸é CSIPÀÇ ÇØ´Â ´ÙÀ½°ú °°ÀÌ ÁÖ¾îÁø´Ù.

\newpage\leftline{\large{\it 182}\ \ \ \small Çй® ¿¬±¸ÀÇ
µ¿Çâ°ú ÀïÁ¡ -- ¼öÇÐ}

\vspace{0.8cm}



{\bf Theorem 3.2.2.} {\rm \cite{FF}} {\sl
$$
\mbox{CSIP is solvable}\ \Longleftrightarrow\ C\equiv
\begin{pmatrix}
c_0\\
c_1&c_0&&\mbox{\rm{\Huge{O}}}\\
c_2&c_1&c_0\\
\vdots&\vdots&\ddots&\ddots\\
c_{N-1}&c_{N-2}&\cdots&c_1&c_0
\end{pmatrix}\ \mbox{is a contraction.}
$$
Moreover, if $\varphi$ is a solution if and only if  $T_\varphi$
is a contractive lifting of $C$ which commutes with the unilateral
shift.}
%\end{thm}

\vs{0.3cm} ÀÌÁ¦ $\varphi$°¡ ´ÙÀ½°ú °°Àº »ï°¢´ÙÇ×ÇÔ¼ö¶ó°í ÇÏÀÚ:
$$
\varphi(z)=\sum_{n=-N}^{N} a_{n}z^n\ (a_{N}\neq 0).
$$
ÇÔ¼ö $k\in{\bold H}^{\infty}$°¡ $\varphi -k\overline{\varphi}\in {\bold H}^{\infty}$
¸¦ ¸¸Á·Çϸé, $k$´Â ´ÙÀ½À» ¸¸Á·ÇØ¾ß ÇÑ´Ù.
\begin{equation}\label{3.2.1}
k\sum_{n=1}^{N}\overline{a_n} z^{-n} -\sum_{n=1}^{N}a_{-n}z^{-n}\in{\bold H}^{\infty}.
\end{equation}
½Ä (\ref{3.2.1})·ÎºÎÅÍ ¿ì¸®´Â Fourier °è¼ö $\widehat{k}(0),\cdots
,\widehat{k}(N-1)$°¡ $\widehat{k}(n)=c_n\ (n=0,1,\cdots,N-1)$À»
¸¸Á·Çϵµ·Ï ÇÏ´Â °ÍÀ» ±¸ÇÒ ¼ö ÀÖ´Ù. ¿©±â¿¡¼­,
$c_0,c_1,\cdots,\linebreak c_{N-1}$Àº $\varphi$ÀÇ °è¼ö·ÎºÎÅÍ
À¯ÀÏÇÏ°Ô ´ÙÀ½ °ü°è½ÄÀ» ½á¼­ ¾ò¾îÁø´Ù.
\begin{equation}\label{3.2.2}
\begin{pmatrix}c_0\\c_1\\ \vdots\\ \vdots\\c_{N-1}\end{pmatrix}=
\begin{pmatrix}\overline{a_1}&\overline{a_2}&\overline{a_3}&\cdots&\overline{a_N}\\
\overline{a_2}&\overline{a_3}&\hdots&\cdot\\
\overline{a_3}&\hdots&\cdots\\
\vdots&\cdots & &\mbox{{\Huge{O}}}\\
\overline{a_N}
\end{pmatrix}^{-1}
\begin{pmatrix}
a_{-1}\\ a_{-2}\\ \vdots\\ \vdots\\ a_{-N}
\end{pmatrix}.
\end{equation}
µû¶ó¼­, ¸¸ÀÏ $k(z)=\sum_{j=0}^{\infty} c_{j}z^j$°¡ ${\bold
H}^{\infty}$¿¡ ¼ÓÇÏ¸é ´ÙÀ½ÀÌ ¼º¸³ÇÑ´Ù.
$$
\varphi -k\overline{\varphi}\in {\bold H}^{\infty}\ \Longleftrightarrow
\ c_0,c_1,\cdots ,c_{N-1}\mbox{ÀÌ (\ref{3.2.2})¿¡ ÀÇÇØ ÁÖ¾îÁø´Ù}.
$$
±×·¯¸é  CowenÀÇ Á¤¸®¿¡ ÀÇÇØ, ¸¸ÀÏ $ c_0,c_1,\cdots ,c_{N-1}$ÀÌ
(\ref{3.2.2})À¸·Î ÁÖ¾îÁö¸é,\linebreak $T_\varphi$ÀÇ ¾ÆÁ¤±Ô¼ºÀº
´ÙÀ½À» ¸¸Á·ÇÏ´Â ÇÔ¼ö $k\in{\bold H}^{\infty}$ÀÇ Á¸Àç¿Í µ¿Ä¡ÀÌ´Ù:
$$
\begin{cases}\widehat{k}(j)=c_j\ (j=0,\cdots , N-1)\\||k||_{\infty}\le 1
\end{cases}
$$


\newpage\rightline{\small À̿쿵: Åä¿¡Çø®Ã÷ ÀÛ¿ë¼Ò¿¡ ´ëÇÑ ºê¸®Áö ÀÌ·Ð\ \ \
\large\it 183} \vspace{.8 cm}


ÀÌ´Â Á¤È®È÷ CSIP¿Í °°Àº ¹®Á¦ÀÌ´Ù. µû¶ó¼­ ´ÙÀ½À» Áï½Ã ¾ò´Â´Ù.


\vs{0.2cm}{\bf Theorem 3.2.3.} {\sl If $\varphi(z)=\sum_{n=-N}^{N}
a_{n}z^n$, where $a_N\neq 0$ and if $c_0,c_1,\cdots,\linebreak
c_{N-1}$ are given by (\ref{3.2.2}) then
$$
T_\varphi\ \mbox{is hyponormal}\ \Longleftrightarrow\ C\equiv
\begin{pmatrix}
c_0\\
c_1&c_0&&\mbox{\rm{\Huge{O}}}\\
c_2&c_1&c_0\\
\vdots&\vdots&\ddots&\ddots\\
c_{N-1}&c_{N-2}&\cdots&c_1&c_0
\end{pmatrix}\ \mbox{is a contraction.}
$$}
%\end{thm}


%%%%%%%%%%%%%%%%%%%%%%%%%%%%%%%%%%%%%%%%%%%%%%%%%%%%%%%%%%%%%%%%%%%%%%%%%%%%%%%%%%%%%%%%%%%%%%%%%%%%%%
\vs{0.3cm} { \bf 3.3 À¯¸®ÇÔ¼ö ½É¹úÀÇ °æ¿ì}

\vs{0.3cm} ÇÔ¼ö $\varphi\in \bold L^\infty$°¡ À¯°èÇü(bounded
type)À̶ó ÇÔÀº $\bold H^\infty (\mathbb{D})$ »óÀÇ ´ÙÀ½À» ¸¸Á·ÇÏ´Â
ÇÔ¼ö $\psi_1, \psi_2$°¡ Á¸ÀçÇÒ ¶§ÀÌ´Ù:
$$
\varphi(z)=\frac{\psi_1(z)}{\psi_2(z)}\quad\hbox{(°ÅÀÇ ¸ðµç $z\in\mathbb{T}$¿¡ ´ëÇÏ¿©)}
$$
¸í¹éÈ÷ À¯¸®ÇÔ¼ö´Â À¯°èÇüÀÌ´Ù. ÀÌÁ¦, $\theta$°¡ ³»ÀûÇÔ¼ö(inner
function)¶ó ÇÏÀÚ. ¸¸ÀÏ $\theta$°¡ ´ÙÀ½ÀÇ À¯ÇÑ Blaschke °ö(finite
Blaschke product)À̸é,
$$
\theta(z)=e^{i\xi}\prod_{j=1}^n
\frac{z-\beta_j}{1-\overline{\beta_j}z}\quad \hbox{($|\beta_j|<1$,
$j=1,\cdots,n$)},
$$
$\theta$ÀÇ Â÷¼ö(degree) $\text{deg}(\theta)$´Â °³ ´ÜÀ§¿øÆÇ $\mathbb
D$ ¾ÈÀÇ ±ÙÀÇ °³¼ö·Î Á¤ÀǵǸç, ±×·¸Áö ¾ÊÀ¸¸é ¹«ÇÑ´ë·Î Á¤ÀÇÇÑ´Ù. ÀÌÁ¦,
³»ÀûÇÔ¼ö $\theta$¿¡ ´ëÇÏ¿© ´ÙÀ½°ú °°ÀÌ ¾²ÀÚ:
$$
\mathcal H(\theta):=\bold H^2 \ominus \theta \bold H^2.
$$

±×·¯¸é, $f\in \bold H^2$¿¡ ´ëÇÏ¿© ´ÙÀ½ÀÌ ¼º¸³ÇÑ´Ù.

\begin{align*}
\langle [T_\varphi^*,T_\varphi]f,f\rangle
=||T_\varphi f||^2-||T_{\overline{\varphi}}f||^2
&=||\varphi f||^2-||H_\varphi f||^2-(||\overline{\varphi} f||^2-||H_{\overline{\varphi}} f||^2)\\
&=||H_{\overline\varphi}f||^2-||H_\varphi f||^2.
\end{align*}

µû¶ó¼­ ´ÙÀ½À» ¾ò´Â´Ù.
$$
T_\varphi\ \text{hyponormal}\ \Longleftrightarrow\
||H_{\overline\varphi} f||\ge ||H_\varphi f||\quad (f\in \bold H^2).
$$
\newpage\leftline{\large{\it 184}\ \ \ \small Çй® ¿¬±¸ÀÇ
µ¿Çâ°ú ÀïÁ¡ -- ¼öÇÐ}


\vspace{0.8cm}

ÀÌÁ¦ $\varphi = \overline{g}+ f \in \bold L^{\infty}$ ($f,g\in\bold
H^2$)¶ó ÇÏÀÚ. ±×·¯¸é $H_\varphi U=U^*H_\varphi$
($U=$À̵¿ÀÛ¿ë¼Ò)À̹ǷÎ, Beurling Á¤¸®·ÎºÎÅÍ ´ÙÀ½À» ¾ò´Â´Ù.
$$
\text{ker}\, H_{\overline f}=\theta_0 \bold H^2\quad\text{and}\quad \text{ker}\,H_{\overline g}
=\theta_1 \bold H^2\quad\text{(¾î¶² ³»ÀûÇÔ¼ö $\theta_0, \theta_1$¿¡ ´ëÇÏ¿©)}.
$$
¸¸ÀÏ $T_\varphi$°¡ ¾ÆÁ¤±ÔÀ̸é, $||H_{\overline f} h||\ge ||H_{\overline g} h||$
($h\in \bold H^2$)À̹ǷΠ´ÙÀ½À» ¾ò´Â´Ù.
\begin{equation}\label{3.2.4}
\theta_0 \bold H^2=\text{ker}\, H_{\overline f}\subset \text{ker}\, H_{\overline g}=\theta_1 \bold H^2.
\end{equation}
ÀÌ´Â $\theta_1$°¡ $\theta_0$À» ³ª´«´Ù´Â °ÍÀ» ÀǹÌÇϸç, µû¶ó¼­
$\theta_0=\theta_1\theta_2$ ($\theta_2$´Â ³»ÀûÇÔ¼ö). ÇÑÆí, ÇÔ¼ö
$f\in \bold H^2$¿¡ ´ëÇÏ¿© $\overline{f}$°¡ À¯°èÇüÀ̸é, Áï,
$\overline{f}=\psi_2/\psi_1$ ($\psi_i\in \bold H^\infty$)À̸é
$\psi_1$ÀÇ ¿ÜÀû ºÎºÐ(outer part)À»  $\psi_2$·Î ³ª´©¾î
$\overline{f}=\psi/\theta$ ($\theta$´Â ³»ÀûÇÔ¼ö, $\psi\in \bold
H^\infty$)ÀÌ´Ù. µû¶ó¼­, $f=\theta\overline{\psi}$·Î ³ªÅ¸³­´Ù. ±×·¯³ª
$f\in \bold H^2$À̹ǷÎ, $\psi\in\mathcal{H}(\theta)$ÀÌ´Ù. µû¶ó¼­
$f\in \bold H^2$ÀÌ°í $\overline{f}$ÀÌ À¯°èÇüÀÌ¸é ´ÙÀ½°ú °°ÀÌ
¾²¿©Áø´Ù.
\begin{equation}\label{3.2.5}
f=\theta\overline{\psi}\quad (\theta\ \text{inner},\ \psi\in \mathcal{H}(\theta)).
\end{equation}
±×·¯¹Ç·Î ¸¸ÀÏ $\varphi=\overline g+f$ÀÌ À¯°èÇüÀÌ°í $T_\varphi$°¡
¾ÆÁ¤±ÔÀ̸é, (\ref{3.2.4})¿Í (\ref{3.2.5})¿¡ ÀÇÇÏ¿© ´ÙÀ½°ú °°ÀÌ ¾µ ¼ö
ÀÖ´Ù:
$$
f=\theta_1\theta_2\overline{a}\quad\text{and}\quad g=\theta_1\overline b,
$$
(¿©±â¿¡¼­, $a\in \mathcal{H}(\theta_1\theta_2)$,
$b\in\mathcal{H}(\theta_1)$ÀÌ´Ù).


´ÙÀ½ º¸Á¶Á¤¸®´Â \cite{CuL1}À¸·ÎºÎÅÍ ³ª¿Â´Ù.

\vs{0.2cm}{\bf Lemma 3.3.1.} {\sl Let $\varphi = \overline{g}+ f \in
\bold L^{\infty}$, where $f$ and $g$ are in $\bold H^2$. Assume that
\begin{equation}\label{3.2.6}
f= \theta_1 \theta_2 \overline{a}\quad\text{and}\quad g = \theta_1
\overline{b}
\end{equation}
for $a \in \mathcal H (\theta_1 \theta_2)$ and $b \in \mathcal H
(\theta_1)$. Let $\psi: = \theta_1 \overline{P_{ \mathcal H
(\theta_1)}(a)} + \overline{g}$. Then $T_{\varphi}$ is hyponormal
if and only if $T_{\psi}$ is.}
%\end{lem}

\vs{0.2cm} Lemma 3.3.1ÀÇ °üÁ¡¿¡¼­, À¯°èÇü ½É¹úÀ» °¡Áø Toeplitz
ÀÛ¿ë¼ÒÀÇ ¾ÆÁ¤±Ô¼ºÀ» ¿¬±¸ÇÒ ¶§, ½É¹ú $\varphi= \overline{g}+ f \in
\bold L^{\infty}$´Â ´ÙÀ½°ú °°ÀÌ ¾µ ¼ö ÀÖ´Ù.
\begin{equation}\label{3.2.7}
f=\theta \overline{a},\quad \quad g=\theta \overline{b},
\end{equation}

\newpage\rightline{\small À̿쿵: Åä¿¡Çø®Ã÷ ÀÛ¿ë¼Ò¿¡ ´ëÇÑ ºê¸®Áö ÀÌ·Ð\ \ \
\large\it 185}

 \vspace{.8 cm} \noindent (¿©±â¿¡¼­, $\theta$´Â
³»ÀûÇÔ¼öÀÌ°í $a,b \in \mathcal H (\theta)$¿Í $\theta$\,´Â ¼­·Î
¼ÒÀÌ´Ù). ÇÑÆí, $f \in \bold H^{\infty}$¸¦ À¯¸®ÇÔ¼ö¶ó°í ÇÏÀÚ. ±×·¯¸é
´ÙÀ½°ú °°ÀÌ ¾µ ¼ö ÀÖ´Ù.
$$
f=p_m(z)+\sum_{i=1}^n \sum_{j=0}^{l_i -1}
\frac{a_{ij}}{(1-\overline{\alpha_i}z)^{l_i -j}} \quad
(0<|\alpha_i|<1),
$$
(¿©±â¿¡¼­, $p_m(z)$´Â Â÷¼ö°¡ $m$ÀÎ ´ÙÇ×½ÄÀÌ´Ù). ÀÌÁ¦ $\theta$°¡
´ÙÀ½°ú °°Àº À¯ÇÑ Blaschke °öÀ̶ó°í ÇÏÀÚ:
$$
\theta=z^{m} \prod_{i=1}^{n} \left(\frac{z- \alpha_i}{1- \overline{\alpha_i}z}\right)^{l_i}.
$$
´ÙÀ½À» °üÂûÇÏÀÚ.
$$
\frac{a_{ij}}{1-\overline{\alpha_i}z}=
\frac{\overline{\alpha_i}a_{ij}}{1 - |\alpha_i|^2}
\Bigl(\frac{z-\alpha_i}{1- \overline{\alpha_i}z}
+\frac{1}{\overline{\alpha_i}} \Bigr).
$$
µû¶ó¼­ $f \in \mathcal H (z\theta)$ÀÌ´Ù. ÀÌÁ¦ $a := \theta
\overline{f}$¶ó Çϸé, $a \in \mathcal H (z \theta)$ÀÌ°í  $f= \theta
\overline{a}$ÀÌ´Ù. µû¶ó¼­ ¸¸ÀÏ $\varphi = \overline{g} + f \in \bold
L^{\infty}$ ($f,g$´Â À¯¸®ÇÔ¼ö)ÀÌ°í $T_{\varphi}$°¡ ¾ÆÁ¤±ÔÀÌ¸é ´ÙÀ½°ú
°°ÀÌ ¾µ ¼ö ÀÖ´Ù.
$$
f=\theta \overline{a},\quad\quad g=\theta \overline{b}
$$
($\theta$\,´Â À¯ÇÑ Blaschke °öÀÌ°í $\theta (0)=0$, $a,b \in \mathcal
H (\theta)$). ÀÌÁ¦ $\theta$\,¸¦ Â÷¼ö $d$\;ÀÇ À¯ÇÑ Blaschke °öÀ̶ó
ÇÏ¸é ´ÙÀ½°ú °°ÀÌ ¾µ ¼ö ÀÖ´Ù.
\begin{equation}\label{3.2.8}
\theta = e^{i \xi} \prod_{i=1}^n B_i ^{n_i},
\end{equation}

\noindent (¿©±â¿¡¼­, $B_i(z) := \frac{z- \alpha_i}{1-
\overline{\alpha_i}z}$, $|\alpha_i| < 1$, $n_i \geq 1$,
$\sum_{i=1}^n n_i =d$). ±×¸®°í $\theta =e^{i\xi} \prod_{j=1}^d B_j
$¶ó ÇÏ°í $\theta$\,ÀÇ °¢ ±ÙÀÌ ±×ÀÇ °è¼ö(multiplicity)¸¸Å­ ¹Ýº¹µÈ´Ù°í
ÇÏÀÚ. ±×·¯¸é ÀÌ Blaschke °öÀº Á¤È®È÷ (\ref{3.2.8})¿¡ ÀÖ´Â Blaschke
°ö°ú °°´Ù. ´ÙÀ½°ú °°ÀÌ µÎÀÚ.
\begin{equation}\label{3.2.9}
\phi_j :=\frac{d_j}{1- \overline{\alpha_j}z}\,B_{j-1}B_{j-2} \cdots
B_1 \quad (1 \leq j \leq d)
\end{equation}

\noindent (¿©±â¿¡¼­, $\phi_1 :=d_1(1- \overline{\alpha_1}z)^{-1}$,
$d_j:=(1-|\alpha_j |^2)^{\frac{1}{2}}$). ±×·¯¸é $\{\phi_j \}_1^d$°¡
$\mathcal H (\theta)$ÀÇ Á¤±ÔÁ÷±³±âÀú¸¦ ÀÌ·ëÀÌ Àß ¾Ë·ÁÁ® ÀÖ´Ù (cf.
\cite[Theorem X.1.5]{FF}). ÀÌÁ¦ $\varphi=\overline g+f\in \bold
L^\infty$ (¿©±â¿¡¼­, $g=\theta \overline b$, $f=\theta \overline a$,
$a, b\in \mathcal H (\theta)$)¶ó ÇÏ°í ´ÙÀ½°ú °°ÀÌ ¾²ÀÚ:
$$
\mathcal{C}(\varphi):= \{k \in \bold H^{\infty} : \varphi -k
\overline{\varphi} \in \bold H^{\infty}\}.
$$
\newpage\leftline{\large{\it 186}\ \ \ \small Çй® ¿¬±¸ÀÇ
µ¿Çâ°ú ÀïÁ¡ -- ¼öÇÐ}

\vspace{.8 cm} \noindent ±×·¯¸é, $k$°¡ $\mathcal{C}(\varphi)$ ¾È¿¡
ÀÖ±â À§ÇÑ ÇÊ¿äÃæºÐÁ¶°ÇÀº $\overline {\theta} b-k \overline{\theta}a
\in \bold H^2$ÀÌ´Ù. Áï,
\begin{equation}\label{3.2.10}
b-ka \in \theta \bold H^2 .
\end{equation}
±×·¯¸é, $\theta^{(n)} (\alpha_i) =0$ (¸ðµç $0 \leq n  < n_i$).
µû¶ó¼­ Á¶°Ç (\ref{3.2.10})Àº ´ÙÀ½°ú µ¿Ä¡ÀÌ´Ù:
¸ðµç $1 \leq i \leq n,$¿¡ ´ëÇÏ¿©,
\begin{equation}\label{3.2.11}
\begin{pmatrix} k_{i,0}\\
k_{i,1}\\
k_{i,2}\\
\vdots\\
k_{i,n_i -2}\\
k_{i,n_i -1}
\end{pmatrix}
=\begin{pmatrix} a_{i,0}&0&0&0&\cdots&0\\
a_{i,1}&a_{i,0}&0&0&\cdots&0\\
a_{i,2}&a_{i,1}&a_{i,0}&0&\cdots&0\\
\vdots&\ddots&\ddots&\ddots&\ddots&\vdots \\
a_{i,n_i -2}&a_{i,n_i -3}&\ddots&\ddots&a_{i,0}&0\\
a_{i,n_i -1}&a_{i,n_i -2}&\hdots&a_{i,2}&a_{i,1}&a_{i,0}
\end{pmatrix}^{-1}
\begin{pmatrix} b_{i,0}\\
b_{i,1}\\
b_{i,2}\\
\vdots\\
b_{i,n_i -2}\\
b_{i,n_i -1}
\end{pmatrix},
\end{equation}
¿©±â¿¡¼­,
$$
k_{i,j}:= \frac{k^{(j)}(\alpha_i)}{j!},\quad a_{i,j}:=
\frac{a^{(j)}(\alpha_i)}{j!} \quad\text{and}\quad
b_{i,j}:=\frac{b^{(j)}(\alpha_i)}{j!}.
$$
¿ªÀ¸·Î, $k \in \bold H^{\infty}$°¡ (\ref{3.2.11})À» ¸¸Á·Çϸé $k$´Â
$\mathcal{C}(\varphi)$¿¡ ¼ÓÇÑ´Ù. µû¶ó¼­, $k$°¡
$\mathcal{C}(\varphi)$¿¡ ¼ÓÇϱâ À§ÇÑ ÇÊ¿äÃæºÐÁ¶°ÇÀº  $k$°¡ ´ÙÀ½À»
¸¸Á·ÇÏ´Â $\bold H^{\infty}$ ¾ÈÀÇ ÇÔ¼öÀÌ´Ù:
\begin{equation}\label{3.2.12}
\frac{k^{(j)}(\alpha_i)}{j!} = k_{i,j} \qquad (1 \leq i \leq n, \
0 \leq j < n_i),
\end{equation}
(¿©±â¿¡¼­, $k_{i,j}$´Â ¹æÁ¤½Ä (\ref{3.2.11})¿¡ ÀÇÇØ ÁÖ¾îÁø´Ù).
´õ±¸³ª $||k||_\infty\le 1$\,À̶ó¸é ÀÌ´Â Á¤È®È÷  classical
Hermite-Fej\' er Interpolation Problem (HFIP)ÀÌ´Ù. ±×·¯¹Ç·Î ´ÙÀ½À»
¾ò´Â´Ù.

\medskip{\bf Theorem 3.3.2.}\ {\sl Let $\varphi=\overline{g}+f\in
\bold L^\infty$, where $f$ and $g$ are rational\linebreak functions.
Then $T_\varphi$ is hyponormal if and only if the corresponding HFIP
$(12)$ is solvable.}


\vspace{0.2cm} ÀÌÁ¦ ¿ì¸®´Â Toeplitz ÀÛ¿ë¼Ò $T_\varphi$ÀÇ ¾ÆÁ¤±Ô¼º¿¡
´ëÇÑ °è»ê°¡´ÉÇÑ ÆÇÁ¤¹ýÀÌ ½É¹ú $\varphi$°¡ »ï°¢´ÙÇ×ÇÔ¼öÀ̰ųª
À¯¸®ÇÔ¼öÀÌ¸é º¸°£¹®Á¦ÀÇ ÇØ¿¡ ÀÇÇÏ¿© ¾ò¾îÁú ¼ö ÀÖ´Ù°í ¸»ÇÒ ¼ö ÀÖ´Ù.

\newpage\rightline{\small À̿쿵: Åä¿¡Çø®Ã÷ ÀÛ¿ë¼Ò¿¡
´ëÇÑ ºê¸®Áö ÀÌ·Ð\ \ \ \large\it 187}

\vspace{0.8cm}

{\bf 3.4. À¯°èÇü ½É¹úÀÇ °æ¿ì}

\vspace{0.3cm}

ÀÌ ºÐÀýÀº Toeplitz ÀÛ¿ë¼ÒÀÇ ¾ÆÁ¤±Ô¼º¿¡ °üÇÏ¿© ÀúÀÚ¿Í ÀúÀÚÀÇ
°øÀúÀÚµéÀÌ ¾òÀº °¡Àå ÃÖ±ÙÀÇ °á°ú·Î¼­, ÀϹÝÀûÀÎ ½É¹úÀÎ À¯°èÇü ½É¹ú¿¡
´ëÇÑ ÆÇÁ¤¹ýÀ» Á¦½ÃÇÒ °ÍÀÌ´Ù. ÀÌ °á°ú°¡ CowenÀÇ Á¤¸®¸¦ Á¦¿ÜÇÏ°í
Áö±Ý±îÁö ¾ò¾îÁø Toeplitz ÀÛ¿ë¼ÒÀÇ ¾ÆÁ¤±Ô¼º¿¡ °üÇÑ °¡Àå ÀϹÝÀûÀÎ
°á°ú¶ó ÇÒ ¼ö ÀÖÀ» °ÍÀÌ´Ù. ¿ì¼± À̸¦ ¾Ë¾Æº¸±â Àü¿¡ À̵¿ÀÛ¿ë¼Ò $U
\equiv T_z$ÀÇ ¾ÐÃà(compression)¿¡ ´ëÇÑ »ï°¢Çà·Ä Ç¥ÇöÀ» ¾Ë¾Æº¸ÀÚ.
³»ÀûÇÔ¼ö $\theta$¿¡ ´ëÇÏ¿©, $U_\theta$¸¦ ´ÙÀ½°ú °°ÀÌ Á¤ÀÇÇÏÀÚ.
\begin{equation}\label{3.1}
U_\theta= P_{\mathcal H (\theta)} U \vert_{\mathcal H (\theta)}.
\end{equation}
ÀÌÁ¦ ¼¼ °¡Áö °æ¿ì·Î ³ª´©¾î »ý°¢ÇÑ´Ù.

\vspace{0.3cm}

\noindent {\bf °æ¿ì 1} : $B$¸¦ Blaschke °öÀ̶ó ÇÏ°í
$\Lambda:=\{\lambda_n : n \geq 1\}$¸¦ $B$ÀÇ ±ÙµéÀ̶ó°í ÇÏÀÚ. ´ÙÀ½°ú
°°ÀÌ ¾²ÀÚ.
$$
\beta_1:=1, \quad
\beta_k:=\prod_{n=1}^{k-1}\frac{\lambda_n-z}{1-\overline{\lambda}_n z}\cdot
\frac{|\lambda_n|}{\lambda_n}\qquad (k\geq 2);
$$
$$
\delta_j\label{deltaj}:=\frac{d_j}{1-\overline{\lambda}_j z}\beta_j \qquad (j \geq
1),
$$
¿©±â¿¡¼­, $d_j\label{dj}:=(1-|\lambda_j|^2)^{\frac{1}{2}}$.
$\mu_B\label{mub}$¸¦ $\mathbb N$ »óÀÇ Ãøµµ·Î¼­
$\mu_B(\{n\}):=\frac{1}{2}d_n^2, (n \in \mathbb N)$¶ó°í ÇÏÀÚ. ÀÌÁ¦
ÇÔ¼ö  $V_B\label{VB}: L^2(\mu_B)\to \mathcal H(B)$ °¡ ´ÙÀ½°ú °°ÀÌ
Á¤Àǵȴٰí ÇÏÀÚ.
\begin{equation}\label{3.2}
V_B(c):=\frac{1}{\sqrt{2}} \sum_{n \geq 1} c(n)d_n \delta_n, \quad c
\equiv \{c(n)\}_{n \geq 1}.
\end{equation}
±×·¯¸é $V_B$´Â À¯´ÏÅ͸®ÀÌ°í $U_B$´Â ´ÙÀ½ÀÇ ÇÔ¼ö·Î º¸³»Áø´Ù.
\begin{equation}\label{3.3}
V_B^*U_B V_B=(I-J_B)M_B,
\end{equation}
¿©±â¿¡¼­, $(M_B\label{MB} c)(n):=\lambda_n c(n)$ ($n\in \mathbb
N$)Àº °ö ÀÛ¿ë¼ÒÀÌ°í
$$
(J_B\label{JB} c)(n):=\sum_{k=1}^{n-1}c(k)|\lambda_k|^{-2} \cdot
\frac{\beta_n(0)}{\beta_k(0)}d_k d_n \ \ \hbox{$(n \in \mathbb N$)}
$$
´Â Çϻﰢ Hilbert-Schmidt ÀÛ¿ë¼ÒÀÌ´Ù. \

\newpage \leftline{\large{\it 188}\ \ \
\small Çй® ¿¬±¸ÀÇ µ¿Çâ°ú ÀïÁ¡ -- ¼öÇÐ}

\vspace{0.8cm}

\noindent {\bf °æ¿ì 2} : $s$¸¦ ¿¬¼ÓÇ¥ÇöÃøµµ  $\mu\equiv
\mu_{s}\label{mus}$¸¦ °¡Áø ƯÀ̳»ÀûÇÔ¼ö(singular inner function)¶ó°í
ÇÏÀÚ. ÀÌÁ¦, $\mu_{\lambda}$À» $\mu$¿¡¼­ È£ $\{\zeta: \zeta \in
\mathbb T, \ 0< \text{arg}\zeta \leq \text{arg}\lambda\}$ À§·ÎÀÇ
»ç¿µÀ̶ó ÇÏ°í ´ÙÀ½°ú °°ÀÌ µÎÀÚ.
$$
s_{\lambda}(\zeta)\label{slambdazeta}:=\text{exp}\Bigl(-\int_{\mathbb
T}\frac{t+\zeta}{t-\zeta}d\mu_{\lambda}(t) \Bigr) \ \ \hbox{($\zeta
\in \mathbb D$)}.
$$
ÀÌÁ¦ ÇÔ¼ö $V_s\label{Vs}: L^2(\mu)\to \mathcal H (s)$°¡ ´ÙÀ½°ú °°ÀÌ Á¤Àǵȴٰí ÇÏÀÚ.
\begin{equation}\label{3.4}
(V_s c)(\zeta)=\sqrt{2}\int_{\mathbb
T}c(\lambda)s_{\lambda}(\zeta)\frac{\lambda
d\mu(\lambda)}{\lambda-\zeta}\quad\hbox{($\zeta \in \mathbb D$)}
\end{equation}
±×·¯¸é $V_s$´Â À¯´ÏÅ͸®ÀÌ°í $U_s$´Â ´ÙÀ½ÀÇ ÇÔ¼ö·Î º¸³»Áø´Ù.
\begin{equation}\label{3.5}
V_s^*U_s V_s=(I-J_s)M_s,
\end{equation}
¿©±â¿¡¼­, $(M_s c)(\lambda):=\lambda c(\lambda)$ ($\lambda\in\mathbb
T$) ´Â °ö ÀÛ¿ë¼ÒÀÌ°í
$$
(J_s\label{Js} c)(\lambda)=2 \int _{\mathbb T} e^{\mu(t)-\mu(\lambda)}
c(t)d_{\mu_{\lambda}}(t)\ \ \hbox{($\lambda \in \mathbb T$)}
$$
´Â Çϻﰢ Hilbert-Schmidt ÀÛ¿ë¼ÒÀÌ´Ù. \

\vspace{.3 cm}

\noindent {\bf °æ¿ì 3} : $\Delta$¸¦ ¼ø¼öÇÑ Á¡Ç¥ÇöÃøµµ $\mu\equiv
\mu_{\Delta}\label{mudelta}$¸¦ °¡Áø ƯÀ̳»ÀûÇÔ¼ö(singular inner
function)¶ó°í ÇÏÀÚ. ±×¸®°í ÁýÇÕ $\{t\in \mathbb T: \mu(\{t\})>0
\}$À» ¼ö¿­ $\{t_k\}_{k \in \mathbb N}$·Î ¹è¿­ÇÏ°í ´ÙÀ½°ú  °°ÀÌ ¾²ÀÚ:
$\mu_k:=\mu(\{t_k\}), \ k \geq 1$. \  ´õ±¸³ª, $\mu_{\Delta}$¸¦
$\mathbb R_{+}=[0, \infty)$ »óÀÇ Ãøµµ·Î¼­
$d\mu_{\Delta}(\lambda)=\mu_{[\lambda]+1}d \lambda$¶ó°í ÇÏÀÚ.ÀÌÁ¦
´ÜÀ§¿øÆÇ $\mathbb D$ À§ÀÇ ÇÔ¼ö
$\Delta_{\lambda}\label{deltalambda}$¸¦ ´ÙÀ½°ú °°ÀÌ Á¤ÀÇÇÏÀÚ.
$$
\Delta_{\lambda}(\zeta):=\text{exp}\Biggl\{-\sum_{k=1}^{[\lambda]}\mu_k
\frac{t_k+\zeta}{t_k-\zeta}-(\lambda-[\lambda])\mu_{[\lambda]+1}
\frac{t_{[\lambda]+1}+\zeta}{t_{[\lambda]+1}-\zeta}\Biggr \},
$$
¿©±â¿¡¼­, $[\lambda]$´Â $\lambda$ ($\lambda \in \mathbb R_{+}$)ÀÇ
Á¤¼ö ºÎºÐÀÌ°í $\Delta_0:=1$·Î ¾à¼ÓÇÑ´Ù. \ ±×¸®°í ÇÔ¼ö
$V_{\Delta}\label{Vdelta}: L^2(\mu_{\Delta})\to \mathcal H
(\Delta)$°¡ ´ÙÀ½°ú °°ÀÌ Á¤Àǵȴٰí ÇÏÀÚ.
\begin{equation}\label{3.6}
(V_{\Delta}c)(\zeta):=\sqrt{2} \int_{\mathbb
R_{+}}c(\lambda)\Delta_{\lambda}(\zeta)(1-\overline{t}_{[\lambda]+1}
\zeta)^{-1}d\mu_{\Delta}(\lambda) \ \hbox{($\zeta \in \mathbb D$)}
\end{equation}
±×·¯¸é $V_{\Delta}$´Â À¯´ÏÅ͸®ÀÌ°í $U_\Delta$´Â ´ÙÀ½ÀÇ ÇÔ¼ö·Î º¸³»Áø´Ù.
\begin{equation}\label{3.7}
V_{\Delta}^*U_\Delta V_{\Delta}=(I-J_{\Delta})M_{\Delta},
\end{equation}

\newpage\rightline{\small À̿쿵:
Åä¿¡Çø®Ã÷ ÀÛ¿ë¼Ò¿¡ ´ëÇÑ ºê¸®Áö ÀÌ·Ð\ \ \ \large\it 189}

\vspace{.8 cm}

\noindent ¿©±â¿¡¼­, $(M_{\Delta}c)(\lambda):= t_{[\lambda]+1}
c(\lambda)$, ($\lambda\in \mathbb R_+$)Àº °ö ÀÛ¿ë¼ÒÀÌ°í
$$
(J_{\Delta}\label{Jdelta} c)(\lambda) := 2 \int _{0}^{\lambda}c(t)
\frac{\Delta_{\lambda}(0)}{\Delta_{t}(0)} d\mu_{\Delta}(t) \ \
\hbox{($\lambda \in \mathbb R_+$)}
$$
´Â Çϻﰢ Hilbert-Schmidt ÀÛ¿ë¼ÒÀÌ´Ù. \

\medskip

ÀÌÁ¦ À§ÀÇ ¼¼ °æ¿ì¸¦ Á¾ÇÕÇÏ¸é ´ÙÀ½À» ¾ò´Â´Ù.

\medskip

{\bf Triangularization theorem.} \cite [p.123] {Ni} {\sl Let
$\theta$ be an inner function with the canonical factorization
$\theta=B\cdot s \cdot \Delta$, where $B$ is a Blaschke product,
and $s$ and $\Delta$ are singular functions with representing
measures $\mu_{s}$ and $\mu_{\Delta}$ respectively, with $\mu_{s}$
continuous and $\mu_{\Delta}$ a pure point measure. \  Then the
map $V:\,L^2(\mu_B)\times L^2(\mu_s) \times L^2(\mu_{\Delta})\to
\mathcal H(\theta)$ defined by
\begin{equation}\label{3.8}
V:=\begin{bmatrix}V_B&0&0\\0&BV_s&0\\0&0&BsV_{\Delta}\end{bmatrix}
\end{equation}
is unitary, where $V_B, \mu_B, V_S, \mu_S, V_\Delta, \mu_\Delta$ are
defined in (\ref{3.2}) - (\ref{3.7}) and $U_\theta$ is mapped onto
the operator
$$
M:= V^*U_\theta V =
\begin{bmatrix}M_B&0&0\\0&M_{s}&0\\0&0&M_{\Delta}\end{bmatrix}+J,
$$
where $M_B, M_S, M_{\Delta}$ are defined in (\ref{3.3}), (\ref{3.5})
and (\ref{3.7}) and
$$
J:=-\begin{bmatrix}J_BM_B&0&0\\0&J_sM_s&0\\0&0&J_{\Delta}M_\Delta\end{bmatrix}+A
$$
is a lower-triangular Hilbert-Schmidt operator, with $A^3=0$,
$\text{rank}\,A\leq 3$.}

\

À§ÀÇ Á¤¸®¸¦ ÀÌ¿ëÇϸé À¯°èÇü ½É¹úÀ» °¡Áø Toeplitz ÀÛ¿ë¼ÒÀÇ ¾ÆÁ¤±Ô¼ºÀ»
ÆǺ°ÇÒ ¼ö ÀÖ´Â ÆÇÁ¤¹ýÀ» ¾òÀ» ¼ö ÀÖ´Ù.

\newpage\leftline{\large{\it 190}\ \ \ \small Çй® ¿¬±¸ÀÇ µ¿Çâ°ú ÀïÁ¡ --
¼öÇÐ} \vspace{0.8cm}

{\bf Theorem 3.4.1.} {\rm \cite{CHL1}} {\sl Let $\varphi\equiv
\varphi_-^*+\varphi_+\in \bold L^\infty$ be such that $\varphi$
and $\overline\varphi$ are of bounded type of the form
$$
\varphi_+=\theta_1\theta_0\overline a\quad\hbox{and}\quad
\varphi_-=\theta_1\overline b,
$$
where $\theta_1$ and $\theta_0$ are inner functions and $a,b\in
H^2$. \  If $k\in\mathcal{C}(\varphi)$ then
$$
T_\varphi\ \hbox{is hyponormal}\ \Longleftrightarrow\ k(M)\ \hbox{is
contractive},
$$
where $M$ is defined as follows:
\begin{equation}\label{3.10}
\begin{cases}
V: L\equiv L^2(\mu_B) \times L^2(\mu_s)
         \times L^2(\mu_{\Delta})\to \mathcal H(\theta_1 \theta_0)\ \
            \hbox{\rm is unitary as in (\ref{3.8})};\\
M:= V^*U_{\theta_1\theta_0}V;\\
\mathcal L :=L \otimes \mathbb C^n\\
\mathcal V := V \otimes I_n.
\end{cases}
\end{equation}}
%\end{thm}
À§¿¡¼­ $k(M)$Àº $H^\infty$-ÇÔ¼ö°è»êÀ» µû¸¥´Ù.


%%%%%%%%%%%%%%%%%%%%%%%%%%%%%%%%%%%%%%%%%%%%%%%%%%%%%%%%%%%%%%%%%%%%%%%%%%%%%%%%%%%%%%%%%%%%%%%%%%%%%%%%%%%%%%%%%%%

\vspace{.8 cm} {\large\bf 4. Toeplitz ÀÛ¿ë¼ÒÀÇ ºÎºÐÁ¤±Ô¼º}

\vspace{.5 cm}


ÀÌ ÀýÀº ´ÙÀ½ÀÇ ¹®Á¦¿¡ °üÇÑ °ÍÀÌ´Ù: {\it ¾î¶² Toeplitz ÀÛ¿ë¼Ò°¡
ºÎºÐÁ¤±ÔÀΰ¡?} Toeplitz ÀÛ¿ë¼Ò $T_\varphi$°¡ Çؼ®Àû(analytic)À̶ó
ÇÔÀº ½É¹ú $\varphi$°¡ $\bold H^\infty$¿¡ ¼ÓÇÒ ¶§¸¦ ¸»ÇÑ´Ù, Áï,
$\varphi$°¡ $\mathbb{D}$ »óÀÇ À¯°è Çؼ®ÇÔ¼öÀÏ ¶§ÀÌ´Ù. ÀÌ´Â
ºÎºÐÁ¤±ÔÀÓÀ» ½±°Ô ¾Ë ¼ö ÀÖ´Ù. ½ÇÁ¦·Î,
$$
T_\varphi h=P(\varphi h)=\varphi h=M_\varphi h\quad \hbox{($h\in \bold H^2$)},
$$
¿©±â¿¡¼­, $M_\varphi$´Â   $\bold L^2$ »ó¿¡¼­ $\varphi$¸¦ °öÇÏ´Â °ö
ÀÛ¿ë¼Ò(multiplication operator)·Î¼­ Á¤±ÔÀÛ¿ë¼ÒÀÌ´Ù. 1970³â¿¡
P.\,R.\ Halmos´Â {\it Ten Problems in Hilbert Space} (cf.
\cite{Ha1}, \cite{Ha2})¶ó´Â °­ÀÇ·Ï¿¡¼­ ÀÛ¿ë¼Ò Çа迡  Hilbert °ø°£
»óÀÇ ¹®Á¦ ¿­ °³¸¦ Á¦½ÃÇÏ¿´´Âµ¥, ±× Áß ´Ù¼¸¹ø° ¹®Á¦ (¼ÒÀ§ Halmos'
Problem 5·Î ºÒ¸®¿ò)°¡ Toeplitz ÀÛ¿ë¼ÒÀÇ ºÎºÐÁ¤±Ô¼º¿¡ ´ëÇÑ
°ÍÀ̾ú´Ù:

\vspace{.2 cm}\label{4.1.1}\ \ \ \ \ {\it ¸ðµç ºÎºÐÁ¤±Ô Toeplitz
ÀÛ¿ë¼Ò´Â Á¤±ÔÀ̰ųª Çؼ®ÀûÀϱî\,?}

\vspace{.2 cm}  ÀÌ ¹®Á¦´Â Á¤±Ô Toeplitz ÀÛ¿ë¼Ò¿Í Çؼ®Àû Toeplitz
ÀÛ¿ë¼Ò°¡ Àß ÀÌÇصǰí, ¶Ç ¸ðµÎ ºÎºÐÁ¤±ÔÀ̹ǷΠ¸Å¿ì ÀÚ¿¬½º·¯¿î ¹®Á¦¶ó
ÇÏ°Ú´Ù.

\vspace{.2 cm}

%%%%%%%%%%%%%%%%%%%%%%%%%%%%%%%%%%%%%%%%%%%%%%%%%%%%%%%%%%%%%%%%%%%%%%%%%%%%%%%%%%%%%%%%%%%
\newpage\rightline{\small À̿쿵: Åä¿¡Çø®Ã÷ ÀÛ¿ë¼Ò¿¡ ´ëÇÑ ºê¸®Áö ÀÌ·Ð\
\ \ \large\it 191}


\vspace{.8 cm}

{\bf 4.1. Halmos' Problem 5}

\vspace{0.3cm} ÀÌÁ¦ Halmos' Problem 5¿Í °ü·ÃµÈ ¿¬±¸¿¡ ´ëÇÏ¿©
»ìÆ캻´Ù. 1976³â¿¡, M. Abrahamse \cite{Ab}°¡ Halmos' Problem 5¿¡
´ëÇÑ ¸Å¿ì ÀϹÝÀûÀÎ ÃæºÐÁ¶°ÇÀ» ÁÖ¾ú´Âµ¥, ÀÌ ¿À·¡ µÈ Á¤¸®°¡ ¾à°£
³î¶ø°Ôµµ ¿À´Ã³¯±îÁö ¾Ë·ÁÁø °¡Àå ÁÁÀº ÃæºÐÁ¶°ÇÀÌ´Ù.

\vspace{0.3cm} {\bf Theorem 4.1.1.} Abrahamse's Theorem) {\rm
\cite{Ab}} {\sl If
\begin{itemize}
\item[{\rm (i)}] $T_{\varphi}$ is hyponormal;
\item[{\rm (ii)}] $\varphi$ or $\overline\varphi$ is of bounded type;
\item[{\rm (iii)}] ${\rm ker}[T_{\varphi}^*,T_{\varphi}]$ is invariant for $T_{\varphi}$,
\end{itemize}
then $T_{\varphi}$ is normal or analytic.}
%\end{thm}

\vspace{0.3cm} ÇÑÆí, ¸¸ÀÏ $S$°¡ $\mathcal H$ »óÀÇ ºÎºÐÁ¤±ÔÀÌ°í
$N:=\mbox{mne}\,(S)$À̸é,
$$
{\rm ker}[S^*,S]=\{ f:\ <f, [S^*,S]f>=0\}=\{ f:\ ||S^*f||=||Sf||\}=\{ f:\ N^{*}f\in\mathcal{H}\}.
$$
±×·¯¹Ç·Î $S({\rm ker}[S^*,S])\subseteq {\rm ker}[S^*,S]$. µû¶ó¼­,
Theorem 4.1.1¿¡ ÀÇÇì¼­ ´ÙÀ½À» ¾ò´Â´Ù.

\vspace{0.3cm} {\bf Corollary 4.1.2.}\ {\sl If $T_{\varphi}$ is
subnormal and if $\varphi$ or $\overline\varphi$ is of bounded type,
then $T_{\varphi}$ is normal or analytic.} \vspace{0.3cm}

À§ÀÇ Abrahamse's TheoremÀº ÃÖ±Ù¿¡ ÀúÀÚ¿Í °øÀúÀÚµéÀÌ Çà·ÄÇÔ¼ö½É¹ú·Î
È®ÀåÇÏ´Â µ¥ ¼º°øÇÏ¿´´Ù (\cite{CHL2}). ´ÙÀ½Àº M. Abrahamse
\cite{Ab}ÀÌ  ¹àÇô³½ À¯°èÇü ÇÔ¼öÀÇ Æ¯¼ºÈ­ÀÌ´Ù.

\vspace{0.3cm} {\bf Corollary 4.1.3} {\sl  A function $\varphi$ is
of bounded type if and only if ${\rm ker}H_{\varphi}\neq\{0\}$. }

\vspace{0.3cm} Toeplitz ÀÛ¿ë¼ÒÀÇ ºÎºÐÁ¤±Ô¼ºÀ» Àß ¿¬±¸ÇÒ ¼ö ÀÖ´Â ¹æ¹ý
ÁßÀÇ Çϳª°¡ ±âÁ¸ÀÇ Àß ¾Ë·ÁÁø ºÎºÐÁ¤±Ô ÀÛ¿ë¼Ò¿Í À¯´ÏÅ͸® µ¿Ä¡ÀÎ
°æ¿ì¸¦ ã¾Æ³»´Â °ÍÀε¥, °¡ÁßÀ̵¿ ÀÛ¿ë¼Ò°¡ ºñ±³Àû ºÎºÐÁ¤±Ô¸¦ ¾Ë±â
½¬¿î °æ¿ìÀ̹ǷΠ¾î¶² °¡ÁßÀ̵¿ ÀÛ¿ë¼Ò°¡ Toeplitz ÀÛ¿ë¼Ò¿Í À¯´ÏÅ͸®
µ¿Ä¡ÀÎÁö ¹°¾îº¼ ¼ö ÀÖÀ» °ÍÀÌ´Ù. ´ÙÀ½Àº ±× ù¹ø° ´äÀÌ´Ù.


\newpage\leftline{\large{\it 192}\ \ \
\small Çй® ¿¬±¸ÀÇ µ¿Çâ°ú ÀïÁ¡ -- ¼öÇÐ} \vspace{0.8cm}

{\bf Proposition 4.1.4.} {\sl If $A$ is a weighted shift with
weights $a_0, a_1, a_2, \cdots $ such that
$$
0\le a_0 \le a_1 \le \cdots < a_N = a_{N+1}= \cdots = 1,
$$
then $A$ is not unitarily equivalent to any Toeplitz operator. }

\vspace{.2 cm} ÀÌÁ¦ À¯¸íÇÑ Bergman À̵¿ÀÛ¿ë¼Ò(shift) (°¡Áß¼ö¿­ÀÌ
$\sqrt{{n+1}\over{n+2}}$·Î ÁÖ¾îÁö´Â À̵¿ ÀÛ¿ë¼Ò)´Â ºÎºÐÁ¤±ÔÀ̹ǷÎ
¿ì¸®´Â ¾à°£ÀÇ ±â´ë°¨À» °¡Áö°í ´ÙÀ½À» ¹°À» ¼ö ÀÖÀ» °ÍÀÌ´Ù.
\begin{equation}\label{4.1.2}
\mbox{Bergman À̵¿ ÀÛ¿ë¼Ò´Â ¾î¶² Toeplitz ÀÛ¿ë¼Ò¿Í À¯´ÏÅ͸®
µ¿Ä¡Àΰ¡?}
\end{equation}

¸¸ÀÏ À§ÀÇ Áú¹® (\ref{4.1.2})ÀÇ ´ë´äÀÌ ±àÁ¤ÀûÀ̶ó¸é, Halmos' Problem
5ÀÇ ´ë´äÀº ºÎÁ¤ÀûÀÌ µÉ °ÍÀÌ´Ù. À̸¦ ¾Ë±â À§Çؼ­ Bergman À̵¿ÀÛ¿ë¼Ò
$S$°¡ Toeplitz ÀÛ¿ë¼Ò $T_\varphi$¿Í À¯´ÏÅ͸® µ¿Ä¡¶ó°í °¡Á¤ÇÏÀÚ.
±×·¯¸é,
$$
\frak{R}(\varphi)\subseteq\ \sigma
_e(T_\varphi )=\sigma_e(S)=\mbox{the unit circle}\ \mathbb{T}.
$$
µû¶ó¼­, $\varphi$´Â ±× Àý´ë°ªÀÌ 1ÀÎ ÇÔ¼ö(unimodular)ÀÌ´Ù. ±×·±µ¥
$S$°¡ µîÀåÇÔ¼ö°¡ ¾Æ´Ï¹Ç·Î $\varphi$´Â ³»ÀûÇÔ¼ö°¡ ¾Æ´Ï´Ù. ±×·¯¹Ç·Î,
$T_\varphi$´Â Çؼ®Àû Toeplitz ÀÛ¿ë¼Ò°¡ ¾Æ´Ï´Ù. 1983³â¿¡ S. Sun
\cite{Sun}ÀÌ ¹®Á¦ (\ref{4.1.2})¿¡ ºÎÁ¤ÀûÀÎ ÇØ´äÀ» ÁÖ¾ú´Ù. \vspace{.2
cm}

{\bf Theorem 4.1.5.} (Sun's Theorem) {\rm \cite{Sun}} {\sl Let $T$
be a weighted shift with a strictly increasing weight sequence
$\{a_{n}\}_{n=0}^{\infty}$. If $T\cong T_{\varphi}$ then
$$
a_{n}=\sqrt{1-{\alpha}^{2n+2}} \,||T_{\varphi}||\quad (0<\alpha <1).
$$}
\indent À§ÀÇ Theorem 4.1.5¿¡ ÀÇÇÏ¿© ´ÙÀ½À» ¾ò´Âµ¥ ÀÌ´Â ¹®Á¦
(\ref{4.1.2})ÀÇ ´äÀ» ÁØ´Ù. \vspace{.2 cm}

{\bf Corollary 4.1.6} {\sl The Bergman shift is not unitarily
equivalent to any Toeplitz operator.} \vspace{.2 cm}
%\end{cor}

\vspace{.2 cm}

¸¶Ä§³», 1984³â¿¡ Cowen°ú Long \cite{CoL}ÀÌ Halmos' Problem 5¸¦
ÇØ°áÇÏ¿´´Âµ¥, ¿ì¼± À̸¦ À§ÇØ ´ÙÀ½ÀÌ ÇÊ¿äÇÏ´Ù.

\vspace{.2 cm} {\bf Lemma 4.1.7.} {\sl The weighted shift $T\equiv
W_{\alpha}$ with weights $\alpha_n\equiv
(1-\alpha^{2n+2})^{\frac{1}{2}}\ (0<\alpha <1)$ is subnormal.}
%\end{lem}

\vspace{.2 cm} ´õ±¸³ª Toeplitz ÀÛ¿ë¼Ò°¡ °¡ÁßÀ̵¿ ÀÛ¿ë¼Ò¿Í À¯´ÏÅ͸®
µ¿Ä¡ÀÌ¸é ºÎºÐÁ¤±Ô°¡ µÊÀÌ ¾Ë·ÁÁ³´Ù.



\newpage\rightline{\small À̿쿵:
Åä¿¡Çø®Ã÷ ÀÛ¿ë¼Ò¿¡ ´ëÇÑ ºê¸®Áö ÀÌ·Ð\ \ \ \large\it 193}

\vspace{.8 cm}

{\bf Corollary 4.1.8.} {\sl If $T_{\varphi} \cong$ a weighted shift,
then $T_\varphi$ is subnormal.} \vspace{.2 cm}

¸¸ÀÏ  $T_\varphi$°¡ °¡ÁßÀ̵¿ ÀÛ¿ë¼Ò¿Í µ¿Ä¡À̸é,
$\varphi$´Â ¾î¶² ÇüÅÂÀϱî?
\cite[Theorem 3.7]{Sun}ÀÇ Áõ¸íÀ» ºÐ¼®Çغ¸¸é,
´ÙÀ½À» ¾Ë ¼ö ÀÖ´Ù.
$$
\psi=\varphi -\alpha\overline{\varphi}\in {\bold H}^{\infty}.
$$
±×·¯³ª,
\begin{align*}
T_\psi =T_\varphi -\alpha T_{\varphi}^* &=
{\begin{pmatrix}
0 &-\alpha a_0\\
a_0&0&-\alpha a_1\\
&a_1&0&-\alpha a_2\\
&&a_2&0&\ddots\\
&&&\ddots&\ddots
 \end{pmatrix}}\\
&=\begin{pmatrix}
0 &-\alpha\\
1&0&-\alpha\\
&1&0&-\alpha\\
&&1&0&\ddots\\
 &&&\ddots&\ddots
\end{pmatrix} + K\quad \mbox{($K$ compact)}\\
&\cong T_{z-\alpha\overline{z}}+K.
\end{align*}
µû¶ó¼­,
$\mbox{ran}\,(\psi)=\sigma_{e}(T_{\psi})
=\sigma_{e}(T_{z-\alpha\overline{z}})={\rm ran}(z-\alpha\overline{z})$.
±×·¯¹Ç·Î, $\psi$´Â ´ÜÀ§¿øÆÇÀ» ²ÀÁöÁ¡ÀÌ  $\pm i(1+\alpha)$ÀÌ°í,  $\pm(1-\alpha)$À»
Áö³ª´Â Ÿ¿øÀ¸·Î º¸³»´Â µî°¢»ç»óÀÌ´Ù.
ÇÑÆí, $\psi=\varphi -\alpha\overline{\varphi}$.
±×·¡¼­,
$\alpha\overline{\psi}=\alpha\overline{\varphi} -\alpha^2\varphi$
ÀÌ°í, ´ÙÀ½ÀÌ ¼º¸³ÇÑ´Ù.
$$
\varphi=\frac{1}{1-\alpha^2}(\psi +\alpha\overline{\psi}).
$$

ÀÌÁ¦ ´ÙÀ½À» ¾ò°Ô µÈ´Ù.

\vspace{.2 cm}{\bf Theorem 4.1.9.} (Cowen and Long Theorem) {\rm
\cite{CoL}} {\sl For $0<\alpha <1$, let $\psi$ be a conformal map of
$\mathbb{D}$ onto the interior of the ellipse with vertices $\pm
i(1-\alpha)^{-1}$ and passing through $\pm (1+\alpha)^{-1}$. Then
$T_{\psi +\alpha\overline{\psi}}$ is a subnormal weighted shift that
is neither analytic nor normal.}
%\end{thm}
\vspace{.2 cm}

Abrahamse's Theorem °ú Theorem 4.1.9\,·ÎºÎÅÍ ´ÙÀ½À» ¾ò´Â´Ù.

\vspace{.2 cm}

\newpage\leftline{\large{\it 194}\ \ \ \small Çй® ¿¬±¸ÀÇ µ¿Çâ°ú ÀïÁ¡ --
¼öÇÐ}\vspace{0.8cm}

{\bf Corollary 4.1.10.} {\sl If $\varphi=\psi
+\alpha\overline{\psi}$ is as in Theorem 4.1.9, then neither
$\varphi$ nor $\overline{\varphi}$ is bounded type.} \vspace{.2 cm}

ÀÌÁ¦ Halmos' Problem 5°¡ ÇØ°áµÇ¾ú´Ù°í´Â Çϳª, º»ÁúÀûÀ¸·Î Toeplitz
ÀÛ¿ë¼ÒÀÇ ºÎºÐÁ¤±Ô¼ºÀº ¿©ÀüÈ÷ ¾ÏÈæ ¼Ó¿¡ °¤Çô ÀÖ´Ù°í Çصµ °ú¾ðÀÌ
¾Æ´Ï´Ù. Abrahamse's Theorem Á¤µµ°¡ °¡Àå ÁÁÀº ´ë´äÀ̹ǷΠ¾ÕÀ¸·Î °è¼Ó
Toeplitz ÀÛ¿ë¼ÒÀÇ ºÎºÐÁ¤±Ô¼ºÀ» ŽÇèÇØ °¥ ÇÊ¿ä°¡ ÀÖ´Ù. Áö³­ ½Ê ¼ö³â°£
ÀúÀÚ¿Í °øµ¿ ¿¬±¸ÀÚµéÀº ÀÌ ¹®Á¦¸¦ ²÷ÀÓ¾øÀÌ °íÂûÇØ ¿ÔÀ¸¸ç ºÎºÐÀûÀ¸·Î
Èï¹Ì·Î¿î °á°úµéÀ» ¸¸µé¾î³»±â´Â ÇÏ¿´Áö¸¸, ¿©ÀüÈ÷ ±× ¿©Á¤Àº
¹Ì¹ÌÇÏ¿´À¸¸ç ¾ÕÀ¸·Îµµ ¾ó¸¶³ª ´õ ±ä ¿©Á¤À» °ÅÃÄ¾ß ÇÒ Áö ¸ð¸¥´Ù. ¹°·Ð
Áö³ª¿Â ¿©Á¤ Áß¿¡ Á¾Á¾ ½Å±â·ç °°Àº °ÍÀ» º¸±ä ÇÏ¿´Áö¸¸ °¡±îÀÌ °¡º¸¸é
±×³É ½Å±â·ç¿´À» »Ó, ¿©ÀüÈ÷ ±× ´Ü¼­¸¦ ãÁö ¸øÇÏ°í ÀÖ´Ù. ´ÙÀ½ Àý¿¡¼­´Â
±× Áß ÇÑ °¡Áö¸¦ »ìÆ캼 °ÍÀÌ´Ù.

%%%%%%%%%%%%%%%%%%%%%%%%%%%%%%%%%%%%%%%%%%%%%%%%%%%%%%%%%%%%%%%%%%%%%%%%%%%%%%%%%%
\vspace{.3 cm} {\bf 4.2. Toeplitz ÀÛ¿ë¼ÒÀÇ $k$-¾ÆÁ¤±Ô¼º°ú ºÎºÐÁ¤±Ô¼º
»çÀÌÀÇ Æ´}

\vspace{.3 cm} Toeplitz ÀÛ¿ë¼ÒÀÇ $k$-¾ÆÁ¤±Ô¼º°ú ºÎºÐÁ¤±Ô¼º »çÀÌÀÇ Æ´
À» ÀÌÇØÇÏ´Â °ÍÀº ¸Å¿ì Èï¹Ì·Î¿ö º¸ÀδÙ. ÀÌ ¹®Á¦¿¡ ´ëÇÑ Ã¹¹ø°
Èĺ¸·Î¼­ ¿ì¸®´Â ´ÙÀ½À» »ý°¢ÇÒ ¼ö ÀÖ´Ù (\cite{CuL1}): ¸ðµç 2-¾ÆÁ¤±Ô
Toeplitz ÀÛ¿ë¼Ò´Â ºÎºÐÁ¤±ÔÀΰ¡?

\medskip

ÀÌ ¹®Á¦¿¡ ´ëÇÏ¿© \cite{CuL1}¿¡¼­ ´ÙÀ½ÀÌ º¸¿©Á³´Ù:

\vspace{.2 cm}{\bf Theorem 4.2.1.} {\rm \cite{CuL1}}  {\sl Every
trigonometric Toeplitz operator whose square is hyponormal must be
normal or analytic. Hence, in particular, every 2-hyponormal
trigonometric Toeplitz operator is subnormal.}
%\end{thm}

\vspace{.2 cm} \cite{Cu1}¿¡¼­ °¡ÁßÀ̵¿ ÀÛ¿ë¼Ò¿¡ ´ëÇÑ ¾ÆÁ¤±Ô¼º°ú
2-¾ÆÁ¤±Ô¼º »çÀÌ¿¡ Æ´ÀÌ Á¸ÀçÇÔÀ» º¸¿´´Ù. À§ÀÇ Theorem 4.2.1\,Àº ¿ª½Ã
Toeplitz ÀÛ¿ë¼Ò¿¡ ´ëÇؼ­µµ  ¾ÆÁ¤±Ô¼º°ú 2-¾ÆÁ¤±Ô¼º »çÀÌ¿¡ Æ´ÀÌ
Á¸ÀçÇÔÀ» º¸¿©ÁÖ°í ÀÖ´Ù. ¿¹¸¦ µé¾î, ¸¸ÀÏ
$$
\varphi(z)=\sum_{n=-m}^N a_n z^n\quad (m<N)
$$
¿¡ ´ëÇÏ¿© $T_\varphi$°¡ ¾ÆÁ¤±ÔÀ̸é, Theorem 4.2.1¿¡ ÀÇÇØ
$T_\varphi$´Â °áÄÚ 2-¾ÆÁ¤±Ô°¡ ¾Æ´Ï´Ù (±× ÀÌÀ¯: $T_\varphi$°¡ Á¤±Ôµµ
Çؼ®Àûµµ ¾Æ´Ô). ºñ±³ÇÏ¿© ´ÙÀ½À» »ó±âÇÒ ÇÊ¿ä°¡ ÀÖ´Ù: ¸¸ÀÏ,
$\varphi(z)=\sum_{n=-m}^N a_n z^n$¿¡ ´ëÇÏ¿© $T_\varphi$°¡ Á¤±ÔÀ̸é,
$m=N$ (cf. \cite{FL1}).

\vspace{.2 cm} ¿ì¸®´Â Abrahamse's TheoremÀ» ¾à°£ È®ÀåÇÒ ¼ö ÀÖ´Ù.
À̸¦ À§ÇÏ¿© ´ÙÀ½ °üÂûÀÌ ÇÊ¿äÇÏ´Ù.


\newpage\rightline{\small À̿쿵: Åä¿¡Çø®Ã÷ ÀÛ¿ë¼Ò¿¡ ´ëÇÑ ºê¸®Áö ÀÌ·Ð\
\ \ \large\it 195}

\vspace{.8 cm}

{\bf Proposition 4.2.2.} {\rm \cite{CuL2}} {\sl If
$T\in\mathcal{B(H)}$ is 2-hyponormal then
\begin{equation}\label{4.2.1}
T\bigl(\text{ker}\,[T^*,T]\bigr)\ \subseteq \text{ker}\,[T^*,T].
\end{equation}}

±×·¯¸é  Áï½Ã ´ÙÀ½À» ¾ò´Â´Ù.

\vspace{.2 cm}{\bf Corollary 4.2.3.} {\sl If $T_\varphi$ is
$2$-hyponormal and if $\varphi$ or $\bar\varphi$ is of bounded type
then $T_\varphi$ is normal or analytic, so that $T_\varphi$ is
subnormal.}  \vspace{.2 cm}

¿ì¸®´Â ¿ª½Ã ´ÙÀ½À» Áõ¸íÇÒ ¼ö ÀÖ´Ù.

\vspace{.2 cm}{\bf Corollary 4.2.4.} {\sl If $T_\varphi$ is a
$2$-hyponormal operator such that $\mathcal{E}(\varphi)$ contains at
least two elements then $T_\varphi$ is normal or analytic, so that
$T_\varphi$ is subnormal.}
%\end{cor}
\vspace{.2 cm}

Corollary 4.2.3\,°ú Corollary 4.2.4\,·ÎºÎÅÍ, ¸¸ÀÏ $T_\varphi$°¡
$2$-¾ÆÁ¤±ÔÀÌÁö¸¸ ºÎºÐÁ¤±Ô°¡ ¾Æ´Ï¸é, $\varphi$´Â À¯°èÇü ÇÔ¼ö°¡ ¾Æ´Ï°í
$\mathcal{E}(\varphi)$´Â ²À ÇϳªÀÇ ¿ø¼Ò·Î ÀÌ·ç¾îÁüÀ» ¾Ë ¼ö ÀÖ´Ù.
¿ì¸®´Â ³ª¾Æ°¡ Toeplitz ÀÛ¿ë¼Ò¿¡ ´ëÇÏ¿© ÀÓÀÇÀÇ $k$-¾ÆÁ¤±Ô¼º°ú
ºÎºÐÁ¤±Ô¼º »çÀÌ¿¡ Æ´ÀÌ Á¸ÀçÇÔÀ» º¸¿´´Ù.

\vspace{.2 cm}{\bf Theorem 4.2.5.} {\rm \cite{CLL}} {\sl Let
$0<\alpha<1$ and let $\psi$ be the conformal map of the unit disk
onto the interior of the ellipse with vertices $\pm(1+\alpha)i$ and
passing through $\pm(1-\alpha)$. Let $\varphi=\psi+\lambda\bar\psi$
and let $T_\varphi$ be the corresponding Toeplitz operator on $H^2$.
Then $T_\varphi$ is $k$-hyponormal if and only if $\lambda$ is in
the circle $\left|z-\frac{\alpha(1-\alpha^{2j})}
{1-\alpha^{2j+2}}\right|
=\frac{\alpha^j(1-\alpha^2)}{1-\alpha^{2j+2}}$ for $j=0,1,\cdots
,k-2 $ or in the closed disk
$\left|z-\frac{\alpha(1-\alpha^{2(k-1)})} {1-\alpha^{2k}}\right| \le
\frac{\alpha^{k-1}(1-\alpha^2)}{1-\alpha^{2k}}$. }

%%%%%%%%%%%%%%%%%%%%%%%%%%%%%%%%%%%%%%%%%%%%%%%%%%%%%%%%%%%%%
\vspace{.3 cm} {\bf 4.3 ¹ÌÇØ°á ¹®Á¦}

\vspace{.3 cm} Cowen°ú LongÀÇ ¾ÆÀ̵ð¾î \cite{CoL}´Â Toeplitz
ÀÛ¿ë¼Ò¿Í ºÎºÐÁ¤±Ô¼º »çÀÌÀÇ °ü°è¿¡ ´ëÇÑ ¾î¶² ÀϹÝÀûÀΠƯ¡µµ ÁÖÁö
¸øÇß´Ù. °á±¹ Áö±Ý±îÁö ´©±¸µµ Toeplitz ÀÛ¿ë¼ÒÀÇ ºÎºÐÁ¤±Ô¼º¿¡ ´ëÇÏ¿©
½É¹ú¿¡ ÀÇÇÑ Æ¯¼ºÈ­¸¦ ¸¸µéÁö ¸øÇß´Ù. µû¶ó¼­ ´ÙÀ½ ¹®Á¦´Â ¿©ÀüÈ÷
¸Å·ÂÀûÀÌ°í Èï¹Ì·Ó´Ù:

\vspace{.2 cm}

{\bf Problem A.}  {\sl Characterize the subnormality of
$T_\varphi$ in terms of $\varphi$\,? }

\vspace{.2 cm} ÀÌÁ¦ Toeplitz ÀÛ¿ë¼Ò¿¡ ´ëÇÑ ºÎºÐÁ¤±Ô¼º°ú °ü·ÃµÈ º¸´Ù
±¸Ã¼ÀûÀÎ ¹ÌÇØ°á ¹®Á¦µéÀ» Á¦½ÃÇÑ´Ù.



\newpage \leftline{\large{\it 196}\ \ \ \small Çй® ¿¬±¸ÀÇ
µ¿Çâ°ú ÀïÁ¡ -- ¼öÇÐ} \vspace{.8 cm}

{\bf Problem B.}  {\sl For which $f\in{\bold H}^{\infty}$, is there
$\lambda\ (0<\lambda<1)$ with $T_{f +\lambda\overline{f}}$
subnormal\,? }

\vspace{.2 cm} {\bf Problem C.} {\sl Suppose $\psi$ is as in Theorem
4.1.9 (i.e., the ellipse map). Are there $g\in{\bold H}^{\infty}$,
$g\neq\lambda\psi +c$, such that $T_{\psi+\overline{g}}$ is
subnormal\,? }

\vspace{.2 cm}

{\bf Problem D.} {\sl More generally, if $\psi
\in\bold{H}^\infty$, define
$$
\mathcal{S}(\psi):=\{g\in\bold{H}^\infty:\ T_{\psi+\overline g}\ \mbox{is subnormal}\ \}.
$$
Describe $\mathcal{S}(\psi)$. For example, for which $\psi\in \bold{H}^\infty$, is it balanced?, or is it convex?, or is it weakly closed?
What is ${\rm ext}\,\mathcal{S}(\psi)$\,?
For which $\psi\in \bold{H}^\infty$, is it strictly convex ?, i.e.,
$\partial\mathcal{S}(\psi)\subset{\rm ext}\,\mathcal{S}(\psi)$\,?
}

\vspace{.2 cm} C.\ Cowen \cite{Cow3}Àº ¾î¶² ³íÁõµµ ¾øÀÌ ´ÙÀ½ÀÇ
Èï¹Ì·Î¿î ¾ð±ÞÀ» ÇÏ¿´´Ù: ``{\it If $T_\varphi$ is subnormal then
$\mathcal{E}(\varphi)=\{\lambda\}$ with $|\lambda|<1$".} ±×·¯³ª
¿ì¸®´Â ÀÌ°ÍÀÌ ÂüÀÎÁö ¾Æ´ÑÁö º¸ÀÏ ¼ö°¡ ¾ø¾ú´Ù. ¹°·Ð ¸¸ÀÏ
$T_\varphi$°¡ Á¤±ÔÀ̸é, $\mathcal{E}(\varphi)=\{e^{i\theta}\}$ ÀÌ´Ù.
µû¶ó¼­ ´ÙÀ½Àº ÀÚ¿¬½º·¯¿î Áú¹®ÀÌ´Ù.

\vspace{.2 cm} {\bf Problem E.} {\sl Is the above Cowen's remark
true\,? That is, if $T_\varphi$ is subnormal, does it follow that
$\mathcal{E}(\varphi)=\{\lambda\}$ with $|\lambda|<1$\,? }

\vspace{.2 cm}

¸¸ÀÏ Problem EÀÇ ´ë´äÀÌ ±àÁ¤ÀûÀ̸é, Áï,
À§ÀÇ CowenÀÇ ¾ð±ÞÀÌ ÂüÀ̸é, $\varphi=\overline g+f$¿¡ ´ëÇÏ¿©,
$$
\mbox{$T_\varphi$ is subnormal}\ \Longrightarrow \ \overline
g-\lambda \overline f\in \bold{H}^2\ \ \mbox{with}\ |\lambda|<1\
\Longrightarrow\ g=\overline\lambda f+c.
$$
($c$´Â »ó¼ö). ÀÌ´Â Problem CÀÇ ÇØ´äÀÌ ºÎÁ¤ÀûÀÓÀ» º¸¿©ÁØ´Ù.

\vspace{.2 cm}

ÇÑÆí, Corollary 4.9.3\,À¸·ÎºÎÅÍ $T_\varphi$°¡ $2$-¾ÆÁ¤±ÔÀÌ°í
$\varphi$ ¶Ç´Â $\bar\varphi$°¡ À¯°èÇü½Ä ÇÔ¼öÀ̸é, $T_\varphi$´Â
ºñÀÚ¸í ºÒº¯ ºÎºÐ°ø°£(nontrivial invariant subspace)À» °¡Áø´Ù. ±×·¡¼­
´ÙÀ½ ¹®Á¦°¡ ÀÚ¿¬½º·´°Ô ¶°¿À¸¥´Ù: \vspace{.2 cm}

 {\bf Problem F.} {\sl Does every $2$-hyponormal
Toeplitz operator have a nontrivial invariant subspace\,? More
generally, does every $2$-hyponormal operator have a nontrivial
invariant subspace\,? }

\vspace{.2 cm}


\newpage\rightline{\small À̿쿵: Åä¿¡Çø®Ã÷ ÀÛ¿ë¼Ò¿¡ ´ëÇÑ ºê¸®Áö ÀÌ·Ð\ \ \
\large\it 197} \vspace{0.8cm}

 ¸¸ÀÏ, $T$°¡ ¾ÆÁ¤±ÔÀÌ°í $R(\sigma(T))\ne
C(\sigma(T))$À̸é $T$°¡ ºñÀÚ¸í ºÒº¯ºÎºÐ°ø°£À» °¡Áø´Ù´Â »ç½ÇÀÌ ³Î¸®
¾Ë·ÁÁ® ÀÖ´Ù(\cite{Bro}). ±×·¯³ª $T$°¡ ¾ÆÁ¤±ÔÀÌ°í,
$R(\sigma(T))=C(\sigma(T))$À̸é (Áï, thin spectrumÀ» °¡Áö¸é) $T$°¡
ºñÀÚ¸í ºÒº¯ ºÎºÐ°ø°£À» °¡Áö´ÂÁö´Â ¿©ÀüÈ÷ ¹ÌÇØ°á ¹®Á¦ÀÌ´Ù. ÇÑÆí,
$T\in\mathcal{B(H)}$°¡ von Neumann ÀÛ¿ë¼Ò¶ó°í ÇÔÀº $\sigma(T)$
¹Û¿¡¼­ poleÀ» °¡Áö´Â À¯¸®ÇÔ¼ö $f$¿¡ ´ëÇÏ¿© $f(T)$°¡ normaloid (i.e.,
norm = spectral radius)ÀÓÀ» ¶æÇÑ´Ù. B. Prunaru \cite{Pru}´Â ¸ðµç
´ÙÇ× ¾ÆÁ¤±ÔÀÛ¿ë¼Ò´Â ºñÀÚ¸í ºÒº¯ ºÎºÐ°ø°£À» °¡ÁüÀ» º¸¿´´Ù. ¿ª½Ã
\cite{Ag}¿¡¼­ von Neumann ÀÛ¿ë¼Òµµ ºñÀÚ¸í ºÒº¯ºÎºÐ°ø°£À» °¡ÁüÀ»
º¸¿´´Ù. ´ÙÀ½Àº Problem FÀÇ ºÎºÐ ¹®Á¦ÀÌ´Ù.

\vspace{.2 cm}


{\bf Problem G.} {\sl Is every 2-hyponormal operator with thin
spectrum a von-Neumann operator\,? }

\vspace{.2 cm}

ºñ·Ï, ºÎºÐÁ¤±Ô°¡ ¾Æ´Ñ ´ÙÇ×Á¤±Ô °¡ÁßÀ̵¿ÀÛ¿ë¼ÒÀÇ Á¸Àç°¡ ¾Ë·ÁÁ® ÀÖ±â´Â
ÇÏÁö¸¸ (cf. \cite{CP1}, \cite{CP2}), Åä¿¡Çø®Ã÷ ÀÛ¿ë¼Ò¿¡ ´ëÇÏ¿©
``´ÙÇ×Á¤±Ô $\Rightarrow$ ºÎºÐÁ¤±Ô" ÀÎÁö ¾Æ´ÑÁö´Â ¾Ë·ÁÁ® ÀÖÁö ¾Ê´Ù.
±×·¡¼­ ´ÙÀ½ ¹®Á¦µµ ÀÚ¿¬½º·´°Ô ¶°¿À¸¥´Ù.

\vspace{.2 cm}

{\bf Problem H.} {\sl Does there exist a Toeplitz operator which is
polynomially hyponormal but not subnormal\,? }

\vspace{.2 cm}

ÇÑÆí, \cite{CuL2}¿¡¼­ ¸ðµç ¼ø¼ö(pure -- it has no normal direct
summand) $2$-¾ÆÁ¤±Ô ÀÛ¿ë¼Ò $T$ÀÇ ÀÚ±â-°¡È¯ÀÚÀÇ °è¼ö(rank)°¡ 1À̸é
$T$´Â À̵¿ÀÛ¿ë¼ÒÀÇ ¼±ÇüÇÔ¼öÀÓÀÌ ¹àÇôÁ³´Ù. McCarthy ¿Í Yang
\cite{McCYa}Àº À¯ÇÑ °è¼ö¸¦ °¡Áö´Â ÀÚ±â-°¡È¯ÀÚÀÇ ¸ðµç À¯¸®
¼øȯ(rationally cyclic) ºÎºÐÁ¤±Ô ÀÛ¿ë¼Ò¸¦ Ư¼ºÈ­ÇÏ¿´´Ù. ±×·¯³ª
¿©ÀüÈ÷ À¯ÇÑ °è¼öÀÇ ÀÚ±â-°¡È¯ÀÚ¸¦ °¡Áö´Â ¸ðµç ¼ø¼ö ºÎºÐÁ¤±Ô ÀÛ¿ë¼Ò¸¦
Ư¼ºÈ­ÇÏÁö´Â ¸øÇß´Ù. ±×·¡¼­ ´ÙÀ½ÀÇ Áú¹®ÀÌ Áï½Ã ¶°¿À¸¥´Ù.

\vspace{.2 cm} {\bf Problem I.} {\sl If $T_\varphi$ is a
$2$-hyponormal Toeplitz operator with nonzero finite rank
self-commutator, does it follow that $T_\varphi$ is analytic\,? }

\vspace{.2 cm}

À§ÀÇ ¹ÌÇØ°á ¹®Á¦µéÀº ¸ðµÎ ½±Áö ¾ÊÀº ¹®Á¦µé·Î º¸¿©Áø´Ù. ±×·¯³ª
Toeplitz ÀÛ¿ë¼ÒÀÇ ¼ºÁúÀ» ÃæºÐÈ÷ ÀÌÇØÇϱâ À§Çؼ­´Â À§ ¹®Á¦ÀÇ ´äÀ»
ã´Â °ÍÀº ÇʼöºÒ°¡°áÇÑ ÀÏÀÌ´Ù. ´õ±¸³ª ÀÛ¿ë¼ÒÀÇ ºê¸®Áö ÀÌ·ÐÀÇ ¿Ï¼ºÀ»
À§Çؼ­´Â Toeplitz ÀÛ¿ë¼ÒÀÇ ºÎºÐÁ¤±Ô¼º È®¸³¿¡ ´ëÇÑ ³ë·ÂÀ» ´õ¿í °æÁÖÇÒ
ÇÊ¿ä°¡ ÀÖ´Ù. ¹ÌÇØ°á ¹®Á¦¿¡ ´ëÇÑ µµÀüÀ̾߸»·Î ¼öÇÐ ÃÖ°íÀÇ ¹Ì´öÀÌ ¾Æ´Ò
¼ö ¾ø´Ù

\newpage
\leftline{\large{\it 198}\ \ \ \small Çй® ¿¬±¸ÀÇ µ¿Çâ°ú ÀïÁ¡ --
¼öÇÐ}

\vspace{0.8cm}

ÂüÀ¸·Î, Morris Kline\,ÀÌ Bertrand RusselÀÇ ÇÑ ÀÛÇ°¿¡ Çå»çÇÑ ¼­¹®¿¡¼­
ó·³ ``¹ÌÇØ°á ¹®Á¦¿¡ ´ëÇÑ µµÀüÀº Àηù¸¦ ±ú¾î ÀÖ°Ô ¸¸µé°í ±Ã±ØÀûÀ¸·Î
Àηù¸¦ Á¤½ÅÀû ¹«±â·ÂÀ¸·ÎºÎÅÍ º¸È£ÇÒ °ÍÀÌ´Ù."


%%%%%%%%%%%%%%%%%%%%%%%%%%%%%%%%%%%%%%%%%%%%%%%%%%%%%%%%%%%%%%%%%%%%%%%%%%%%%%%%%%%%%%%%

\begin{thebibliography}{99}\footnotesize


\bibitem[1]{Ab} M. B. Abrahamse,
{\it Subnormal Toeplitz operators and functions of bounded type},
Duke Math. J. {\bf 43} (1976), 597--604.

\bibitem[2]{Ag} J. Agler, {\it An invariant subspace problem}, J. Funct. Anal.
{\bf 38} (1980), 315--323.

\bibitem[3]{Bro} S. Brown, {\it Hyponormal operators with thick spectra have invariant subspaces},
Ann. of Math. {\bf 125} (1987), 93--103.

\bibitem[4]{BH} A. Brown and P. R. Halmos, {\it Algebraic properties of Toeplitz operators},
J. Reine Angew. Math. {\bf 213} (1963-1964), 89--102.


\bibitem[5]{Con2} J. B. Conway, The Theory of Subnormal Operators,
Math. Surveys and Monographs, vol. 36, Amer. Math. Soc., Providence, 1991.


\bibitem[6]{Cow2} C. Cowen, {\it Hyponormal and subnormal Toeplitz operators},
Surveys of Some Recent Results in Operator Theory, I (J. B. Conway
and B. B. Morrel, eds.) Pitman Research Notes in Mathematics,
Longman, Vol.{\bf 171} (1988), 155--167.

\bibitem[7]{Cow3} C. Cowen, {\it Hyponormality of Toeplitz operators}, Proc. Amer. Math. Soc.
{\bf 103} (1988), 809--812.

\bibitem[8]{CoL} C. C. Cowen and J. J. Long, {\it Some subnormal Toeplitz operators},
J. Reine Angew. Math. {\bf 351} (1984), 216--220.

\bibitem[9]{Cu1} R. E. Curto, {\it Quadratically hyponormal weighted shifts},
Integral Equations Operator Theory, {\bf 13} (1990), 49--66.

\bibitem[10]{CHL1}R. E. Curto, I. S. Hwang and W. Y. Lee,
{\it Hyponormality and subnormality of block Toeplitz operators},
Adv. Math. {\bf 230} (2012), 2094--2151.


\bibitem[11]{CHL2}  R. E. Curto, I. S. Hwang and W. Y. Lee,
{\it Which subnormal Toeplitz operators are either normal or
analytic\,?}, J. Funct. Anal. {\bf 263}(8) (2012), 2333--2354.


\newpage  \rightline{\small À̿쿵: Åä¿¡Çø®Ã÷ ÀÛ¿ë¼Ò¿¡ ´ëÇÑ ºê¸®Áö
ÀÌ·Ð\ \ \ \large\it 199}

\vspace{.4 cm}

\bibitem[12]{CLL} R. E. Curto, S. H. Lee and W. Y. Lee, {\it Subnormality and 2-hyponormality for Toeplitz operators},
Integral Equations Operator Theory, {\bf 44} (2002), 138--148.


\bibitem[13]{CuL1} R. E. Curto and W. Y. Lee, {\it Joint hyponormality of Toeplitz pairs},
Memoirs Amer. Math. Soc. No.{\bf 712}, Amer. Math. Soc., Providence,
2001.

\bibitem[14]{CuL2} R. E. Curto and W. Y. Lee, {\it Towards a model theory for $2$--hyponormal operators},
Integral Equations Operator Theory, {\bf 44} (2002), 290--315.

\bibitem[15]{CP1} R. E. Curto and M. Putinar, {\it Existence of non-subnormal polynomially hyponormal operators},
Bull. Amer. Math. Soc. (N.S.), {\bf 25} (1991), 373--378.

\bibitem[16]{CP2} R. E. Curto and M. Putinar, {\it Nearly subnormal operators and moment problems},
J. Funct. Anal. {\bf 115} (1993), 480--497.

\bibitem[17]{FL1} D. R. Farenick and W. Y. Lee, {\it Hyponormality and spectra of Toeplitz operators},
Trans. Amer. Math. Soc., {\bf 348} (1996), 4153--4174.

\bibitem[18]{FF} C. Foia\c s and A. Frazho, {\it The commutant lifting approach to
interpolation problems}, Operator Theory: Adv. Appl., vol.{\bf 44},
Birkh\" auser-Verlag, Boston, 1990.

\bibitem[19]{GGK} I. Gohberg, S. Goldberg and M.A. Kaashoek, {\it Classes Linear Operators},
vol.{\bf II}, Birkh\" auser-Verlag, Boston, 1993.

\bibitem[20]{Ha1} P. R. Halmos, {\it Ten problems in Hilbert space},
Bull. Amer. Math. Soc., {\bf 76} (1970), 887--933.

\bibitem[21]{Ha2} P. R. Halmos, {\it Ten years in Hilbert space},
Integral Equations Operator Theory, {\bf 2} (1979), 529--564.

\bibitem[22]{McCYa} J. E. McCarthy and L. Yang, {\it Subnormal operators and quadrature domains},
Adv. Math. {\bf 127} (1997), 52--72.

\bibitem[23]{NaT} T. Nakazi and K. Takahashi, {\it Hyponormal Toeplitz operators and extremal problems of Hardy spaces}
Trans. Amer. Math. Soc. {\bf 338} (1993), 753--767.

\bibitem[24]{Ni} N. K. Nikolskii, {\it Treatise on the shift operator}, Springer, New York, 1986.

\bibitem[25]{Pru} B. Prunaru, {\it Invariant subspaces for polynomially hyponormal operators},
Proc. Amer. Math. Soc. {\bf 125} (1997), 1689--1691.

\bibitem[26]{Sun} S. Sun, {\it Bergman shift is not unitarily equivalent to a Toeplitz operator},
Kexue Tongbao(English Ed.) {\bf 28} (1983), 1027--1030.


\end{thebibliography}

%%%%%%%%%%%%%%%%%%%%%%%%%%%%%%%%%%%%%%%%%%%%%%%%%%%%%%%%%%%%%%%%%%%%%%%%%%%%%%%%%%%%%%%%%%%%%%%%%
\newpage\leftline{\large{\it 200}\ \ \ \small Çй®
¿¬±¸ÀÇ µ¿Çâ°ú ÀïÁ¡ -- ¼öÇÐ}

 \vspace{0.5cm}

\centerline{\bf Abstract}

\vspace{.3 cm}

\centerline{\large Bridge theory for Toeplitz operators}

\vspace{.3 cm} \centerline{\sc Woo Young Lee}

\vspace{0.4 cm} Toeplitz operators arise naturally in several fields
of mathematics and in a variety of problems in physics. \ Also the
theory of hyponormal and subnormal operators is an extensive and
highly developed area, which has made important contributions to a
number of problems in functional analysis, operator theory, and
mathematical physics. \ Thus, it becomes of central significance to
describe in detail hyponormality and subnormality for Toeplitz
operators. In this sense, the following question is challenging and
interesting:
$$
\hbox{Which Toeplitz operators are hyponormal or subnormal ?}
$$
While the precise relation between normality and subnormality has
been extensively studied, as have been the classes of subnormal and
hyponormal operators, the relative position of the class of
subnormals inside the classes of hyponormals is still far from being
well understood. We call it a `bridge theory' for operators to
explore the gap between hyponormality and subnormality for bounded
linear operators acting on an infinite dimensional complex Hilbert
(or Banach) space. In this survey note, we provide a bridge theory
for Toeplitz operators. This is originated from Halmos' Problem 5
(in 1970): $\hbox{\it Is every subnormal Toeplitz operator either
normal or analytic\,?}$ \ Even though Halmos's Problem 5 was, in
1984, answered in the negative by C. Cowen and J. Long, until now
researchers have been unable to characterize subnormal Toeplitz
operators in terms of their symbols. In this survey note we study
the subnormality and hyponormality of Toeplitz operators acting on
the vector-valued Hardy space $H^2_{\mathbf{C}^n}$ of the unit
circle.

\vspace{.3 cm}

{\footnotesize 2010 Mathematics Subject Classification: 47B20,
47B35, 46J15, 15A83, 30H10, 47A20.

\vspace{.2 cm}

\indent Key words and phrases: Hardy spaces, Toeplitz operators,
Hankel operators, normal, subnormal, hyponormal, $k$-hyponormal. }


\end{document}
\end


%%%%%%%%%%%%%%%%%%%%%%%%%%%%%%%%%%%%%%%%%%%%%%%%%%%%%%%%%%%%%%%%%%%%%%%%%%%%%%%%%%%%%%%%%%%%%%%%%
%%
%%
%%
%%                           END
%%
%%
%%%%%%%%%%%%%%%%%%%%%%%%%%%%%%%%%%%%%%%%%%%%%%%%%%%%%%%%%%%%%%%%%%%%%%%%%%%%%%%%%%%%%%%%%%%%%%%%%
