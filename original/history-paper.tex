%%%%%%%%%%%%%%%%%%%%%%%%%%%%%%%%%%%%%%%%%%%%%%%%%%%%%%%%%%%%%%%%%%%%%%%%%%%%%%%%%%%%%%%%%%%%%%
%
%
%              학술원 총서 평설 논문
%
%                토에플리츠 작용소에 대한 브리지 이론
%                        이 우 영
%                  서울대학교 수리과학부
%
%                     2013년 9월 21일
%
%%%%%%%%%%%%%%%%%%%%%%%%%%%%%%%%%%%%%%%%%%%%%%%%%%%%%%%%%%%%%%%%%%%%%%%%%%%%%%%%%%%%%%%%%%%%%%
\documentclass[12pt,a4paper,2sided]{article}
\usepackage{hangul}
\usepackage{amsmath}
\usepackage{amssymb}
\usepackage{graphicx}
\usepackage{amscd}
\usepackage{amsfonts}
%\usepackage{fancyhdr}

\topmargin 2mm \textwidth 140mm \textheight 230mm

\pagestyle{empty}

\newcommand{\mm}{\multimap}
\newcommand{\sub}{\subset}
\newcommand{\g}{\gamma}
\newcommand{\G}{\Gamma}
\newcommand{\ol}{\overline}
\newcommand{\ul}{\underline}
\newcommand{\ti}{\times}
\newcommand{\la}{\lambda}
\newcommand{\La}{\Lambda}
\newcommand{\bcup}{\bigcup}
\newcommand{\bcap}{\bigcap}
\newcommand{\De}{\Delta}
\newcommand{\de}{\delta}
\newcommand{\back}{\backslash}
\newcommand{\f}{\frak}
\newcommand{\ve}{\varepsilon}
\newcommand{\lan}{\langle}
\newcommand{\ran}{\rangle}
\newcommand{\wh}{\widehat}
\newcommand{\emp}{\emptyset}
\newcommand{\vs}{\vspace}
\newcommand{\hs}{\hspace}


\newtheorem{thm}{Theorem}[subsection]
\newtheorem{df}[thm]{Definition}
\newtheorem{pro}[thm]{Proposition}
\newtheorem{cor}[thm]{Corollary}
\newtheorem{ex}[thm] {Example}
\newtheorem{rema}[thm] {Remark}
\newtheorem{lem}[thm] {Lemma}
\newtheorem{prob}[thm]{Problem}


%%%%%%%%%%%%%%%%%%%%%%%%%%%%%%%%%%%%%%%%%%%%%%%%%%%%%%%%%%%%%%%%%%%%%%%%%%%%%%%%%%%%%%%%%%%%%%%


\begin{document}

\rightline{\large\it 173} \vspace{1.5cm}

\centerline{\Large{\bf 토에플리츠 작용소에 대한 브리지 이론}}
\vspace{1.5 cm} \rightline{\large{\bf 이  우  영}\
{\small(서울대학교 교수)}}

\vspace{.5 cm}
\begin{abstract}
Toeplitz 작용소는 수학과 물리학의 다양한 문제에서 자연스럽게
발생한다. 또한 아정규와 부분정규 작용소 이론은 오늘날 광범위하게 잘
발달된 분야로 성장하였으며 함수해석학, 작용소 이론, 수리물리학 등의
많은 문제를 푸는 데 지대한 기여를 해왔다. 따라서, Toeplitz 작용소에
대한 아정규성과 부분정규성을 상세하게 묘사하는 것은 매우 중요한
문제로 떠올랐다. 그런 의미에서 다음 질문은 흥미롭고 매력적이다: {\it
어떤 Toeplitz 작용소가 아정규이거나 부분정규인가?} 정규성과
부분정규성에 대한 정확한 관계가 광범위하게 연구되어 온 반면에,
아정규성 안에서의 부분정규성의 위치에 관한 연구는 여전히 이해가
부족한 편이다. 무한차원 복소 Hilbert (혹은 Banach) 공간 상의
유계작용소에 관한 아정규성과 부분정규성 사이의 틈을 연구하는 분야를
`브리지 이론'이라 부르는데, 이 글에서는 Toeplitz 작용소에 대한
브리지 이론의 발전을 설명한다. 이는 1970년에 P.\,R.\ Halmos가 제시한
열 개의 문제 중 다음과 같은 다섯번째 문제로부터 야기된다: {\it 모든
부분정규 Toeplitz 작용소는 정규이거나 해석적인가?}
\end{abstract}

\vspace{.5 cm} {\large\bf 1. 서론}

\vspace{.5 cm}

작용소 이론(Operator Theory)은 함수 공간과 그 공간 상에서 작용하는
함수를 다루는 학문이다. 작용소 이론의 연구는 1713년에 B.\ Taylor가
진동하는\linebreak

\vspace{.5 cm} \footnoterule

\vspace{.2 cm} {\footnotesize 2010 Mathematics Subject
Classification: 47B20, 47B35, 46J15, 15A83, 30H10, 47A20.

Key words and phrases: Hardy 공간, Toeplitz 작용소, Hankel 작용소,
정규성, 부분정규성, 아정규성, $k$-아정규성.}

\newpage

\leftline{\large{\it 174}\ \ \ \small 학문 연구의 동향과 쟁점 --
수학}  \vspace{0.8cm}

\noindent 끈을 논의하면서 그 씨앗이 뿌려졌는데 이 때가 Newton의
`프린키피아'가 출간된지 25년쯤 지난 후이다. 현대수학의 역사 치고 꽤
유서가 깊은 학문이라 하겠다.

  그 후 수학이 급격히 발전하게 되고 19세기 초의
Sturm과 Liouville에 의해서 매우 강력한 이론이 만들어졌으며 1910년에
H.\ Weyl의 자기수반 작용소에 관한 중요한 발견들과 20세기 초의 F.\
Volterra, M.\ Riesz, D.\ Hilbert, 괴팅겐학파, Banach, Mazur,
Schauder, 그리고 폴란드의 많은 수학자들에 의하여 풍성한 열매가
열렸다. 또 1930년과 1970년대 사이에 Gelfand, Krein, Naimark,
Kakutani, Kato, 프랑스의 Bourbaki 그룹, 루마니아의 Colojoara, Foias,
미국의 Hille, Phillips, Friedrichs, von Neumann, Paley, Wiener 등이
이 분야의 연구를 더욱 풍성하게 하였다. 오늘날, 작용소 이론이
현대수학에서 차지하는 비중은 실로 대단하다. 우선 수학 분야에서만
보더라도 편미분방정식론, 근사이론, 적분방정식론, 확률론 등에 깊은
영향을 주고 있으며, 응용분야에서는 양자역학, 유체역학, 전자기학,
생명공학 등 다양한 분야로 응용되고 있다.

Toeplitz 작용소에 대하여 끊임없이 관심이 증대되는 이유는 적어도 두
가지가 있다. 첫째는 Toeplitz 작용소가 물리학, 확률론, 정보 및
제어이론, 그 밖의 여러 분야의 다양한 문제들과 밀접한 관련이 있기
때문이며, 둘째는 Toeplitz 작용소가 유계 작용소의 집합 중 가장 큰
집합으로서 작용소 이론, 함수론, Banach 대수 이론 등의 주제들 사이의
매력적인 상호 연관성을 보여주고 풍부한 예들을 제공하기 때문이다.

한편, 1950년에는 Paul Halmos가 아정규성(hyponormality)과
부분정규성\newline\noindent(subnormality)을 도입하였는데, 그 이후로
아정규성과 부분정규성 이론은 작용소론에서 광범위하게 연구되어 매우
훌륭한 이론으로 자리를 잡았다. 그런 의미에서, 다음 문제는 흥미롭고
매력적이다:

\vspace{.2 cm}\centerline {\it 어떤 Toeplitz 작용소가 아정규인가? 또
부분정규인가?}\vspace{.2 cm}

이 글에서는 이 문제에 대한 역사와 발전, 관련된 미해결 문제들을
제시하고자 한다.


%%%%%%%%%%%%%%%%%%%%%%%%%%%%%%%%%%%%%%%%%%%%%%%%%%%%%%%%%%%%%%%%%%%%%%%%%%%%%%%%%%%%%%%%%%%%%%%%%%%%%%

\vspace{.8 cm} {\large\bf 2. 예비 이론}

\vspace{.5 cm}

이  글에서는 모든 Hilbert 공간은 복소 Hilbert 공간으로 이해하며
$\mathcal{H}$를\linebreak

\rightline{\small 이우영: 토에플리츠 작용소에 대한 브리지 이론\ \ \
\large\it 175}

\vspace{.8 cm}

\noindent  가산 Hilbert 공간으로 약속한다. 또한, $\mathcal{B(H)}$는
$\mathcal{H}$ 상의 모든 유계 선형작용소의 집합이라 하고
$\mathcal{K(H)}$는 컴팩트 작용소들의 집합이라 하자. 만일
$T\in\mathcal{B(H)}$이면 $T$의 스펙트럼(spectrum) $\sigma(T)$과 점
스펙트럼(point spectrum) $\sigma_p(T)$은 다음과 같이 정의된다.
\begin{align*}
\sigma(T)&:=\{\lambda\in\mathbb{C}: T-\lambda\ \text{가 가역이 아니다}\};\\
\sigma_p(T)&:=\{\lambda\in\mathbb{C}: T-\lambda\ \text{가 일대일이 아니다}\}.
\end{align*}



만일 $U$가 $\ell^2$ 상의 이동 작용소(the unilateral shift)
$$
U:=
\begin{pmatrix}
0\\
1&0\\
&1&0\\&&1&0\\&&&\ddots&\ddots
\end{pmatrix}
$$
이면 $\sigma_p(T)=\emptyset$이다. \ 양의 유계수열
$\alpha:\alpha_0,\alpha_1,\cdots$이 주어질 때,
$\ell^2(\mathbb{Z}_+)$ 상의 $\alpha$에 대한 가중이동 작용소 (the
unilateral weighted shift) $W_\alpha$는 다음과 같이 정의된다.
$$
W_\alpha e_n:=\alpha_n e_{n+1}\quad \hbox{(모든 $n\ge 0$)},
$$
여기에서 $\{e_n\}_{n=0}^\infty$은 $\ell^2$에 대한
단위직교기저이다. 작용소 $T\in\mathcal{B(H)}$의
수반작용소(adjoint)를 $T^*$로 나타낸다. $T$가 정규(normal)라 함은
$T^*T=TT^*$이고, 아정규(hyponormal)라 함은
자기가환자(self-commutator) $[T^*,T]\equiv T^*T-TT^*\ge 0$이고,
부분정규(subnormal)라 함은 $T=N\vert_{\mathcal{H}}$ (여기에서,
$N$은 어떤 Hilbert 공간 $\mathcal{K}\supseteq \mathcal{H}$ 상에서
정규이다). 명백히, $T$가 부분정규이면 $T$는 아정규이다.

다음은 문헌에서 잘 알려진 아정규 작용소의 기본 성질이다.

\vs{0.3cm}{\bf Proposition 2.0.1.} {\rm (Basic Properties of
Hyponormal Operators)\cite{Con2}}\label{pro3.4} {\sl Let
$T\in\mathcal{L(H)}$ be a hyponormal operator. Then we have:
\begin{itemize}
\item[\rm(a)] If $T\cong S$ then $S$ is also hyponormal;

\item[\rm(b)] $T-\lambda$ is hyponormal for every
$\lambda\in\mathbb{C}$;

\item[\rm(c)] If $T\mathcal{M}\subset\mathcal{M}$ then
$T\vert_\mathcal{M}$ is hyponormal; \end{itemize}}

\newpage
\leftline{\large{\it 176}\ \ \ \small 학문 연구의 동향과 쟁점 --
수학}

 \vspace{0.4cm}

{\sl \begin{itemize} \item[\rm(d)] $||T^*h||\le ||Th||$ for all $h$,
so that $\text{\rm ker}(T-\lambda)\subset \text{\rm
ker}(T-\lambda)^*$;
\item[\rm(e)] If $f$ and $g$ are eigenvectors corresponding to
distinct eigenvalues of $T$ then $f\perp g$; \item[\rm(f)] If
$\lambda\in\sigma_p(T)$ then $\text{\rm ker}\,(T-\lambda)$ reduces
$T$; \item[\rm(g)] If $T$ is invertible then $T^{-1}$ is hyponormal;
\item[\rm(h)] \text{\rm(Stampfli, 1962)} $||T^n||=||T||^n$, so that
$||T||=r(T)$ {\rm (}$r(\cdot)$ denotes spectral radius{\rm )};
\item[\rm(i)] $T$ is isoloid, i.e., $\text{\rm
iso}\,\sigma(T)\subset \sigma_p(T)$; \item[\rm(j)] If $\lambda\notin
\sigma(T)$ then $\text{\rm dist}(\lambda,\,
\sigma(T))=||(T-\lambda)^{-1}||^{-1}$. \item[\rm(k)] \text{\rm
(Berger-Shaw theorem)} If $T$ is cyclic then $\text{\rm
tr}\,[T^*,T]\le\frac{1}{\pi}\mu(\sigma(T))$; \item[\rm(l)] \text{\rm
(Putnam's Inequality)} $||\,[T^*,T]\,||\le\frac{1}{\pi}
\mu(\sigma(T))$.
\end{itemize} }
%\end{pro}



다음도 문헌에서 잘 알려진 부분정규의 특성화이다.

\vs{0.3cm}{\bf Proposition 2.0.2.} {\rm (A Characterization of
Subnormality) \cite{Con2}}\label{thm3.5} {\sl If
$T\in\mathcal{L(H)}$ then the following are equivalent:
\begin{itemize}
\item[\rm(a)] $T$ is subnormal;
\item[\rm(b)] {\rm (Bram-Halmos, 1955)}
$$
\begin{pmatrix}
I&T^*&\hdots& T^{*k}\\
T& T^*T& \hdots& T^{*k}T\\
\vdots&\vdots& \ddots& \vdots\\
T^k& T^*T^k& \hdots& T ^{*k}T^k
\end{pmatrix}
\ge 0\qquad\text{(all $k\ge 1$)}.
$$
\item[\rm(c)]
$$
\begin{pmatrix}
[T^*,T]&[T^{*2},T]&\hdots&[T^{*k},T]\\
[T^*,T^2]&[T^{*2},T^2]&\hdots&[T^{*k},T^2]\\
\vdots&\vdots&\hdots&\vdots\\
[T^*,T^k]&[T^{*2},T^k]&\hdots&[T^{*k},T^k]
\end{pmatrix}
\ge 0\qquad\text{(all $k\ge 1$)}.
$$  \end{itemize}}

\newpage

\rightline{\small 이우영: 토에플리츠 작용소에 대한 브리지 이론\ \ \
\large\it 177} \vspace{.4 cm}

{\sl \begin{itemize}
\item[\rm(d)] {\rm (Embry, 1973)}  There is a positive
operator-valued measure $Q$ on some interval $[0,a]\subset
\mathbb{R}$ such that
$$
T^{*n}T^n=\int t^{2n} dQ(t)\quad\text{for all}\ n\ge 0.
$$
\end{itemize}}
%\end{pro}

위의 조건 (b) (혹은 동치로서 조건 (c))는 아정규와 부분정규 사이의
틈에 대한 측도를 제공한다. 사실, $k=1$에 대한 양의 조건 (b)가 $T$의
아정규성이고, 부분정규는 모든 $k$에 대한 조건 (b)이다. 따라서, (b)에
있는 $(k+1)\times (k+1)$ 작용소 행렬이 양의 행렬일 때, $T$를
$k$-아정규라고 정의할 수 있다. 그러면, Bram-Halmos 특성화는 $T$가
부분정규이기 위한 필요충분조건은 모든 $k$에 대하여 $T$가
$k$-아정규임을 말해준다.

가중이동 작용소의 부분정규성은 측도와 관련된 모멘트 문제로 판정될 수 있다.

\vs{0.2cm}{\bf Proposition 2.0.2.} {\rm (Berger's
Theorem)}\label{thm3.6} {\sl Let $T\equiv W_\alpha$ be a weighted
shift with weight sequence $\alpha\equiv\{\alpha_n\}$ and define the
moment of $T$ by
$$
\gamma_0:=1\quad\text{and}\quad \gamma_n:=\alpha_0^2\alpha_1^2\cdots \alpha_{n-1}^2\ (n\ge 1).
$$
Then $T$ is subnormal if and only if there exists a probability
measure $\nu$ on $[0,||T||^2]$ such that}
\begin{equation}\label{2.0.1}
\gamma_n=\int_{[0,||T||^2]} t^{n} d\nu(t)\quad (n\ge 1).
\end{equation}
%\end{pro}

\vs{0.2cm} 한편,  $T\in\mathcal{B(H)}$가 {약 $k$-아정규}(weakly
$k$-hyponormal)라 함은 다음의 집합
$$
LS((T,T^2,\cdots,T^k)):=\left\{\sum_{j=1}^k \alpha_jT^j: \alpha=
(\alpha_1,\cdots,\alpha_k)\in {\mathbb C}^k\right\}
$$
가 아정규 작용소로 이루어지는 것이다. 만일 $k=2$이면, $T$를
2차-아정규 (quadratically hyponormal)라 하고, $p(T)$가 모든 다항식
$p\in{\mathbb C}[z]$에 대하여 아정규이면 $T$를 다항정규(polynomially
hyponormal)라 한다. 일반적으로, $k$-아정규 $\Rightarrow$ 약
$k$-아정규이고, 그 역은 성립하지 않는다. 약-아정규 작용소는 아정규와
부분정규 사이의 틈을 연결하는 시도에서 많은 연구가 이루어졌다.

\newpage\leftline{\large{\it 178}\ \ \ \small 학문 연구의
동향과 쟁점 -- 수학}

 \vspace{0.8cm}

이제, $P$를 ${\bold L}^2(\mathbb T)\equiv{\bold L}^2$에서 ${\bold
H}^2(\mathbb T)\equiv {\bold H}^2$ 위로의 직교사영이라고 하자.
그러면 함수 $\varphi\in{\bold L}^{\infty}(\mathbb T)\equiv {\bold
L}^{\infty}$에 대하여, 심벌 $\varphi$를 가지는 Toeplitz 작용소
$T_\varphi$는 다음과 같이 정의된다:
$$
T_\varphi f=P(\varphi f)\quad\text{($f\in {\bold H}^2$)}.
$$
한편, $\{z^n : n=0,1,2,\cdots \}$이 ${\bold H}^{2}$의 정규직교기저라
하고 $\varphi\in{\bold L}^{\infty}$가 다음과 같은 Fourier 계수를
가진다고 하자:
$$
\widehat\varphi(n)=\frac{1}{2\pi}\int_{0}^{2\pi} \varphi{\overline z}^n dt
$$
그러면 기저 $\{z^n : n=0,1,2,\cdots\}$에 관한 $T_\varphi$의 행렬 $(a_{ij})$는 다음과 같다:
$$
a_{ij}=(T_{\varphi}z^j,z^i)=\frac{1}{2\pi}\int_{0}^{2\pi} \varphi \overline{z}^{i-j} dt=\widehat\varphi(i-j).
$$
즉, $T_\varphi$에 대한 행렬은 대각선이 모두 일정한 행렬이다:
$$
(a_{ij})=\begin{pmatrix}
c_0 & c_{-1}& c_{-2}& c_{-3}& \cdots \\
c_1 & c_{0}& c_{-1}& c_{-2}& \cdots \\
c_2 & c_{1}& c_{0}& c_{-1}& \cdots \\
c_3 & c_{2}& c_{1}& c_{0}& \cdots \\
\vdots& \ddots& \ddots& \ddots& \ddots
\end{pmatrix}, \quad \hbox{(여기서, $c_{j}=\widehat\varphi(j)$)}:
$$
이러한 행렬을 Toeplitz 행렬이라고 한다.



%%%%%%%%%%%%%%%%%%%%%%%%%%%%%%%%%%%%%%%%%%%%%%%%%%%%%%%%%%%%%%%%%%%%%%%%%%%%%%%%%%%%%%%%%%%%%%%%


\vspace{.8 cm} {\large\bf 3. Toeplitz 작용소의 아정규성}

\vspace{.5 cm}


1988년에, C. Cowen \cite{Cow3}의 우아하고 유용한 정리가 단위원
${\mathbb T}\subset{\mathbb C}$의 Hardy 공간 $\bold H^2$ 상의
Toeplitz 작용소의 아정규성을 그의 심벌 $\varphi\in \bold
L^{\infty}$을 써서 특성화하였다. 이 결과는 작용소 이론으로부터
떠오른 대수적 문제 -- 즉, $T_{\varphi}$가 아정규인가? -- 의 해답이
함수 $\varphi$를 연구함으로써 가능하게 되었다. 정규 Toeplitz
작용소는 1960년대에 A.\ Brown과 P.\,R.\ Halmos \cite{BH}가 그 심벌을
써서 특성화하였다. 그런데 다소 놀랍게도 아정규 Toeplitz 작용소의
심벌에 의한 특성화(Cowen의 정리)가 발견되기까지 무려 25년이
흘러갔다. 이는 Cowen이 그의 해설논문 \cite{Cow2}에서 지적하였듯이
아마도 1970년대와 1980년대에 부분정규\linebreak

\newpage
\rightline{\small 이우영: 토에플리츠 작용소에 대한 브리지 이론\ \ \
\large\it 179}
\vspace{.8 cm}


\noindent  Toeplitz 작용소에 대한 뜨거운 연구 분위기가 아정규의
연구를 뒤늦게 했을 것이라는 추측이 가능하다. Cowen의 정리에 의한
아정규성의 특성화는 $\bold H^{\infty}$의 단위구에서의 어떤
함수방정식을 푸는 것이다. 그러나 이는 이론적으로 가능하다는 것이지
실제로 이 함수방정식을 푸는 문제는 무척 어렵다. 참으로, 심벌
$\varphi$에 관한 어떤 제한이 없다면 $T_{\varphi}$의 아정규성을
$\varphi$의 Fourier 계수로 나타낸다는 것은 어쩌면 불가능할 수도
있다. 이 절에서는 이 연구의 최근의 발전 상황을 살펴본다.

\vspace{.3 cm} {\bf 3.1. Cowen의 정리} \vspace{.3 cm}

이 절에서는 Cowen 정리를 제시한다. Cowen의 정리는 Toeplitz 작용소의
아정규성에 대한 작용소론의 문제를 그 심벌과 관련된 함수방정식의 해를
발견하는 문제로 바꾸어 준다. 이와 같은 접근법은 Toeplitz 작용소
연구의 많은 논문에 등장하였다.

\vs{0.3cm} 다음의 두 결과는 \cite{BH}에서 알려진 사실이다.

\vs{0.3cm}{\bf Lemma 3.1.1.} {\sl A necessary and sufficient
condition that two Toeplitz operators commute is that either both be
analytic or both be co-analytic or one be a linear function of the
other.}

\vs{0.3cm}{\bf Theorem 3.1.2.} {\rm(Brown-Halmos) \cite{BH}} {\sl
Normal Toeplitz operators are translations and rotations of
hermitian Toeplitz operators, i.e.,
$$
T_\varphi\ \mbox{normal}\ \Longleftrightarrow\ \exists\ \alpha,\beta\in\mathbb{C},
\ \mbox{a real valued}\ \psi\in {\bold L}^{\infty}\ \mbox{s.t.}\
T_{\varphi}=\alpha T_{\psi} +\beta 1.
$$}

함수 $\psi\in{\bold L}^{\infty}$에 대하여, ${\bold H}^2$ 상의 Hankel
작용소 $H_\psi$는 다음과 같이 정의된다:
$$
H_{\psi}f=J(I-P)(\psi f)\quad (f\in {\bold H}^{2}),
$$
여기에서 $J$\,는 아래와 같이 정의된 $({{\bold H}^2})^{\perp}$\,에서
${\bold H}^2$ 위로의 유니터리 작용소이다:
$$
J(z^{-n})=z^{n-1}\ (n\ge 1).
$$
\medskip
만일 $\psi$가 Fourier 급수 $\psi:=\sum_{n=-\infty}^{\infty}a_n
z^n$로 표현되면, $H_\psi$의 행렬은 다음과 같이 주어진다:

\bigskip\bigskip

\newpage\leftline{\large{\it 180}\ \ \ \small 학문 연구의
동향과 쟁점 -- 수학}  \vspace{0.3cm}

$$
H_\psi \equiv
\begin{pmatrix}
a_{-1}&a_{-2}&a_{-3}&\cdots\\
a_{-2}&a_{-3}& & \\
a_{-3}& &\ddots&  \\
\vdots& & &\ddots \\
\end{pmatrix}.
$$

다음은 Hankel 작용소의 기본 성질들이다.
\begin{itemize}
\item[1.]  $H_{\psi}^*=H_{{\psi}^*}$;
\item[2.]  $H_{\psi}U=U^*H_{\psi}$ ($U$는 이동작용소);
\item[3.]  ${\rm Ker}H_{\psi}=\{0\}$ 또는 $\theta{\bold H}^2$ (어떤 내적(inner)함수 $\theta$) (Beurling's theorem에 의해);
\item[4.]  $T_{\varphi\psi}-T_{\varphi}T_{\psi}=H_{\overline{\varphi}}^*H_{\psi}$;
\item[5.]  $H_{\varphi}T_{h}=H_{\varphi h}=T_{h^*}^*H_{\varphi}\ (h\in{\bold H}^{\infty})$.
\end{itemize}

1988년에 Carl Cowen이 Toeplitz 작용소 $T_\varphi$의 아정규성을 심벌
$\varphi$에 의해 특성화하는 데 성공하였다.

\vs{0.2cm}{\bf Theorem 3.1.3.} (Cowen's Theorem)\ [7]\ {\sl If
$\varphi\in{\bold L}^{\infty}$ is such that \linebreak
$\varphi=\overline{g} +f$ {\rm (}$f,g\in {\bold H}^2${\rm )}, then
$$
T_\varphi\ \mbox{is hyponormal}\ \Longleftrightarrow\ g=c+T_{\overline{h}}f
$$
for some constant $c$ and some $h\in{\bold
H}^{\infty}(\mathbb{D})$ with $||h||_{\infty}\le 1$.}
%\end{thm}


\vs{0.2cm}{\bf Theorem 3.1.4.} (Nakazi-Takahashi Variation) [23]
{\sl For $\varphi\in{\bold L}^{\infty}$, put
$$
\mathcal{E} (\varphi):=\{k\in{\bold H}^{\infty}:||k||_{\infty}\le 1\
\mbox{and}\ \varphi-k{\overline{\varphi}}\in{\bold H}^{\infty}\}.
$$
Then $T_\varphi$ is hyponormal if and only if} $\mathcal{E}
(\varphi)\neq\varnothing$.
%\end{thm}


\vspace{.3 cm}

{\bf 3.2. 삼각다항 심벌의 경우} \vspace{.3 cm}

이 분절에서는 삼각다항 심벌을 가진 Toeplitz 작용소의 아정규성을
살펴본다. 이를 위하여  팽창이론(dilation theory)을 먼저 살펴본다.
만일 $B=\begin{pmatrix} A&*\\ *&* \end{pmatrix}$이면, $B$를 $A$의
팽창이라 하고, $A$를 $B$의 축소라고 한다. 모든 수축 \linebreak

\rightline{\small 이우영: 토에플리츠 작용소에 대한 브리지 이론\ \ \
\large\it 181}

\vspace{.8 cm} \noindent 작용소(contraction)는 유니터리 팽창을
가진다는 것이 잘 알려져 있다. 실제로, $||A||\le 1$이면,
$$
B\equiv
\begin{pmatrix} A&(I-AA^*)^{\frac{1}{2}}\\(I-A^*A)^{\frac{1}{2}}&-A^*\end{pmatrix}
$$
는 유니터리이다. 한편, 모든 자연수 $n$에 대하여 $B^n$이 $A^n$의
팽창이면, 작용소 $B$를 $A$의 강팽창(strong dilation)이라고 한다.
따라서 만일 $B$가 $A$의 강팽창이면, $B$는 $B=\begin{pmatrix} A &0\\
*&*\end{pmatrix}$의 꼴로 나타난다. 종종 $B$를 $A$의
리프팅(lifting)이라 하고, $A$가 $B$로 승강된다고 말한다. 모든 수축
작용소가 등장(isometric) 강팽창을 가진다는 것이 알려져 있다. 사실
수축 작용소 $A$의 최소 등장 강팽창은 다음과 같이 주어진다:
$$
B\equiv
\begin{pmatrix}A&0&0&0&\cdots\\
(I-A^*A)^{\frac{1}{2}}&0&0&0&\cdots\\
0&I&0&0&\cdots\\
0&0&I&0&\cdots\\
\vdots&\vdots&\vdots&\ddots&
\end{pmatrix}.
$$

그러면  다음의 유명한 정리를 얻을 수 있다.


\vs{0.2cm}{\bf Theorem 3.2.1.} (Commutant Lifting Theorem) {\rm
\cite[p.658]{GGK}} {\sl  Let $A$ be a contraction and $T$ be a
minimal isometric dilation of $A$. If $BA=AB$ then there exists a
dilation $S$ of $B$ such that}
$$
S=\begin{pmatrix} B &0\\ *&*\end{pmatrix},\quad ST=TS,\quad \mbox{and}\quad ||S||=||B||.
$$
%\end{thm}


이제 다음의 보간문제(interpolation problem) -- 소위 Carath\'
eodory-Schur 보간문제로 불림 -- (CSIP)를 살펴보자. 복소수 $c_0,
\cdots, c_{N-1}$이 주어질 때, 다음을 만족하는 $\mathbb{D}$ 상의
해석함수 $\varphi$를 찾아라:
\begin{itemize}
\item[(i)] $\widehat{\varphi}(j)=c_j\ (j=0,\cdots, N-1)$;
\item[(ii)] $||\varphi||_{\infty}\le 1$.
\end{itemize}

그러면 CSIP의 해는 다음과 같이 주어진다.

\newpage\leftline{\large{\it 182}\ \ \ \small 학문 연구의
동향과 쟁점 -- 수학}

\vspace{0.8cm}



{\bf Theorem 3.2.2.} {\rm \cite{FF}} {\sl
$$
\mbox{CSIP is solvable}\ \Longleftrightarrow\ C\equiv
\begin{pmatrix}
c_0\\
c_1&c_0&&\mbox{\rm{\Huge{O}}}\\
c_2&c_1&c_0\\
\vdots&\vdots&\ddots&\ddots\\
c_{N-1}&c_{N-2}&\cdots&c_1&c_0
\end{pmatrix}\ \mbox{is a contraction.}
$$
Moreover, if $\varphi$ is a solution if and only if  $T_\varphi$
is a contractive lifting of $C$ which commutes with the unilateral
shift.}
%\end{thm}

\vs{0.3cm} 이제 $\varphi$가 다음과 같은 삼각다항함수라고 하자:
$$
\varphi(z)=\sum_{n=-N}^{N} a_{n}z^n\ (a_{N}\neq 0).
$$
함수 $k\in{\bold H}^{\infty}$가 $\varphi -k\overline{\varphi}\in {\bold H}^{\infty}$
를 만족하면, $k$는 다음을 만족해야 한다.
\begin{equation}\label{3.2.1}
k\sum_{n=1}^{N}\overline{a_n} z^{-n} -\sum_{n=1}^{N}a_{-n}z^{-n}\in{\bold H}^{\infty}.
\end{equation}
식 (\ref{3.2.1})로부터 우리는 Fourier 계수 $\widehat{k}(0),\cdots
,\widehat{k}(N-1)$가 $\widehat{k}(n)=c_n\ (n=0,1,\cdots,N-1)$을
만족하도록 하는 것을 구할 수 있다. 여기에서,
$c_0,c_1,\cdots,\linebreak c_{N-1}$은 $\varphi$의 계수로부터
유일하게 다음 관계식을 써서 얻어진다.
\begin{equation}\label{3.2.2}
\begin{pmatrix}c_0\\c_1\\ \vdots\\ \vdots\\c_{N-1}\end{pmatrix}=
\begin{pmatrix}\overline{a_1}&\overline{a_2}&\overline{a_3}&\cdots&\overline{a_N}\\
\overline{a_2}&\overline{a_3}&\hdots&\cdot\\
\overline{a_3}&\hdots&\cdots\\
\vdots&\cdots & &\mbox{{\Huge{O}}}\\
\overline{a_N}
\end{pmatrix}^{-1}
\begin{pmatrix}
a_{-1}\\ a_{-2}\\ \vdots\\ \vdots\\ a_{-N}
\end{pmatrix}.
\end{equation}
따라서, 만일 $k(z)=\sum_{j=0}^{\infty} c_{j}z^j$가 ${\bold
H}^{\infty}$에 속하면 다음이 성립한다.
$$
\varphi -k\overline{\varphi}\in {\bold H}^{\infty}\ \Longleftrightarrow
\ c_0,c_1,\cdots ,c_{N-1}\mbox{이 (\ref{3.2.2})에 의해 주어진다}.
$$
그러면  Cowen의 정리에 의해, 만일 $ c_0,c_1,\cdots ,c_{N-1}$이
(\ref{3.2.2})으로 주어지면,\linebreak $T_\varphi$의 아정규성은
다음을 만족하는 함수 $k\in{\bold H}^{\infty}$의 존재와 동치이다:
$$
\begin{cases}\widehat{k}(j)=c_j\ (j=0,\cdots , N-1)\\||k||_{\infty}\le 1
\end{cases}
$$


\newpage\rightline{\small 이우영: 토에플리츠 작용소에 대한 브리지 이론\ \ \
\large\it 183} \vspace{.8 cm}


이는 정확히 CSIP와 같은 문제이다. 따라서 다음을 즉시 얻는다.


\vs{0.2cm}{\bf Theorem 3.2.3.} {\sl If $\varphi(z)=\sum_{n=-N}^{N}
a_{n}z^n$, where $a_N\neq 0$ and if $c_0,c_1,\cdots,\linebreak
c_{N-1}$ are given by (\ref{3.2.2}) then
$$
T_\varphi\ \mbox{is hyponormal}\ \Longleftrightarrow\ C\equiv
\begin{pmatrix}
c_0\\
c_1&c_0&&\mbox{\rm{\Huge{O}}}\\
c_2&c_1&c_0\\
\vdots&\vdots&\ddots&\ddots\\
c_{N-1}&c_{N-2}&\cdots&c_1&c_0
\end{pmatrix}\ \mbox{is a contraction.}
$$}
%\end{thm}


%%%%%%%%%%%%%%%%%%%%%%%%%%%%%%%%%%%%%%%%%%%%%%%%%%%%%%%%%%%%%%%%%%%%%%%%%%%%%%%%%%%%%%%%%%%%%%%%%%%%%%
\vs{0.3cm} { \bf 3.3 유리함수 심벌의 경우}

\vs{0.3cm} 함수 $\varphi\in \bold L^\infty$가 유계형(bounded
type)이라 함은 $\bold H^\infty (\mathbb{D})$ 상의 다음을 만족하는
함수 $\psi_1, \psi_2$가 존재할 때이다:
$$
\varphi(z)=\frac{\psi_1(z)}{\psi_2(z)}\quad\hbox{(거의 모든 $z\in\mathbb{T}$에 대하여)}
$$
명백히 유리함수는 유계형이다. 이제, $\theta$가 내적함수(inner
function)라 하자. 만일 $\theta$가 다음의 유한 Blaschke 곱(finite
Blaschke product)이면,
$$
\theta(z)=e^{i\xi}\prod_{j=1}^n
\frac{z-\beta_j}{1-\overline{\beta_j}z}\quad \hbox{($|\beta_j|<1$,
$j=1,\cdots,n$)},
$$
$\theta$의 차수(degree) $\text{deg}(\theta)$는 개 단위원판 $\mathbb
D$ 안의 근의 개수로 정의되며, 그렇지 않으면 무한대로 정의한다. 이제,
내적함수 $\theta$에 대하여 다음과 같이 쓰자:
$$
\mathcal H(\theta):=\bold H^2 \ominus \theta \bold H^2.
$$

그러면, $f\in \bold H^2$에 대하여 다음이 성립한다.

\begin{align*}
\langle [T_\varphi^*,T_\varphi]f,f\rangle
=||T_\varphi f||^2-||T_{\overline{\varphi}}f||^2
&=||\varphi f||^2-||H_\varphi f||^2-(||\overline{\varphi} f||^2-||H_{\overline{\varphi}} f||^2)\\
&=||H_{\overline\varphi}f||^2-||H_\varphi f||^2.
\end{align*}

따라서 다음을 얻는다.
$$
T_\varphi\ \text{hyponormal}\ \Longleftrightarrow\
||H_{\overline\varphi} f||\ge ||H_\varphi f||\quad (f\in \bold H^2).
$$
\newpage\leftline{\large{\it 184}\ \ \ \small 학문 연구의
동향과 쟁점 -- 수학}


\vspace{0.8cm}

이제 $\varphi = \overline{g}+ f \in \bold L^{\infty}$ ($f,g\in\bold
H^2$)라 하자. 그러면 $H_\varphi U=U^*H_\varphi$
($U=$이동작용소)이므로, Beurling 정리로부터 다음을 얻는다.
$$
\text{ker}\, H_{\overline f}=\theta_0 \bold H^2\quad\text{and}\quad \text{ker}\,H_{\overline g}
=\theta_1 \bold H^2\quad\text{(어떤 내적함수 $\theta_0, \theta_1$에 대하여)}.
$$
만일 $T_\varphi$가 아정규이면, $||H_{\overline f} h||\ge ||H_{\overline g} h||$
($h\in \bold H^2$)이므로 다음을 얻는다.
\begin{equation}\label{3.2.4}
\theta_0 \bold H^2=\text{ker}\, H_{\overline f}\subset \text{ker}\, H_{\overline g}=\theta_1 \bold H^2.
\end{equation}
이는 $\theta_1$가 $\theta_0$을 나눈다는 것을 의미하며, 따라서
$\theta_0=\theta_1\theta_2$ ($\theta_2$는 내적함수). 한편, 함수
$f\in \bold H^2$에 대하여 $\overline{f}$가 유계형이면, 즉,
$\overline{f}=\psi_2/\psi_1$ ($\psi_i\in \bold H^\infty$)이면
$\psi_1$의 외적 부분(outer part)을  $\psi_2$로 나누어
$\overline{f}=\psi/\theta$ ($\theta$는 내적함수, $\psi\in \bold
H^\infty$)이다. 따라서, $f=\theta\overline{\psi}$로 나타난다. 그러나
$f\in \bold H^2$이므로, $\psi\in\mathcal{H}(\theta)$이다. 따라서
$f\in \bold H^2$이고 $\overline{f}$이 유계형이면 다음과 같이
쓰여진다.
\begin{equation}\label{3.2.5}
f=\theta\overline{\psi}\quad (\theta\ \text{inner},\ \psi\in \mathcal{H}(\theta)).
\end{equation}
그러므로 만일 $\varphi=\overline g+f$이 유계형이고 $T_\varphi$가
아정규이면, (\ref{3.2.4})와 (\ref{3.2.5})에 의하여 다음과 같이 쓸 수
있다:
$$
f=\theta_1\theta_2\overline{a}\quad\text{and}\quad g=\theta_1\overline b,
$$
(여기에서, $a\in \mathcal{H}(\theta_1\theta_2)$,
$b\in\mathcal{H}(\theta_1)$이다).


다음 보조정리는 \cite{CuL1}으로부터 나온다.

\vs{0.2cm}{\bf Lemma 3.3.1.} {\sl Let $\varphi = \overline{g}+ f \in
\bold L^{\infty}$, where $f$ and $g$ are in $\bold H^2$. Assume that
\begin{equation}\label{3.2.6}
f= \theta_1 \theta_2 \overline{a}\quad\text{and}\quad g = \theta_1
\overline{b}
\end{equation}
for $a \in \mathcal H (\theta_1 \theta_2)$ and $b \in \mathcal H
(\theta_1)$. Let $\psi: = \theta_1 \overline{P_{ \mathcal H
(\theta_1)}(a)} + \overline{g}$. Then $T_{\varphi}$ is hyponormal
if and only if $T_{\psi}$ is.}
%\end{lem}

\vs{0.2cm} Lemma 3.3.1의 관점에서, 유계형 심벌을 가진 Toeplitz
작용소의 아정규성을 연구할 때, 심벌 $\varphi= \overline{g}+ f \in
\bold L^{\infty}$는 다음과 같이 쓸 수 있다.
\begin{equation}\label{3.2.7}
f=\theta \overline{a},\quad \quad g=\theta \overline{b},
\end{equation}

\newpage\rightline{\small 이우영: 토에플리츠 작용소에 대한 브리지 이론\ \ \
\large\it 185}

 \vspace{.8 cm} \noindent (여기에서, $\theta$는
내적함수이고 $a,b \in \mathcal H (\theta)$와 $\theta$\,는 서로
소이다). 한편, $f \in \bold H^{\infty}$를 유리함수라고 하자. 그러면
다음과 같이 쓸 수 있다.
$$
f=p_m(z)+\sum_{i=1}^n \sum_{j=0}^{l_i -1}
\frac{a_{ij}}{(1-\overline{\alpha_i}z)^{l_i -j}} \quad
(0<|\alpha_i|<1),
$$
(여기에서, $p_m(z)$는 차수가 $m$인 다항식이다). 이제 $\theta$가
다음과 같은 유한 Blaschke 곱이라고 하자:
$$
\theta=z^{m} \prod_{i=1}^{n} \left(\frac{z- \alpha_i}{1- \overline{\alpha_i}z}\right)^{l_i}.
$$
다음을 관찰하자.
$$
\frac{a_{ij}}{1-\overline{\alpha_i}z}=
\frac{\overline{\alpha_i}a_{ij}}{1 - |\alpha_i|^2}
\Bigl(\frac{z-\alpha_i}{1- \overline{\alpha_i}z}
+\frac{1}{\overline{\alpha_i}} \Bigr).
$$
따라서 $f \in \mathcal H (z\theta)$이다. 이제 $a := \theta
\overline{f}$라 하면, $a \in \mathcal H (z \theta)$이고  $f= \theta
\overline{a}$이다. 따라서 만일 $\varphi = \overline{g} + f \in \bold
L^{\infty}$ ($f,g$는 유리함수)이고 $T_{\varphi}$가 아정규이면 다음과
같이 쓸 수 있다.
$$
f=\theta \overline{a},\quad\quad g=\theta \overline{b}
$$
($\theta$\,는 유한 Blaschke 곱이고 $\theta (0)=0$, $a,b \in \mathcal
H (\theta)$). 이제 $\theta$\,를 차수 $d$\;의 유한 Blaschke 곱이라
하면 다음과 같이 쓸 수 있다.
\begin{equation}\label{3.2.8}
\theta = e^{i \xi} \prod_{i=1}^n B_i ^{n_i},
\end{equation}

\noindent (여기에서, $B_i(z) := \frac{z- \alpha_i}{1-
\overline{\alpha_i}z}$, $|\alpha_i| < 1$, $n_i \geq 1$,
$\sum_{i=1}^n n_i =d$). 그리고 $\theta =e^{i\xi} \prod_{j=1}^d B_j
$라 하고 $\theta$\,의 각 근이 그의 계수(multiplicity)만큼 반복된다고
하자. 그러면 이 Blaschke 곱은 정확히 (\ref{3.2.8})에 있는 Blaschke
곱과 같다. 다음과 같이 두자.
\begin{equation}\label{3.2.9}
\phi_j :=\frac{d_j}{1- \overline{\alpha_j}z}\,B_{j-1}B_{j-2} \cdots
B_1 \quad (1 \leq j \leq d)
\end{equation}

\noindent (여기에서, $\phi_1 :=d_1(1- \overline{\alpha_1}z)^{-1}$,
$d_j:=(1-|\alpha_j |^2)^{\frac{1}{2}}$). 그러면 $\{\phi_j \}_1^d$가
$\mathcal H (\theta)$의 정규직교기저를 이룸이 잘 알려져 있다 (cf.
\cite[Theorem X.1.5]{FF}). 이제 $\varphi=\overline g+f\in \bold
L^\infty$ (여기에서, $g=\theta \overline b$, $f=\theta \overline a$,
$a, b\in \mathcal H (\theta)$)라 하고 다음과 같이 쓰자:
$$
\mathcal{C}(\varphi):= \{k \in \bold H^{\infty} : \varphi -k
\overline{\varphi} \in \bold H^{\infty}\}.
$$
\newpage\leftline{\large{\it 186}\ \ \ \small 학문 연구의
동향과 쟁점 -- 수학}

\vspace{.8 cm} \noindent 그러면, $k$가 $\mathcal{C}(\varphi)$ 안에
있기 위한 필요충분조건은 $\overline {\theta} b-k \overline{\theta}a
\in \bold H^2$이다. 즉,
\begin{equation}\label{3.2.10}
b-ka \in \theta \bold H^2 .
\end{equation}
그러면, $\theta^{(n)} (\alpha_i) =0$ (모든 $0 \leq n  < n_i$).
따라서 조건 (\ref{3.2.10})은 다음과 동치이다:
모든 $1 \leq i \leq n,$에 대하여,
\begin{equation}\label{3.2.11}
\begin{pmatrix} k_{i,0}\\
k_{i,1}\\
k_{i,2}\\
\vdots\\
k_{i,n_i -2}\\
k_{i,n_i -1}
\end{pmatrix}
=\begin{pmatrix} a_{i,0}&0&0&0&\cdots&0\\
a_{i,1}&a_{i,0}&0&0&\cdots&0\\
a_{i,2}&a_{i,1}&a_{i,0}&0&\cdots&0\\
\vdots&\ddots&\ddots&\ddots&\ddots&\vdots \\
a_{i,n_i -2}&a_{i,n_i -3}&\ddots&\ddots&a_{i,0}&0\\
a_{i,n_i -1}&a_{i,n_i -2}&\hdots&a_{i,2}&a_{i,1}&a_{i,0}
\end{pmatrix}^{-1}
\begin{pmatrix} b_{i,0}\\
b_{i,1}\\
b_{i,2}\\
\vdots\\
b_{i,n_i -2}\\
b_{i,n_i -1}
\end{pmatrix},
\end{equation}
여기에서,
$$
k_{i,j}:= \frac{k^{(j)}(\alpha_i)}{j!},\quad a_{i,j}:=
\frac{a^{(j)}(\alpha_i)}{j!} \quad\text{and}\quad
b_{i,j}:=\frac{b^{(j)}(\alpha_i)}{j!}.
$$
역으로, $k \in \bold H^{\infty}$가 (\ref{3.2.11})을 만족하면 $k$는
$\mathcal{C}(\varphi)$에 속한다. 따라서, $k$가
$\mathcal{C}(\varphi)$에 속하기 위한 필요충분조건은  $k$가 다음을
만족하는 $\bold H^{\infty}$ 안의 함수이다:
\begin{equation}\label{3.2.12}
\frac{k^{(j)}(\alpha_i)}{j!} = k_{i,j} \qquad (1 \leq i \leq n, \
0 \leq j < n_i),
\end{equation}
(여기에서, $k_{i,j}$는 방정식 (\ref{3.2.11})에 의해 주어진다).
더구나 $||k||_\infty\le 1$\,이라면 이는 정확히  classical
Hermite-Fej\' er Interpolation Problem (HFIP)이다. 그러므로 다음을
얻는다.

\medskip{\bf Theorem 3.3.2.}\ {\sl Let $\varphi=\overline{g}+f\in
\bold L^\infty$, where $f$ and $g$ are rational\linebreak functions.
Then $T_\varphi$ is hyponormal if and only if the corresponding HFIP
$(12)$ is solvable.}


\vspace{0.2cm} 이제 우리는 Toeplitz 작용소 $T_\varphi$의 아정규성에
대한 계산가능한 판정법이 심벌 $\varphi$가 삼각다항함수이거나
유리함수이면 보간문제의 해에 의하여 얻어질 수 있다고 말할 수 있다.

\newpage\rightline{\small 이우영: 토에플리츠 작용소에
대한 브리지 이론\ \ \ \large\it 187}

\vspace{0.8cm}

{\bf 3.4. 유계형 심벌의 경우}

\vspace{0.3cm}

이 분절은 Toeplitz 작용소의 아정규성에 관하여 저자와 저자의
공저자들이 얻은 가장 최근의 결과로서, 일반적인 심벌인 유계형 심벌에
대한 판정법을 제시할 것이다. 이 결과가 Cowen의 정리를 제외하고
지금까지 얻어진 Toeplitz 작용소의 아정규성에 관한 가장 일반적인
결과라 할 수 있을 것이다. 우선 이를 알아보기 전에 이동작용소 $U
\equiv T_z$의 압축(compression)에 대한 삼각행렬 표현을 알아보자.
내적함수 $\theta$에 대하여, $U_\theta$를 다음과 같이 정의하자.
\begin{equation}\label{3.1}
U_\theta= P_{\mathcal H (\theta)} U \vert_{\mathcal H (\theta)}.
\end{equation}
이제 세 가지 경우로 나누어 생각한다.

\vspace{0.3cm}

\noindent {\bf 경우 1} : $B$를 Blaschke 곱이라 하고
$\Lambda:=\{\lambda_n : n \geq 1\}$를 $B$의 근들이라고 하자. 다음과
같이 쓰자.
$$
\beta_1:=1, \quad
\beta_k:=\prod_{n=1}^{k-1}\frac{\lambda_n-z}{1-\overline{\lambda}_n z}\cdot
\frac{|\lambda_n|}{\lambda_n}\qquad (k\geq 2);
$$
$$
\delta_j\label{deltaj}:=\frac{d_j}{1-\overline{\lambda}_j z}\beta_j \qquad (j \geq
1),
$$
여기에서, $d_j\label{dj}:=(1-|\lambda_j|^2)^{\frac{1}{2}}$.
$\mu_B\label{mub}$를 $\mathbb N$ 상의 측도로서
$\mu_B(\{n\}):=\frac{1}{2}d_n^2, (n \in \mathbb N)$라고 하자. 이제
함수  $V_B\label{VB}: L^2(\mu_B)\to \mathcal H(B)$ 가 다음과 같이
정의된다고 하자.
\begin{equation}\label{3.2}
V_B(c):=\frac{1}{\sqrt{2}} \sum_{n \geq 1} c(n)d_n \delta_n, \quad c
\equiv \{c(n)\}_{n \geq 1}.
\end{equation}
그러면 $V_B$는 유니터리이고 $U_B$는 다음의 함수로 보내진다.
\begin{equation}\label{3.3}
V_B^*U_B V_B=(I-J_B)M_B,
\end{equation}
여기에서, $(M_B\label{MB} c)(n):=\lambda_n c(n)$ ($n\in \mathbb
N$)은 곱 작용소이고
$$
(J_B\label{JB} c)(n):=\sum_{k=1}^{n-1}c(k)|\lambda_k|^{-2} \cdot
\frac{\beta_n(0)}{\beta_k(0)}d_k d_n \ \ \hbox{$(n \in \mathbb N$)}
$$
는 하삼각 Hilbert-Schmidt 작용소이다. \

\newpage \leftline{\large{\it 188}\ \ \
\small 학문 연구의 동향과 쟁점 -- 수학}

\vspace{0.8cm}

\noindent {\bf 경우 2} : $s$를 연속표현측도  $\mu\equiv
\mu_{s}\label{mus}$를 가진 특이내적함수(singular inner function)라고
하자. 이제, $\mu_{\lambda}$을 $\mu$에서 호 $\{\zeta: \zeta \in
\mathbb T, \ 0< \text{arg}\zeta \leq \text{arg}\lambda\}$ 위로의
사영이라 하고 다음과 같이 두자.
$$
s_{\lambda}(\zeta)\label{slambdazeta}:=\text{exp}\Bigl(-\int_{\mathbb
T}\frac{t+\zeta}{t-\zeta}d\mu_{\lambda}(t) \Bigr) \ \ \hbox{($\zeta
\in \mathbb D$)}.
$$
이제 함수 $V_s\label{Vs}: L^2(\mu)\to \mathcal H (s)$가 다음과 같이 정의된다고 하자.
\begin{equation}\label{3.4}
(V_s c)(\zeta)=\sqrt{2}\int_{\mathbb
T}c(\lambda)s_{\lambda}(\zeta)\frac{\lambda
d\mu(\lambda)}{\lambda-\zeta}\quad\hbox{($\zeta \in \mathbb D$)}
\end{equation}
그러면 $V_s$는 유니터리이고 $U_s$는 다음의 함수로 보내진다.
\begin{equation}\label{3.5}
V_s^*U_s V_s=(I-J_s)M_s,
\end{equation}
여기에서, $(M_s c)(\lambda):=\lambda c(\lambda)$ ($\lambda\in\mathbb
T$) 는 곱 작용소이고
$$
(J_s\label{Js} c)(\lambda)=2 \int _{\mathbb T} e^{\mu(t)-\mu(\lambda)}
c(t)d_{\mu_{\lambda}}(t)\ \ \hbox{($\lambda \in \mathbb T$)}
$$
는 하삼각 Hilbert-Schmidt 작용소이다. \

\vspace{.3 cm}

\noindent {\bf 경우 3} : $\Delta$를 순수한 점표현측도 $\mu\equiv
\mu_{\Delta}\label{mudelta}$를 가진 특이내적함수(singular inner
function)라고 하자. 그리고 집합 $\{t\in \mathbb T: \mu(\{t\})>0
\}$을 수열 $\{t_k\}_{k \in \mathbb N}$로 배열하고 다음과  같이 쓰자:
$\mu_k:=\mu(\{t_k\}), \ k \geq 1$. \  더구나, $\mu_{\Delta}$를
$\mathbb R_{+}=[0, \infty)$ 상의 측도로서
$d\mu_{\Delta}(\lambda)=\mu_{[\lambda]+1}d \lambda$라고 하자.이제
단위원판 $\mathbb D$ 위의 함수
$\Delta_{\lambda}\label{deltalambda}$를 다음과 같이 정의하자.
$$
\Delta_{\lambda}(\zeta):=\text{exp}\Biggl\{-\sum_{k=1}^{[\lambda]}\mu_k
\frac{t_k+\zeta}{t_k-\zeta}-(\lambda-[\lambda])\mu_{[\lambda]+1}
\frac{t_{[\lambda]+1}+\zeta}{t_{[\lambda]+1}-\zeta}\Biggr \},
$$
여기에서, $[\lambda]$는 $\lambda$ ($\lambda \in \mathbb R_{+}$)의
정수 부분이고 $\Delta_0:=1$로 약속한다. \ 그리고 함수
$V_{\Delta}\label{Vdelta}: L^2(\mu_{\Delta})\to \mathcal H
(\Delta)$가 다음과 같이 정의된다고 하자.
\begin{equation}\label{3.6}
(V_{\Delta}c)(\zeta):=\sqrt{2} \int_{\mathbb
R_{+}}c(\lambda)\Delta_{\lambda}(\zeta)(1-\overline{t}_{[\lambda]+1}
\zeta)^{-1}d\mu_{\Delta}(\lambda) \ \hbox{($\zeta \in \mathbb D$)}
\end{equation}
그러면 $V_{\Delta}$는 유니터리이고 $U_\Delta$는 다음의 함수로 보내진다.
\begin{equation}\label{3.7}
V_{\Delta}^*U_\Delta V_{\Delta}=(I-J_{\Delta})M_{\Delta},
\end{equation}

\newpage\rightline{\small 이우영:
토에플리츠 작용소에 대한 브리지 이론\ \ \ \large\it 189}

\vspace{.8 cm}

\noindent 여기에서, $(M_{\Delta}c)(\lambda):= t_{[\lambda]+1}
c(\lambda)$, ($\lambda\in \mathbb R_+$)은 곱 작용소이고
$$
(J_{\Delta}\label{Jdelta} c)(\lambda) := 2 \int _{0}^{\lambda}c(t)
\frac{\Delta_{\lambda}(0)}{\Delta_{t}(0)} d\mu_{\Delta}(t) \ \
\hbox{($\lambda \in \mathbb R_+$)}
$$
는 하삼각 Hilbert-Schmidt 작용소이다. \

\medskip

이제 위의 세 경우를 종합하면 다음을 얻는다.

\medskip

{\bf Triangularization theorem.} \cite [p.123] {Ni} {\sl Let
$\theta$ be an inner function with the canonical factorization
$\theta=B\cdot s \cdot \Delta$, where $B$ is a Blaschke product,
and $s$ and $\Delta$ are singular functions with representing
measures $\mu_{s}$ and $\mu_{\Delta}$ respectively, with $\mu_{s}$
continuous and $\mu_{\Delta}$ a pure point measure. \  Then the
map $V:\,L^2(\mu_B)\times L^2(\mu_s) \times L^2(\mu_{\Delta})\to
\mathcal H(\theta)$ defined by
\begin{equation}\label{3.8}
V:=\begin{bmatrix}V_B&0&0\\0&BV_s&0\\0&0&BsV_{\Delta}\end{bmatrix}
\end{equation}
is unitary, where $V_B, \mu_B, V_S, \mu_S, V_\Delta, \mu_\Delta$ are
defined in (\ref{3.2}) - (\ref{3.7}) and $U_\theta$ is mapped onto
the operator
$$
M:= V^*U_\theta V =
\begin{bmatrix}M_B&0&0\\0&M_{s}&0\\0&0&M_{\Delta}\end{bmatrix}+J,
$$
where $M_B, M_S, M_{\Delta}$ are defined in (\ref{3.3}), (\ref{3.5})
and (\ref{3.7}) and
$$
J:=-\begin{bmatrix}J_BM_B&0&0\\0&J_sM_s&0\\0&0&J_{\Delta}M_\Delta\end{bmatrix}+A
$$
is a lower-triangular Hilbert-Schmidt operator, with $A^3=0$,
$\text{rank}\,A\leq 3$.}

\

위의 정리를 이용하면 유계형 심벌을 가진 Toeplitz 작용소의 아정규성을
판별할 수 있는 판정법을 얻을 수 있다.

\newpage\leftline{\large{\it 190}\ \ \ \small 학문 연구의 동향과 쟁점 --
수학} \vspace{0.8cm}

{\bf Theorem 3.4.1.} {\rm \cite{CHL1}} {\sl Let $\varphi\equiv
\varphi_-^*+\varphi_+\in \bold L^\infty$ be such that $\varphi$
and $\overline\varphi$ are of bounded type of the form
$$
\varphi_+=\theta_1\theta_0\overline a\quad\hbox{and}\quad
\varphi_-=\theta_1\overline b,
$$
where $\theta_1$ and $\theta_0$ are inner functions and $a,b\in
H^2$. \  If $k\in\mathcal{C}(\varphi)$ then
$$
T_\varphi\ \hbox{is hyponormal}\ \Longleftrightarrow\ k(M)\ \hbox{is
contractive},
$$
where $M$ is defined as follows:
\begin{equation}\label{3.10}
\begin{cases}
V: L\equiv L^2(\mu_B) \times L^2(\mu_s)
         \times L^2(\mu_{\Delta})\to \mathcal H(\theta_1 \theta_0)\ \
            \hbox{\rm is unitary as in (\ref{3.8})};\\
M:= V^*U_{\theta_1\theta_0}V;\\
\mathcal L :=L \otimes \mathbb C^n\\
\mathcal V := V \otimes I_n.
\end{cases}
\end{equation}}
%\end{thm}
위에서 $k(M)$은 $H^\infty$-함수계산을 따른다.


%%%%%%%%%%%%%%%%%%%%%%%%%%%%%%%%%%%%%%%%%%%%%%%%%%%%%%%%%%%%%%%%%%%%%%%%%%%%%%%%%%%%%%%%%%%%%%%%%%%%%%%%%%%%%%%%%%%

\vspace{.8 cm} {\large\bf 4. Toeplitz 작용소의 부분정규성}

\vspace{.5 cm}


이 절은 다음의 문제에 관한 것이다: {\it 어떤 Toeplitz 작용소가
부분정규인가?} Toeplitz 작용소 $T_\varphi$가 해석적(analytic)이라
함은 심벌 $\varphi$가 $\bold H^\infty$에 속할 때를 말한다, 즉,
$\varphi$가 $\mathbb{D}$ 상의 유계 해석함수일 때이다. 이는
부분정규임을 쉽게 알 수 있다. 실제로,
$$
T_\varphi h=P(\varphi h)=\varphi h=M_\varphi h\quad \hbox{($h\in \bold H^2$)},
$$
여기에서, $M_\varphi$는   $\bold L^2$ 상에서 $\varphi$를 곱하는 곱
작용소(multiplication operator)로서 정규작용소이다. 1970년에
P.\,R.\ Halmos는 {\it Ten Problems in Hilbert Space} (cf.
\cite{Ha1}, \cite{Ha2})라는 강의록에서 작용소 학계에  Hilbert 공간
상의 문제 열 개를 제시하였는데, 그 중 다섯번째 문제 (소위 Halmos'
Problem 5로 불리움)가 Toeplitz 작용소의 부분정규성에 대한
것이었다:

\vspace{.2 cm}\label{4.1.1}\ \ \ \ \ {\it 모든 부분정규 Toeplitz
작용소는 정규이거나 해석적일까\,?}

\vspace{.2 cm}  이 문제는 정규 Toeplitz 작용소와 해석적 Toeplitz
작용소가 잘 이해되고, 또 모두 부분정규이므로 매우 자연스러운 문제라
하겠다.

\vspace{.2 cm}

%%%%%%%%%%%%%%%%%%%%%%%%%%%%%%%%%%%%%%%%%%%%%%%%%%%%%%%%%%%%%%%%%%%%%%%%%%%%%%%%%%%%%%%%%%%
\newpage\rightline{\small 이우영: 토에플리츠 작용소에 대한 브리지 이론\
\ \ \large\it 191}


\vspace{.8 cm}

{\bf 4.1. Halmos' Problem 5}

\vspace{0.3cm} 이제 Halmos' Problem 5와 관련된 연구에 대하여
살펴본다. 1976년에, M. Abrahamse \cite{Ab}가 Halmos' Problem 5에
대한 매우 일반적인 충분조건을 주었는데, 이 오래 된 정리가 약간
놀랍게도 오늘날까지 알려진 가장 좋은 충분조건이다.

\vspace{0.3cm} {\bf Theorem 4.1.1.} Abrahamse's Theorem) {\rm
\cite{Ab}} {\sl If
\begin{itemize}
\item[{\rm (i)}] $T_{\varphi}$ is hyponormal;
\item[{\rm (ii)}] $\varphi$ or $\overline\varphi$ is of bounded type;
\item[{\rm (iii)}] ${\rm ker}[T_{\varphi}^*,T_{\varphi}]$ is invariant for $T_{\varphi}$,
\end{itemize}
then $T_{\varphi}$ is normal or analytic.}
%\end{thm}

\vspace{0.3cm} 한편, 만일 $S$가 $\mathcal H$ 상의 부분정규이고
$N:=\mbox{mne}\,(S)$이면,
$$
{\rm ker}[S^*,S]=\{ f:\ <f, [S^*,S]f>=0\}=\{ f:\ ||S^*f||=||Sf||\}=\{ f:\ N^{*}f\in\mathcal{H}\}.
$$
그러므로 $S({\rm ker}[S^*,S])\subseteq {\rm ker}[S^*,S]$. 따라서,
Theorem 4.1.1에 의헤서 다음을 얻는다.

\vspace{0.3cm} {\bf Corollary 4.1.2.}\ {\sl If $T_{\varphi}$ is
subnormal and if $\varphi$ or $\overline\varphi$ is of bounded type,
then $T_{\varphi}$ is normal or analytic.} \vspace{0.3cm}

위의 Abrahamse's Theorem은 최근에 저자와 공저자들이 행렬함수심벌로
확장하는 데 성공하였다 (\cite{CHL2}). 다음은 M. Abrahamse
\cite{Ab}이  밝혀낸 유계형 함수의 특성화이다.

\vspace{0.3cm} {\bf Corollary 4.1.3} {\sl  A function $\varphi$ is
of bounded type if and only if ${\rm ker}H_{\varphi}\neq\{0\}$. }

\vspace{0.3cm} Toeplitz 작용소의 부분정규성을 잘 연구할 수 있는 방법
중의 하나가 기존의 잘 알려진 부분정규 작용소와 유니터리 동치인
경우를 찾아내는 것인데, 가중이동 작용소가 비교적 부분정규를 알기
쉬운 경우이므로 어떤 가중이동 작용소가 Toeplitz 작용소와 유니터리
동치인지 물어볼 수 있을 것이다. 다음은 그 첫번째 답이다.


\newpage\leftline{\large{\it 192}\ \ \
\small 학문 연구의 동향과 쟁점 -- 수학} \vspace{0.8cm}

{\bf Proposition 4.1.4.} {\sl If $A$ is a weighted shift with
weights $a_0, a_1, a_2, \cdots $ such that
$$
0\le a_0 \le a_1 \le \cdots < a_N = a_{N+1}= \cdots = 1,
$$
then $A$ is not unitarily equivalent to any Toeplitz operator. }

\vspace{.2 cm} 이제 유명한 Bergman 이동작용소(shift) (가중수열이
$\sqrt{{n+1}\over{n+2}}$로 주어지는 이동 작용소)는 부분정규이므로
우리는 약간의 기대감을 가지고 다음을 물을 수 있을 것이다.
\begin{equation}\label{4.1.2}
\mbox{Bergman 이동 작용소는 어떤 Toeplitz 작용소와 유니터리
동치인가?}
\end{equation}

만일 위의 질문 (\ref{4.1.2})의 대답이 긍정적이라면, Halmos' Problem
5의 대답은 부정적이 될 것이다. 이를 알기 위해서 Bergman 이동작용소
$S$가 Toeplitz 작용소 $T_\varphi$와 유니터리 동치라고 가정하자.
그러면,
$$
\frak{R}(\varphi)\subseteq\ \sigma
_e(T_\varphi )=\sigma_e(S)=\mbox{the unit circle}\ \mathbb{T}.
$$
따라서, $\varphi$는 그 절대값이 1인 함수(unimodular)이다. 그런데
$S$가 등장함수가 아니므로 $\varphi$는 내적함수가 아니다. 그러므로,
$T_\varphi$는 해석적 Toeplitz 작용소가 아니다. 1983년에 S. Sun
\cite{Sun}이 문제 (\ref{4.1.2})에 부정적인 해답을 주었다. \vspace{.2
cm}

{\bf Theorem 4.1.5.} (Sun's Theorem) {\rm \cite{Sun}} {\sl Let $T$
be a weighted shift with a strictly increasing weight sequence
$\{a_{n}\}_{n=0}^{\infty}$. If $T\cong T_{\varphi}$ then
$$
a_{n}=\sqrt{1-{\alpha}^{2n+2}} \,||T_{\varphi}||\quad (0<\alpha <1).
$$}
\indent 위의 Theorem 4.1.5에 의하여 다음을 얻는데 이는 문제
(\ref{4.1.2})의 답을 준다. \vspace{.2 cm}

{\bf Corollary 4.1.6} {\sl The Bergman shift is not unitarily
equivalent to any Toeplitz operator.} \vspace{.2 cm}
%\end{cor}

\vspace{.2 cm}

마침내, 1984년에 Cowen과 Long \cite{CoL}이 Halmos' Problem 5를
해결하였는데, 우선 이를 위해 다음이 필요하다.

\vspace{.2 cm} {\bf Lemma 4.1.7.} {\sl The weighted shift $T\equiv
W_{\alpha}$ with weights $\alpha_n\equiv
(1-\alpha^{2n+2})^{\frac{1}{2}}\ (0<\alpha <1)$ is subnormal.}
%\end{lem}

\vspace{.2 cm} 더구나 Toeplitz 작용소가 가중이동 작용소와 유니터리
동치이면 부분정규가 됨이 알려졌다.



\newpage\rightline{\small 이우영:
토에플리츠 작용소에 대한 브리지 이론\ \ \ \large\it 193}

\vspace{.8 cm}

{\bf Corollary 4.1.8.} {\sl If $T_{\varphi} \cong$ a weighted shift,
then $T_\varphi$ is subnormal.} \vspace{.2 cm}

만일  $T_\varphi$가 가중이동 작용소와 동치이면,
$\varphi$는 어떤 형태일까?
\cite[Theorem 3.7]{Sun}의 증명을 분석해보면,
다음을 알 수 있다.
$$
\psi=\varphi -\alpha\overline{\varphi}\in {\bold H}^{\infty}.
$$
그러나,
\begin{align*}
T_\psi =T_\varphi -\alpha T_{\varphi}^* &=
{\begin{pmatrix}
0 &-\alpha a_0\\
a_0&0&-\alpha a_1\\
&a_1&0&-\alpha a_2\\
&&a_2&0&\ddots\\
&&&\ddots&\ddots
 \end{pmatrix}}\\
&=\begin{pmatrix}
0 &-\alpha\\
1&0&-\alpha\\
&1&0&-\alpha\\
&&1&0&\ddots\\
 &&&\ddots&\ddots
\end{pmatrix} + K\quad \mbox{($K$ compact)}\\
&\cong T_{z-\alpha\overline{z}}+K.
\end{align*}
따라서,
$\mbox{ran}\,(\psi)=\sigma_{e}(T_{\psi})
=\sigma_{e}(T_{z-\alpha\overline{z}})={\rm ran}(z-\alpha\overline{z})$.
그러므로, $\psi$는 단위원판을 꼭지점이  $\pm i(1+\alpha)$이고,  $\pm(1-\alpha)$을
지나는 타원으로 보내는 등각사상이다.
한편, $\psi=\varphi -\alpha\overline{\varphi}$.
그래서,
$\alpha\overline{\psi}=\alpha\overline{\varphi} -\alpha^2\varphi$
이고, 다음이 성립한다.
$$
\varphi=\frac{1}{1-\alpha^2}(\psi +\alpha\overline{\psi}).
$$

이제 다음을 얻게 된다.

\vspace{.2 cm}{\bf Theorem 4.1.9.} (Cowen and Long Theorem) {\rm
\cite{CoL}} {\sl For $0<\alpha <1$, let $\psi$ be a conformal map of
$\mathbb{D}$ onto the interior of the ellipse with vertices $\pm
i(1-\alpha)^{-1}$ and passing through $\pm (1+\alpha)^{-1}$. Then
$T_{\psi +\alpha\overline{\psi}}$ is a subnormal weighted shift that
is neither analytic nor normal.}
%\end{thm}
\vspace{.2 cm}

Abrahamse's Theorem 과 Theorem 4.1.9\,로부터 다음을 얻는다.

\vspace{.2 cm}

\newpage\leftline{\large{\it 194}\ \ \ \small 학문 연구의 동향과 쟁점 --
수학}\vspace{0.8cm}

{\bf Corollary 4.1.10.} {\sl If $\varphi=\psi
+\alpha\overline{\psi}$ is as in Theorem 4.1.9, then neither
$\varphi$ nor $\overline{\varphi}$ is bounded type.} \vspace{.2 cm}

이제 Halmos' Problem 5가 해결되었다고는 하나, 본질적으로 Toeplitz
작용소의 부분정규성은 여전히 암흑 속에 갇혀 있다고 해도 과언이
아니다. Abrahamse's Theorem 정도가 가장 좋은 대답이므로 앞으로 계속
Toeplitz 작용소의 부분정규성을 탐험해 갈 필요가 있다. 지난 십 수년간
저자와 공동 연구자들은 이 문제를 끊임없이 고찰해 왔으며 부분적으로
흥미로운 결과들을 만들어내기는 하였지만, 여전히 그 여정은
미미하였으며 앞으로도 얼마나 더 긴 여정을 거쳐야 할 지 모른다. 물론
지나온 여정 중에 종종 신기루 같은 것을 보긴 하였지만 가까이 가보면
그냥 신기루였을 뿐, 여전히 그 단서를 찾지 못하고 있다. 다음 절에서는
그 중 한 가지를 살펴볼 것이다.

%%%%%%%%%%%%%%%%%%%%%%%%%%%%%%%%%%%%%%%%%%%%%%%%%%%%%%%%%%%%%%%%%%%%%%%%%%%%%%%%%%
\vspace{.3 cm} {\bf 4.2. Toeplitz 작용소의 $k$-아정규성과 부분정규성
사이의 틈}

\vspace{.3 cm} Toeplitz 작용소의 $k$-아정규성과 부분정규성 사이의 틈
을 이해하는 것은 매우 흥미로워 보인다. 이 문제에 대한 첫번째
후보로서 우리는 다음을 생각할 수 있다 (\cite{CuL1}): 모든 2-아정규
Toeplitz 작용소는 부분정규인가?

\medskip

이 문제에 대하여 \cite{CuL1}에서 다음이 보여졌다:

\vspace{.2 cm}{\bf Theorem 4.2.1.} {\rm \cite{CuL1}}  {\sl Every
trigonometric Toeplitz operator whose square is hyponormal must be
normal or analytic. Hence, in particular, every 2-hyponormal
trigonometric Toeplitz operator is subnormal.}
%\end{thm}

\vspace{.2 cm} \cite{Cu1}에서 가중이동 작용소에 대한 아정규성과
2-아정규성 사이에 틈이 존재함을 보였다. 위의 Theorem 4.2.1\,은 역시
Toeplitz 작용소에 대해서도  아정규성과 2-아정규성 사이에 틈이
존재함을 보여주고 있다. 예를 들어, 만일
$$
\varphi(z)=\sum_{n=-m}^N a_n z^n\quad (m<N)
$$
에 대하여 $T_\varphi$가 아정규이면, Theorem 4.2.1에 의해
$T_\varphi$는 결코 2-아정규가 아니다 (그 이유: $T_\varphi$가 정규도
해석적도 아님). 비교하여 다음을 상기할 필요가 있다: 만일,
$\varphi(z)=\sum_{n=-m}^N a_n z^n$에 대하여 $T_\varphi$가 정규이면,
$m=N$ (cf. \cite{FL1}).

\vspace{.2 cm} 우리는 Abrahamse's Theorem을 약간 확장할 수 있다.
이를 위하여 다음 관찰이 필요하다.


\newpage\rightline{\small 이우영: 토에플리츠 작용소에 대한 브리지 이론\
\ \ \large\it 195}

\vspace{.8 cm}

{\bf Proposition 4.2.2.} {\rm \cite{CuL2}} {\sl If
$T\in\mathcal{B(H)}$ is 2-hyponormal then
\begin{equation}\label{4.2.1}
T\bigl(\text{ker}\,[T^*,T]\bigr)\ \subseteq \text{ker}\,[T^*,T].
\end{equation}}

그러면  즉시 다음을 얻는다.

\vspace{.2 cm}{\bf Corollary 4.2.3.} {\sl If $T_\varphi$ is
$2$-hyponormal and if $\varphi$ or $\bar\varphi$ is of bounded type
then $T_\varphi$ is normal or analytic, so that $T_\varphi$ is
subnormal.}  \vspace{.2 cm}

우리는 역시 다음을 증명할 수 있다.

\vspace{.2 cm}{\bf Corollary 4.2.4.} {\sl If $T_\varphi$ is a
$2$-hyponormal operator such that $\mathcal{E}(\varphi)$ contains at
least two elements then $T_\varphi$ is normal or analytic, so that
$T_\varphi$ is subnormal.}
%\end{cor}
\vspace{.2 cm}

Corollary 4.2.3\,과 Corollary 4.2.4\,로부터, 만일 $T_\varphi$가
$2$-아정규이지만 부분정규가 아니면, $\varphi$는 유계형 함수가 아니고
$\mathcal{E}(\varphi)$는 꼭 하나의 원소로 이루어짐을 알 수 있다.
우리는 나아가 Toeplitz 작용소에 대하여 임의의 $k$-아정규성과
부분정규성 사이에 틈이 존재함을 보였다.

\vspace{.2 cm}{\bf Theorem 4.2.5.} {\rm \cite{CLL}} {\sl Let
$0<\alpha<1$ and let $\psi$ be the conformal map of the unit disk
onto the interior of the ellipse with vertices $\pm(1+\alpha)i$ and
passing through $\pm(1-\alpha)$. Let $\varphi=\psi+\lambda\bar\psi$
and let $T_\varphi$ be the corresponding Toeplitz operator on $H^2$.
Then $T_\varphi$ is $k$-hyponormal if and only if $\lambda$ is in
the circle $\left|z-\frac{\alpha(1-\alpha^{2j})}
{1-\alpha^{2j+2}}\right|
=\frac{\alpha^j(1-\alpha^2)}{1-\alpha^{2j+2}}$ for $j=0,1,\cdots
,k-2 $ or in the closed disk
$\left|z-\frac{\alpha(1-\alpha^{2(k-1)})} {1-\alpha^{2k}}\right| \le
\frac{\alpha^{k-1}(1-\alpha^2)}{1-\alpha^{2k}}$. }

%%%%%%%%%%%%%%%%%%%%%%%%%%%%%%%%%%%%%%%%%%%%%%%%%%%%%%%%%%%%%
\vspace{.3 cm} {\bf 4.3 미해결 문제}

\vspace{.3 cm} Cowen과 Long의 아이디어 \cite{CoL}는 Toeplitz
작용소와 부분정규성 사이의 관계에 대한 어떤 일반적인 특징도 주지
못했다. 결국 지금까지 누구도 Toeplitz 작용소의 부분정규성에 대하여
심벌에 의한 특성화를 만들지 못했다. 따라서 다음 문제는 여전히
매력적이고 흥미롭다:

\vspace{.2 cm}

{\bf Problem A.}  {\sl Characterize the subnormality of
$T_\varphi$ in terms of $\varphi$\,? }

\vspace{.2 cm} 이제 Toeplitz 작용소에 대한 부분정규성과 관련된 보다
구체적인 미해결 문제들을 제시한다.



\newpage \leftline{\large{\it 196}\ \ \ \small 학문 연구의
동향과 쟁점 -- 수학} \vspace{.8 cm}

{\bf Problem B.}  {\sl For which $f\in{\bold H}^{\infty}$, is there
$\lambda\ (0<\lambda<1)$ with $T_{f +\lambda\overline{f}}$
subnormal\,? }

\vspace{.2 cm} {\bf Problem C.} {\sl Suppose $\psi$ is as in Theorem
4.1.9 (i.e., the ellipse map). Are there $g\in{\bold H}^{\infty}$,
$g\neq\lambda\psi +c$, such that $T_{\psi+\overline{g}}$ is
subnormal\,? }

\vspace{.2 cm}

{\bf Problem D.} {\sl More generally, if $\psi
\in\bold{H}^\infty$, define
$$
\mathcal{S}(\psi):=\{g\in\bold{H}^\infty:\ T_{\psi+\overline g}\ \mbox{is subnormal}\ \}.
$$
Describe $\mathcal{S}(\psi)$. For example, for which $\psi\in \bold{H}^\infty$, is it balanced?, or is it convex?, or is it weakly closed?
What is ${\rm ext}\,\mathcal{S}(\psi)$\,?
For which $\psi\in \bold{H}^\infty$, is it strictly convex ?, i.e.,
$\partial\mathcal{S}(\psi)\subset{\rm ext}\,\mathcal{S}(\psi)$\,?
}

\vspace{.2 cm} C.\ Cowen \cite{Cow3}은 어떤 논증도 없이 다음의
흥미로운 언급을 하였다: ``{\it If $T_\varphi$ is subnormal then
$\mathcal{E}(\varphi)=\{\lambda\}$ with $|\lambda|<1$".} 그러나
우리는 이것이 참인지 아닌지 보일 수가 없었다. 물론 만일
$T_\varphi$가 정규이면, $\mathcal{E}(\varphi)=\{e^{i\theta}\}$ 이다.
따라서 다음은 자연스러운 질문이다.

\vspace{.2 cm} {\bf Problem E.} {\sl Is the above Cowen's remark
true\,? That is, if $T_\varphi$ is subnormal, does it follow that
$\mathcal{E}(\varphi)=\{\lambda\}$ with $|\lambda|<1$\,? }

\vspace{.2 cm}

만일 Problem E의 대답이 긍정적이면, 즉,
위의 Cowen의 언급이 참이면, $\varphi=\overline g+f$에 대하여,
$$
\mbox{$T_\varphi$ is subnormal}\ \Longrightarrow \ \overline
g-\lambda \overline f\in \bold{H}^2\ \ \mbox{with}\ |\lambda|<1\
\Longrightarrow\ g=\overline\lambda f+c.
$$
($c$는 상수). 이는 Problem C의 해답이 부정적임을 보여준다.

\vspace{.2 cm}

한편, Corollary 4.9.3\,으로부터 $T_\varphi$가 $2$-아정규이고
$\varphi$ 또는 $\bar\varphi$가 유계형식 함수이면, $T_\varphi$는
비자명 불변 부분공간(nontrivial invariant subspace)을 가진다. 그래서
다음 문제가 자연스럽게 떠오른다: \vspace{.2 cm}

 {\bf Problem F.} {\sl Does every $2$-hyponormal
Toeplitz operator have a nontrivial invariant subspace\,? More
generally, does every $2$-hyponormal operator have a nontrivial
invariant subspace\,? }

\vspace{.2 cm}


\newpage\rightline{\small 이우영: 토에플리츠 작용소에 대한 브리지 이론\ \ \
\large\it 197} \vspace{0.8cm}

 만일, $T$가 아정규이고 $R(\sigma(T))\ne
C(\sigma(T))$이면 $T$가 비자명 불변부분공간을 가진다는 사실이 널리
알려져 있다(\cite{Bro}). 그러나 $T$가 아정규이고,
$R(\sigma(T))=C(\sigma(T))$이면 (즉, thin spectrum을 가지면) $T$가
비자명 불변 부분공간을 가지는지는 여전히 미해결 문제이다. 한편,
$T\in\mathcal{B(H)}$가 von Neumann 작용소라고 함은 $\sigma(T)$
밖에서 pole을 가지는 유리함수 $f$에 대하여 $f(T)$가 normaloid (i.e.,
norm = spectral radius)임을 뜻한다. B. Prunaru \cite{Pru}는 모든
다항 아정규작용소는 비자명 불변 부분공간을 가짐을 보였다. 역시
\cite{Ag}에서 von Neumann 작용소도 비자명 불변부분공간을 가짐을
보였다. 다음은 Problem F의 부분 문제이다.

\vspace{.2 cm}


{\bf Problem G.} {\sl Is every 2-hyponormal operator with thin
spectrum a von-Neumann operator\,? }

\vspace{.2 cm}

비록, 부분정규가 아닌 다항정규 가중이동작용소의 존재가 알려져 있기는
하지만 (cf. \cite{CP1}, \cite{CP2}), 토에플리츠 작용소에 대하여
``다항정규 $\Rightarrow$ 부분정규" 인지 아닌지는 알려져 있지 않다.
그래서 다음 문제도 자연스럽게 떠오른다.

\vspace{.2 cm}

{\bf Problem H.} {\sl Does there exist a Toeplitz operator which is
polynomially hyponormal but not subnormal\,? }

\vspace{.2 cm}

한편, \cite{CuL2}에서 모든 순수(pure -- it has no normal direct
summand) $2$-아정규 작용소 $T$의 자기-가환자의 계수(rank)가 1이면
$T$는 이동작용소의 선형함수임이 밝혀졌다. McCarthy 와 Yang
\cite{McCYa}은 유한 계수를 가지는 자기-가환자의 모든 유리
순환(rationally cyclic) 부분정규 작용소를 특성화하였다. 그러나
여전히 유한 계수의 자기-가환자를 가지는 모든 순수 부분정규 작용소를
특성화하지는 못했다. 그래서 다음의 질문이 즉시 떠오른다.

\vspace{.2 cm} {\bf Problem I.} {\sl If $T_\varphi$ is a
$2$-hyponormal Toeplitz operator with nonzero finite rank
self-commutator, does it follow that $T_\varphi$ is analytic\,? }

\vspace{.2 cm}

위의 미해결 문제들은 모두 쉽지 않은 문제들로 보여진다. 그러나
Toeplitz 작용소의 성질을 충분히 이해하기 위해서는 위 문제의 답을
찾는 것은 필수불가결한 일이다. 더구나 작용소의 브리지 이론의 완성을
위해서는 Toeplitz 작용소의 부분정규성 확립에 대한 노력을 더욱 경주할
필요가 있다. 미해결 문제에 대한 도전이야말로 수학 최고의 미덕이 아닐
수 없다

\newpage
\leftline{\large{\it 198}\ \ \ \small 학문 연구의 동향과 쟁점 --
수학}

\vspace{0.8cm}

참으로, Morris Kline\,이 Bertrand Russel의 한 작품에 헌사한 서문에서
처럼 ``미해결 문제에 대한 도전은 인류를 깨어 있게 만들고 궁극적으로
인류를 정신적 무기력으로부터 보호할 것이다."


%%%%%%%%%%%%%%%%%%%%%%%%%%%%%%%%%%%%%%%%%%%%%%%%%%%%%%%%%%%%%%%%%%%%%%%%%%%%%%%%%%%%%%%%

\begin{thebibliography}{99}\footnotesize


\bibitem[1]{Ab} M. B. Abrahamse,
{\it Subnormal Toeplitz operators and functions of bounded type},
Duke Math. J. {\bf 43} (1976), 597--604.

\bibitem[2]{Ag} J. Agler, {\it An invariant subspace problem}, J. Funct. Anal.
{\bf 38} (1980), 315--323.

\bibitem[3]{Bro} S. Brown, {\it Hyponormal operators with thick spectra have invariant subspaces},
Ann. of Math. {\bf 125} (1987), 93--103.

\bibitem[4]{BH} A. Brown and P. R. Halmos, {\it Algebraic properties of Toeplitz operators},
J. Reine Angew. Math. {\bf 213} (1963-1964), 89--102.


\bibitem[5]{Con2} J. B. Conway, The Theory of Subnormal Operators,
Math. Surveys and Monographs, vol. 36, Amer. Math. Soc., Providence, 1991.


\bibitem[6]{Cow2} C. Cowen, {\it Hyponormal and subnormal Toeplitz operators},
Surveys of Some Recent Results in Operator Theory, I (J. B. Conway
and B. B. Morrel, eds.) Pitman Research Notes in Mathematics,
Longman, Vol.{\bf 171} (1988), 155--167.

\bibitem[7]{Cow3} C. Cowen, {\it Hyponormality of Toeplitz operators}, Proc. Amer. Math. Soc.
{\bf 103} (1988), 809--812.

\bibitem[8]{CoL} C. C. Cowen and J. J. Long, {\it Some subnormal Toeplitz operators},
J. Reine Angew. Math. {\bf 351} (1984), 216--220.

\bibitem[9]{Cu1} R. E. Curto, {\it Quadratically hyponormal weighted shifts},
Integral Equations Operator Theory, {\bf 13} (1990), 49--66.

\bibitem[10]{CHL1}R. E. Curto, I. S. Hwang and W. Y. Lee,
{\it Hyponormality and subnormality of block Toeplitz operators},
Adv. Math. {\bf 230} (2012), 2094--2151.


\bibitem[11]{CHL2}  R. E. Curto, I. S. Hwang and W. Y. Lee,
{\it Which subnormal Toeplitz operators are either normal or
analytic\,?}, J. Funct. Anal. {\bf 263}(8) (2012), 2333--2354.


\newpage  \rightline{\small 이우영: 토에플리츠 작용소에 대한 브리지
이론\ \ \ \large\it 199}

\vspace{.4 cm}

\bibitem[12]{CLL} R. E. Curto, S. H. Lee and W. Y. Lee, {\it Subnormality and 2-hyponormality for Toeplitz operators},
Integral Equations Operator Theory, {\bf 44} (2002), 138--148.


\bibitem[13]{CuL1} R. E. Curto and W. Y. Lee, {\it Joint hyponormality of Toeplitz pairs},
Memoirs Amer. Math. Soc. No.{\bf 712}, Amer. Math. Soc., Providence,
2001.

\bibitem[14]{CuL2} R. E. Curto and W. Y. Lee, {\it Towards a model theory for $2$--hyponormal operators},
Integral Equations Operator Theory, {\bf 44} (2002), 290--315.

\bibitem[15]{CP1} R. E. Curto and M. Putinar, {\it Existence of non-subnormal polynomially hyponormal operators},
Bull. Amer. Math. Soc. (N.S.), {\bf 25} (1991), 373--378.

\bibitem[16]{CP2} R. E. Curto and M. Putinar, {\it Nearly subnormal operators and moment problems},
J. Funct. Anal. {\bf 115} (1993), 480--497.

\bibitem[17]{FL1} D. R. Farenick and W. Y. Lee, {\it Hyponormality and spectra of Toeplitz operators},
Trans. Amer. Math. Soc., {\bf 348} (1996), 4153--4174.

\bibitem[18]{FF} C. Foia\c s and A. Frazho, {\it The commutant lifting approach to
interpolation problems}, Operator Theory: Adv. Appl., vol.{\bf 44},
Birkh\" auser-Verlag, Boston, 1990.

\bibitem[19]{GGK} I. Gohberg, S. Goldberg and M.A. Kaashoek, {\it Classes Linear Operators},
vol.{\bf II}, Birkh\" auser-Verlag, Boston, 1993.

\bibitem[20]{Ha1} P. R. Halmos, {\it Ten problems in Hilbert space},
Bull. Amer. Math. Soc., {\bf 76} (1970), 887--933.

\bibitem[21]{Ha2} P. R. Halmos, {\it Ten years in Hilbert space},
Integral Equations Operator Theory, {\bf 2} (1979), 529--564.

\bibitem[22]{McCYa} J. E. McCarthy and L. Yang, {\it Subnormal operators and quadrature domains},
Adv. Math. {\bf 127} (1997), 52--72.

\bibitem[23]{NaT} T. Nakazi and K. Takahashi, {\it Hyponormal Toeplitz operators and extremal problems of Hardy spaces}
Trans. Amer. Math. Soc. {\bf 338} (1993), 753--767.

\bibitem[24]{Ni} N. K. Nikolskii, {\it Treatise on the shift operator}, Springer, New York, 1986.

\bibitem[25]{Pru} B. Prunaru, {\it Invariant subspaces for polynomially hyponormal operators},
Proc. Amer. Math. Soc. {\bf 125} (1997), 1689--1691.

\bibitem[26]{Sun} S. Sun, {\it Bergman shift is not unitarily equivalent to a Toeplitz operator},
Kexue Tongbao(English Ed.) {\bf 28} (1983), 1027--1030.


\end{thebibliography}

%%%%%%%%%%%%%%%%%%%%%%%%%%%%%%%%%%%%%%%%%%%%%%%%%%%%%%%%%%%%%%%%%%%%%%%%%%%%%%%%%%%%%%%%%%%%%%%%%
\newpage\leftline{\large{\it 200}\ \ \ \small 학문
연구의 동향과 쟁점 -- 수학}

 \vspace{0.5cm}

\centerline{\bf Abstract}

\vspace{.3 cm}

\centerline{\large Bridge theory for Toeplitz operators}

\vspace{.3 cm} \centerline{\sc Woo Young Lee}

\vspace{0.4 cm} Toeplitz operators arise naturally in several fields
of mathematics and in a variety of problems in physics. \ Also the
theory of hyponormal and subnormal operators is an extensive and
highly developed area, which has made important contributions to a
number of problems in functional analysis, operator theory, and
mathematical physics. \ Thus, it becomes of central significance to
describe in detail hyponormality and subnormality for Toeplitz
operators. In this sense, the following question is challenging and
interesting:
$$
\hbox{Which Toeplitz operators are hyponormal or subnormal ?}
$$
While the precise relation between normality and subnormality has
been extensively studied, as have been the classes of subnormal and
hyponormal operators, the relative position of the class of
subnormals inside the classes of hyponormals is still far from being
well understood. We call it a `bridge theory' for operators to
explore the gap between hyponormality and subnormality for bounded
linear operators acting on an infinite dimensional complex Hilbert
(or Banach) space. In this survey note, we provide a bridge theory
for Toeplitz operators. This is originated from Halmos' Problem 5
(in 1970): $\hbox{\it Is every subnormal Toeplitz operator either
normal or analytic\,?}$ \ Even though Halmos's Problem 5 was, in
1984, answered in the negative by C. Cowen and J. Long, until now
researchers have been unable to characterize subnormal Toeplitz
operators in terms of their symbols. In this survey note we study
the subnormality and hyponormality of Toeplitz operators acting on
the vector-valued Hardy space $H^2_{\mathbf{C}^n}$ of the unit
circle.

\vspace{.3 cm}

{\footnotesize 2010 Mathematics Subject Classification: 47B20,
47B35, 46J15, 15A83, 30H10, 47A20.

\vspace{.2 cm}

\indent Key words and phrases: Hardy spaces, Toeplitz operators,
Hankel operators, normal, subnormal, hyponormal, $k$-hyponormal. }


\end{document}
\end


%%%%%%%%%%%%%%%%%%%%%%%%%%%%%%%%%%%%%%%%%%%%%%%%%%%%%%%%%%%%%%%%%%%%%%%%%%%%%%%%%%%%%%%%%%%%%%%%%
%%
%%
%%
%%                           END
%%
%%
%%%%%%%%%%%%%%%%%%%%%%%%%%%%%%%%%%%%%%%%%%%%%%%%%%%%%%%%%%%%%%%%%%%%%%%%%%%%%%%%%%%%%%%%%%%%%%%%%
